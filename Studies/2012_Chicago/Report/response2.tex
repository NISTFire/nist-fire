\documentclass[12pt]{article}
\usepackage[pdftex]{graphicx}
\frenchspacing
\begin{document}


\section{Comments from Reviewer 1}
\begin{itemize}
\item I have had the experience that FDS completely extinguishes the internal flaming due to the lack of oxygen, burning *all* the fuel at the vents. Surely this is not realistic. If there were no flames inside, no pyrolzate would be produced to sustain the external flaming. I do not doubt the existence of external flaming, and the qualitative comparison to the attic fire experiments (Fig. 5) is indeed very good. But I would have liked the authors to elaborate a little bit more on the internal flaming (or the lack of it), and how this affected the temperature evolution inside. \\
{\it The authors have added text which explains that there is little burning internal to the attic in the model because of a lack of oxygen. At the attic level the burning is external to the structure. On the porch level there is sufficient oxygen penetration to support combustion internal to the structure. The authors agree that FDS does not allow for combustion to occur without oxygen, but all of the fuel is not burned. That is evident by the HRR curve which is analyzed over the domain of the model. The authors disagree with the reviewer's statement that if there were no flames, no pyrolzate would be produced. Pyrolosis does not require oxygen. High temperatures/heat fluxes will pyrolze a material - the presence of oxygen is what drives flaming combustion - phenomena that are appropriately modeled here.}

\end{itemize}


\section{Comments from Reviewer 4}
\begin{itemize}
\item  On p. 2, line 27 of the revised text document (this is revised text document pdf page 11), it reads ``Appendix ?? ...'' \\
{\it This was my error in Latex and been since fixed. Thank you. I am not highlighting the change in red on the updated document because the ?? should be fixed to a reference number.}
\end{itemize}

\end{document}
