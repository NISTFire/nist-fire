\documentclass[12pt,oneside]{book}

%%%%%%%%%%%%%%%%%%%%%%%%%%%%%%%%%%%%%%%%%%%%%%%%%%%%%%%%%%%%%%%%%%%%%%%%%%%%%%%%%%%%%%%%%%%%%%%%%%%
%                                                                                                 %
% The mathematical style of these documents follows                                               %
%                                                                                                 %
% A. Thompson and B.N. Taylor. The NIST Guide for the Use of the International System of Units.   %
%    NIST Special Publication 881, 2008.                                                          %
%                                                                                                 %
% http://www.nist.gov/pml/pubs/sp811/index.cfm                                                    %
%                                                                                                 %
%%%%%%%%%%%%%%%%%%%%%%%%%%%%%%%%%%%%%%%%%%%%%%%%%%%%%%%%%%%%%%%%%%%%%%%%%%%%%%%%%%%%%%%%%%%%%%%%%%%

\input{../../../../../Bibliography/commoncommands}

% Load Packages
\usepackage{placeins}

% Rename chapter headings
\renewcommand{\chaptername}{Section}
\renewcommand{\bibname}{References}

% Math shortcuts
\renewcommand{\sb}[1]{_\mathrm{#1}}
\renewcommand{\C}{\mbox{C}}
\renewcommand{\H}{\mbox{H}}
\renewcommand{\O}{\mbox{O}}
\newcommand{\N}{\mbox{N}}

% Center all figures
\makeatletter
\g@addto@macro\@floatboxreset\centering
\makeatother

\begin{document}

\bibliographystyle{unsrt}
\pagestyle{empty}

\begin{minipage}[t][9in][s]{6.25in}

\headerB{
Overview of Fire Model Validation Experiments in Residential Scale Structures
}

\headerC{
\flushright{
Joseph Willi \\
Kevin McGrattan \\
Randall McDermott \\
Craig Weinschenk \\
{\em Fire Research Division \\
Engineering Laboratory \\
Gaithersburg, Maryland, USA} \\ }
}

\flushright{\today \\
}

\vfill

\flushright{
\includegraphics[width=2.in]{../../../../../Bibliography/nistident_flright_vec} \\[.3in]
}

\titlesigs

\end{minipage}

\newpage

\frontmatter

\pagestyle{plain}
\pagenumbering{roman}

\cleardoublepage
\phantomsection
\addcontentsline{toc}{chapter}{Contents}
\tableofcontents

\cleardoublepage
\phantomsection
\addcontentsline{toc}{chapter}{List of Figures}
\listoffigures

\cleardoublepage
\phantomsection
\addcontentsline{toc}{chapter}{List of Tables}
\listoftables

\chapter{List of Acronyms}

\begin{tabbing}
\hspace{1.5in} \= \\
FDS \> Fire Dynamics Simulator \\
HGL \> Hot Gas Layer \\
HRR \> Heat Release Rate \\
HRRPUA \> Heat Release Rate per Unit Area \\
NIST \> National Institute of Standards and Technology \\
\end{tabbing}

\mainmatter

% ================
% = Introduction =
% ================
\chapter{Introduction}
\label{chap:Introduction}

% ======================
% = EXPERIMENTAL SETUP =
% ======================
\chapter{Experimental Setup}
\label{chap:Experimental_Setup}
The series of field experiments described in this report were conducted in two structures of similar design located at the Delaware County Emergency Services Training Center in Sharon Hill, PA. The fire source for all experiments was generated by three propane burners and various sensors were used to collect gas temperature, gas velocity, heat flux, and gas concentration measurements throughout the structure.

\section{Test Structures}
\label{sec:Test_Structures}

\subsection{Construction}
\label{sec:construction}
Each test structure was built on a concrete slab as shown in Fig.~\ref{fig:struct_pics}. The East Structure was designed to simulate a single-story residential structure, and the West Structure was designed to simulate a two-story residential structure. The first floor of each structure had an outer wall composed of interlocking concrete blocks with equal side lengths of 0.61~m (2~ft). The joints and gaps between the blocks were filled with high temperature insulation.

\begin{figure}[!ht]
	\includegraphics[width=5.25in]{../../Hose_Stream_Report/Figures/Pictures/east_structure}
	\\~\\
	\includegraphics[width=5.25in]{../../Hose_Stream_Report/Figures/Pictures/west_structure}
	\caption[North side of the East and West Structures.]{North side of the East (top) and West (bottom) Test Structures.}
	\label{fig:struct_pics}
\end{figure}

The interior walls of the first floor of each structure were framed with steel studs set to 0.40~m (16~in) centers and track and \textit{were lined with 13~mm (0.5~in) thick cement board. The walls were composed of 16~mm (5/8~in) Type X gypsum board. Additionally, the ceiling was composed of two layers of 13~mm (0.5~in) thick cement board.}
\FloatBarrier

The first floor ceiling support of each structure was composed of wood truss joist I-beams (TJIs) with a 299~mm (11.75~in) depth. Each TJI was composed of laminated veneer lumber flanges with a cross section of 29~mm (1.13~in) x 44~mm (1.75~in) and an 11~mm (0.43~in) thick oriented strand board web as shown in Fig.~\ref{fig:TJI}. Tongue and grove, 18.3~mm (0.72~in) thick, oriented strand board was screwed to the top of the TJIs.

\begin{figure}[!ht]
	\includegraphics[width=6in]{../../Hose_Stream_Report/Figures/Pictures/TJI_support}
	\caption[TJI-constructed ceiling support of the West Structure.]{First floor ceiling support of the West Structure composed of wood truss joist I-beams. View is of the southeast corner of the structure.}
	\label{fig:TJI}
\end{figure}
\FloatBarrier

The second floor of the West Structure was built on the wood ceiling support described above and was connected to the first floor by a stairwell. The second story walls were wood framed with 51~mm (2~in) by 102~mm (4~in) studs set to 0.40~m (16~in) centers. The interior walls were protected by 16~mm (5/8~in) fire rated gypsum board, 16~mm (5/8~in) durarock board, and a second layer of 16~mm (5/8~in) fire rated gypsum board. The exterior walls were protected with 11~mm (7/16~in) oriented strand board and 8~mm (5/16~in) fiber cement lap siding.

% Add info about roof vent in single story
\subsection{Layout}
\label{sec:layout}
Dimensioned floor plans of the East and West Structures are presented in Figures~\ref{fig:east_dimensioned_plan}~and~\ref{fig:west_dimensioned_plan}, respectively.

\begin{figure}[!ht]
	\includegraphics[width=\columnwidth]{../Figures/Floor_Plans/East_Structure_Dimensioned_Full}
	\caption[Dimensioned floor plan of the East Structure.]{East Structure floor layout.}
	\label{fig:east_dimensioned_plan}
\end{figure}
% make new floor plan with ceiling vent included

\begin{figure}[!ht]
	\includegraphics[width=\columnwidth]{..//Figures/Floor_Plans/West_Structure_2nd_Floor_Dimensioned_Full}
	\\~\\
	\includegraphics[width=\columnwidth]{../Figures/Floor_Plans/West_Structure_1st_Floor_Dimensioned_Full}
	\caption[Dimensioned floor plan of the first and second floors of the West Structure.]{Dimensioned floor plan of the second floor (top) and first floor (bottom) of the West Structure.}
	\label{fig:west_dimensioned_plan}
\end{figure}
% make new 1st floor plan with rear door completely sealed off

The interior dimensions of each structure were approximately 6.1~m (20~ft) wide, 11~m (36~ft) long and 2.4~m (8~ft) high. The stairs connecting the two floors of the West Structure started 1.6~m (5.25~ft) off the south wall with a width of 1.2~m (4~ft) off the east wall.

The exterior doorways of each structure and the stairwell doorway on the second level of the West Structure all contained doors that were opened or closed at certain instances to change the ventilation patterns of the structure. All other doorways in the structures did not contain a door, so to close the doorway, a sheet of gypsum board was used to cover the opening and the doorway remained closed for the duration of the test procedure.
\FloatBarrier

\section{Instrumentation}
\label{sec:Instrumentation}

\subsection{East Structure}
\label{sec:East_instrument}

\begin{figure}[!ht]
	\includegraphics[width=\columnwidth]{../Figures/Floor_Plans/East_Structure_Dimensioned_Instrumentation}
	\caption[Dimensioned floor plan of the East Structure.]{East Structure floor layout.}
	\label{fig:east_instrumentation}
\end{figure}

\subsection{West Structure}
\label{sec:West_instruments}

\begin{figure}[!ht]
	\includegraphics[width=\columnwidth]{../Figures/Floor_Plans/West_Structure_2nd_Floor_Dimensioned_Instrumentation}
	\\~\\
	\includegraphics[width=\columnwidth]{../Figures/Floor_Plans/West_Structure_1st_Floor_Dimensioned_Instrumentation}
	\caption[Dimensioned floor plan of the first and second floors of the West Structure.]{Dimensioned floor plan of the second floor (top) and first floor (bottom) of the West Structure.}
	\label{fig:west_instrumentation}
\end{figure}

\subsection{Uncertainty}
\label{sec:Uncertainty}

\section{Propane Burners}
\label{sec:Burners}
Three propane burners, each with a 0.6 m X 0.6 m square opening 0.14 m above the floor, were used as the fuel source in every experiment described in this report. Propane flowed at a known volumetric flow rate during the experiments.

% ==========================
% = EXPERIMENTAL PROCEDURE =
% ==========================
\chapter{Experimental Procedure}
\label{chap:Experimental_Procedure}
A similar procedure was followed for all experiments described in this report. First, three propane burners were ignited in sequential order. Next, different doors and vents in the structure were opened and closed to change the ventilation pattern within the structure. Finally, the burners were turned off in sequential order and the fire was extinguished. 

\section{East Structure}
\label{sec:East_exps}
Three different test series (Test 4, Test 5, and Test 6) were conducted in the east structure. Each test series was composed of three experiments that used an identical procedure to change the ventilation pattern in the structure. The three procedures used during each test series are outlined in Figures~\ref{fig:Test_4_procedure}-\ref{fig:Test_6_procedure}.

% \begin{figure}[!ht]
% \includegraphics[width=6in]{../Drawings/East_Structure_Dimensioned}
% \caption[Plan view of the East Structure.]{East Structure floor layout.}
% \label{fig:Test_4_procedure}
% \end{figure}
% \FloatBarrier

\begin{figure}[!ht]
	\includegraphics[width=\columnwidth]{../Figures/Floor_Plans/East_Structure_Test_4}
	\caption[Dimensioned floor plan of the East Structure.]{East Structure floor layout.}
	\label{fig:east_test_4}
\end{figure}

\begin{figure}[!ht]
	\includegraphics[width=\columnwidth]{../Figures/Floor_Plans/East_Structure_Test_5}
	\caption[Dimensioned floor plan of the East Structure.]{East Structure floor layout.}
	\label{fig:east_test_5}
\end{figure}

\begin{figure}[!ht]
	\includegraphics[width=\columnwidth]{../Figures/Floor_Plans/East_Structure_Test_6}
	\caption[Dimensioned floor plan of the East Structure.]{East Structure floor layout.}
	\label{fig:east_test_6}
\end{figure}

\begin{landscape}
\begin{table}[!ht]
\caption{East Structure experimental event times (mm:ss)}
\begin{tabular}{lcccccccccccccc}
 \toprule
\textbf{Test} & 
\multicolumn{2}{c}{\textbf{\underline{Corner Burner}}} & 
\multicolumn{2}{c}{\textbf{\underline{Middle Burner}}} & 
\multicolumn{2}{c}{\textbf{\underline{Center Burner}}} & 
\multicolumn{2}{c}{\textbf{\underline{W Double Door}}} & 
\multicolumn{2}{c}{\textbf{\underline{E Double Door}}} & 
\multicolumn{2}{c}{\textbf{\underline{Single Door}}} & 
\multicolumn{2}{c}{\textbf{\underline{Roof Vent}}}
\\
\textbf{Number} & 
\textbf{On} & \textbf{Off} & \textbf{On} & \textbf{Off} & \textbf{On} & \textbf{Off} & 
\textbf{Close} & \textbf{Open} & \textbf{Close} & \textbf{Open} &
\textbf{Close} & \textbf{Open} & \textbf{Close} & \textbf{Open}
\\
\midrule
% Test 2
2 & 0:20 & 16:20 & 3:20 & 14:20 & 6:20 & 12:20 & 
N/A & 7:20 & N/A & 9:20 & N/A & 10:30 & N/A & N/A \\
% Test 3
3 & 0:20 & 17:20 & 3:20 & 15:20 & 6:20 & 13:20 & 
N/A & 7:20 & N/A & 9:20 & N/A & 10:20 & N/A & N/A \\
% Test 4
4 & 0:20 & 17:20 & 3:20 & 15:20 & 6:20 & 13:20 & 
N/A & 7:20 & N/A & 9:20 & N/A & 10:20 & N/A & N/A 
\\ \multicolumn{15}{c}{ } \\
% Test 5a
5a & 0:15 & 9:55 & 0:30 & 9:35 & 0:45 & 9:15 & 
N/A & 3:18 & N/A & 6:18 & N/A & N/A & 7:45 & 2:53 \\
% Test 5b
5b & 0:15 & 10:30 & 0:30 & 10:15 & 0:45 & 10:00 & 
N/A & 3:45 & N/A & 5:15 & N/A & N/A & 8:30 & 2:15 \\
% Test 5c
5c & 0:15 & 9:45 & 0:30 & 9:30 & 0:45 & 9:15 & 
N/A & 3:45 & N/A & 5:16 & N/A & N/A & 7:28 & 2:15
\\ \multicolumn{15}{c}{ } \\
% Test 6a
6a & 0:15 & 9:55 & 0:30 & 9:35 & 0:45 & 9:15 & 
N/A & 3:18 & N/A & N/A & N/A & N/A & N/A & 2:53 \\
% Test 6b
6b & 0:15 & 9:55 & 0:30 & 9:35 & 0:45 & 9:15 & 
N/A & 3:18 & N/A & N/A & N/A & N/A & N/A & 2:53 \\
% Test 6c
6c & 0:15 & 9:55 & 0:30 & 9:35 & 0:45 & 9:15 & 
N/A & 3:18 & N/A & N/A & N/A & N/A & N/A & 2:53 \\
\bottomrule
\end{tabular}
\label{table:east_exp_times}
\end{table}
\end{landscape}

\section{West Structure}
\label{sec:West_exps}

\chapter{Acknowledgments}
\label{chap:Acknowledgments}

\bibliography{../../../Bibliography/FDS_refs,../../../Bibliography/FDS_general,references}

\appendix

\chapter{Appendix A}

Placeholder


\end{document}
