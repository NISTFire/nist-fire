\documentclass[12pt,oneside]{book}

%%%%%%%%%%%%%%%%%%%%%%%%%%%%%%%%%%%%%%%%%%%%%%%%%%%%%%%%%%%%%%%%%%%%%%%%%%%%%%%%%%%%%%%%%%%%%%%%%%%
%                                                                                                 %
% The mathematical style of these documents follows                                               %
%                                                                                                 %
% A. Thompson and B.N. Taylor. The NIST Guide for the Use of the International System of Units.   %
%    NIST Special Publication 881, 2008.                                                          %
%                                                                                                 %
% http://www.nist.gov/pml/pubs/sp811/index.cfm                                                    %
%                                                                                                 %
%%%%%%%%%%%%%%%%%%%%%%%%%%%%%%%%%%%%%%%%%%%%%%%%%%%%%%%%%%%%%%%%%%%%%%%%%%%%%%%%%%%%%%%%%%%%%%%%%%%

\input{../../../../../Bibliography/commoncommands}

% Load Packages
\usepackage{graphicx}
\usepackage{placeins}
\usepackage{lscape}

% Rename chapter headings
\renewcommand{\chaptername}{Section}
\renewcommand{\bibname}{References}

% Math shortcuts
\renewcommand{\sb}[1]{_\mathrm{#1}}
\renewcommand{\C}{\mbox{C}}
\renewcommand{\H}{\mbox{H}}
\renewcommand{\O}{\mbox{O}}
\newcommand{\N}{\mbox{N}}

% Center all figures
\makeatletter
\g@addto@macro\@floatboxreset\centering
\makeatother

\begin{document}

\bibliographystyle{unsrt}
\pagestyle{empty}

\begin{minipage}[t][9in][s]{6.25in}

\headerB{
Overview of Fire Model Validation Experiments in Residential Scale Structures
}

\headerC{
\flushright{
Joseph Willi \\
Kevin McGrattan \\
Randall McDermott \\
Craig Weinschenk \\
{\em Fire Research Division \\
Engineering Laboratory \\
Gaithersburg, Maryland, USA} \\ }
}

\flushright{\today \\
}

\vfill

\flushright{
\includegraphics[width=2.in]{../../../../../Bibliography/nistident_flright_vec} \\[.3in]
}

\titlesigs

\end{minipage}

\newpage

\frontmatter

\pagestyle{plain}
\pagenumbering{roman}

\cleardoublepage
\phantomsection
\addcontentsline{toc}{chapter}{Contents}
\tableofcontents

\cleardoublepage
\phantomsection
\addcontentsline{toc}{chapter}{List of Figures}
\listoffigures

\cleardoublepage
\phantomsection
\addcontentsline{toc}{chapter}{List of Tables}
\listoftables

\chapter{List of Acronyms}

\begin{tabbing}
\hspace{1.5in} \= \\
FDS \> Fire Dynamics Simulator \\
HGL \> Hot Gas Layer \\
HRR \> Heat Release Rate \\
HRRPUA \> Heat Release Rate per Unit Area \\
NIST \> National Institute of Standards and Technology \\
\end{tabbing}

\mainmatter

\chapter{Introduction}
\label{chap:Introduction}

% ======================
% = EXPERIMENTAL SETUP =
% ======================
\chapter{Experimental Setup}
\label{chap:Experimental_Setup}
The series of field experiments described in this report were conducted in two structures of similar design located at the Delaware County Emergency Services Training Center in Sharon Hill, PA. The fire source for all experiments was generated by three propane burners and various sensors were used to collect gas temperature, gas velocity, heat flux, and gas concentration measurements throughout the structure.

\section{Test Structures}
\label{sec:Test_Structures}
Each test structure was built on a concrete slab as shown in Fig.~\ref{fig:struct_pics}. The structures were designed to simulate a single-story and a two-story residential structure. The first floor of each structure had an outer wall composed of interlocking concrete blocks with equal side lengths of 0.61 m (2 ft). The joints and gaps between the blocks were filled with high temperature insulation.

\begin{figure}[!ht]
\includegraphics[width=6in]{../../Hose_Stream_Report/Pictures/east_structure}
\\~\\
\includegraphics[width=6in]{../../Hose_Stream_Report/Pictures/west_structure}
\caption[North side of the East and West Structures.]{North side of the East (top) and West (bottom) Test Structures.}
\label{fig:struct_pics}
\end{figure}

The interior walls of the first floor of each structure were framed with steel studs and track. The studs were set to 0.40 m (16 in.) centers. The ceiling/floor support was composed of wood truss joist I-beams (TJIs) with a 299 mm (11.75 in.) depth. Each TJI was composed of laminated veneer lumber flanges with a cross section of 29 mm (1.13 in.) x 44 mm (1.75 in.) and an 11 mm (0.43 in.) thick oriented strand board web as shown in Fig.~\ref{fig:TJI}. Tongue and grove, 18.3 mm (0.72 in.) thick, oriented strand board was screwed to the top of the TJIs.

\begin{figure}[!ht]
\includegraphics[width=6in]{../../Hose_Stream_Report/Pictures/TJI_support}
\caption[TJI-constructed ceiling/floor support of the West Structure.]{Ceiling and floor support of the West Structure composed of wood truss joist I-beams. View is of the east side of the structure.}
\label{fig:TJI}
\end{figure}
\FloatBarrier

\subsection{East Structure}
\label{sec:East_descr}

\subsection{West Structure}
\label{sec:West_descr}

\section{Instrumentation}
\label{sec:Instrumentation}

\subsection{East Structure}
\label{sec:East_instrument}

\subsection{West Structure}
\label{sec:West_instruments}

\subsection{Uncertainty}
\label{sec:Uncertainty}


% ==========================
% = EXPERIMENTAL PROCEDURE =
% ==========================
\chapter{Experimental Procedure}
\label{chap:Experimental_Procedure}
A similar procedure was followed for all experiments described in this report. First, three propane burners were ignited in sequential order. Next, different doors and vents in the structure were opened and closed to change the ventilation pattern within the structure. Finally, the burners were turned off in sequential order and the fire was extinguished. 

\section{East Structure}
\label{sec:East_exps}
Three different test series (Test 4, Test 5, and Test 6) were conducted in the east structure. Each test series was composed of three experiments that used an identical procedure to change the ventilation pattern in the structure. The three procedures used during each test series are outlined in Figures~\ref{fig:Test_4_procedure}-\ref{fig:Test_6_procedure}.

% \begin{figure}[!ht]
% \includegraphics[width=6in]{../Drawings/East_Test_Structure_Dimensioned}
% \caption[Plan view of the East Structure.]{East Structure floor layout.}
% \label{fig:Test_4_procedure}
% \end{figure}
% \FloatBarrier

\begin{landscape}
\begin{table}[!ht]
\caption{East Structure experimental event times (mm:ss)}
\begin{tabular}{lcccccccccccccc}
 \toprule
\textbf{Test} & 
\multicolumn{2}{c}{\textbf{\underline{Corner Burner}}} & 
\multicolumn{2}{c}{\textbf{\underline{Middle Burner}}} & 
\multicolumn{2}{c}{\textbf{\underline{Center Burner}}} & 
\multicolumn{2}{c}{\textbf{\underline{W Double Door}}} & 
\multicolumn{2}{c}{\textbf{\underline{E Double Door}}} & 
\multicolumn{2}{c}{\textbf{\underline{Single Door}}} & 
\multicolumn{2}{c}{\textbf{\underline{Roof Vent}}}
\\
\textbf{Number} & 
\textbf{On} & \textbf{Off} & \textbf{On} & \textbf{Off} & \textbf{On} & \textbf{Off} & 
\textbf{Close} & \textbf{Open} & \textbf{Close} & \textbf{Open} &
\textbf{Close} & \textbf{Open} & \textbf{Close} & \textbf{Open}
\\
\midrule
% Test 2
2 & 0:20 & 16:20 & 3:20 & 14:20 & 6:20 & 12:20 & 
N/A & 7:20 & N/A & 9:20 & N/A & 10:30 & N/A & N/A \\
% Test 3
3 & 0:20 & 17:20 & 3:20 & 15:20 & 6:20 & 13:20 & 
N/A & 7:20 & N/A & 9:20 & N/A & 10:20 & N/A & N/A \\
% Test 4
4 & 0:20 & 17:20 & 3:20 & 15:20 & 6:20 & 13:20 & 
N/A & 7:20 & N/A & 9:20 & N/A & 10:20 & N/A & N/A 
\\ \multicolumn{15}{c}{ } \\
% Test 5a
5a & 0:15 & 9:55 & 0:30 & 9:35 & 0:45 & 9:15 & 
N/A & 3:18 & N/A & 6:18 & N/A & N/A & 7:45 & 2:53 \\
% Test 5b
5b & 0:15 & 10:30 & 0:30 & 10:15 & 0:45 & 10:00 & 
N/A & 3:45 & N/A & 5:15 & N/A & N/A & 8:30 & 2:15 \\
% Test 5c
5c & 0:15 & 9:45 & 0:30 & 9:30 & 0:45 & 9:15 & 
N/A & 3:45 & N/A & 5:16 & N/A & N/A & 7:28 & 2:15
\\ \multicolumn{15}{c}{ } \\
% Test 6a
6a & 0:15 & 9:55 & 0:30 & 9:35 & 0:45 & 9:15 & 
N/A & 3:18 & N/A & N/A & N/A & N/A & N/A & 2:53 \\
% Test 6b
6b & 0:15 & 9:55 & 0:30 & 9:35 & 0:45 & 9:15 & 
N/A & 3:18 & N/A & N/A & N/A & N/A & N/A & 2:53 \\
% Test 6c
6c & 0:15 & 9:55 & 0:30 & 9:35 & 0:45 & 9:15 & 
N/A & 3:18 & N/A & N/A & N/A & N/A & N/A & 2:53 \\
\bottomrule
\end{tabular}
\label{table:east_exp_times}
\end{table}
\end{landscape}

\section{West Structure}
\label{sec:West_exps}

\chapter{Acknowledgments}
\label{chap:Acknowledgments}

\bibliography{../../../Bibliography/FDS_refs,../../../Bibliography/FDS_general,references}

\appendix

\chapter{Appendix A}

Placeholder


\end{document}
