\documentclass[12pt,oneside]{book}

%%%%%%%%%%%%%%%%%%%%%%%%%%%%%%%%%%%%%%%%%%%%%%%%%%%%%%%%%%%%%%%%%%%%%%%%%%%%%%%%%%%%%%%%%%%%%%%%%%%
%                                                                                                 %
% The mathematical style of these documents follows                                               %
%                                                                                                 %
% A. Thompson and B.N. Taylor. The NIST Guide for the Use of the International System of Units.   %
%    NIST Special Publication 881, 2008.                                                          %
%                                                                                                 %
% http://www.nist.gov/pml/pubs/sp811/index.cfm                                                    %
%                                                                                                 %
%%%%%%%%%%%%%%%%%%%%%%%%%%%%%%%%%%%%%%%%%%%%%%%%%%%%%%%%%%%%%%%%%%%%%%%%%%%%%%%%%%%%%%%%%%%%%%%%%%%

\input{../../../../../Bibliography/commoncommands}

% Load Packages
\usepackage{placeins}

% Rename chapter headings
\renewcommand{\chaptername}{Section}
\renewcommand{\bibname}{References}

% Math shortcuts
\renewcommand{\sb}[1]{_\mathrm{#1}}
\renewcommand{\C}{\mbox{C}}
\renewcommand{\H}{\mbox{H}}
\renewcommand{\O}{\mbox{O}}
\newcommand{\N}{\mbox{N}}

% Center all figures
\makeatletter
\g@addto@macro\@floatboxreset\centering
\makeatother

\begin{document}

\bibliographystyle{unsrt}
\pagestyle{empty}
\begin{minipage}[t][9in][s]{6.25in}

\begin{flushright}
\fontsize{20}{24}\selectfont
\bf{NIST Technical Note XXXX}
\end{flushright}

\headerB{
Propane Gas Fire Experiments in Residential Scale Structures \\
}

\normalsize

\headerC{
{
\flushright{

\vspace*{2\baselineskip}

\begingroup
This publication is available free of charge from:
\hypersetup{urlcolor=black}
\href{http://dx.doi.org/10.6028/NIST.TN.XXXX}{http://dx.doi.org/10.6028/NIST.TN.XXXX}
\endgroup
}

\vfill

\flushright{

\includegraphics[width=2.in]{../../../../../Bibliography/nistident_flright_vec} \\[.3in]
}
}
}

\end{minipage}

\newpage
\hspace{5in}
\newpage

\frontmatter

\pagenumbering{roman}

\begin{minipage}[t][9in][s]{6.25in}

\begin{flushright}
\fontsize{20}{24}\selectfont
\bf{NIST Technical Note XXXX}
\end{flushright}

\headerB{
Propane Gas Fire Experiments in Residential Scale Structures \\
}

\headerC{
\flushright{

{\em Fire Research Division \\
Engineering Laboratory} \\

\vspace*{2\baselineskip}

\begingroup
This publication is available free of charge from:
\hypersetup{urlcolor=black}
\href{http://dx.doi.org/10.6028/NIST.TN.XXXX}{http://dx.doi.org/10.6028/NIST.TN.XXXX} \\
\endgroup

\vspace*{2\baselineskip}
February 2016}}

\vfill

\flushright{\includegraphics[width=1in]{../../../../../Bibliography/doc}}

\titlesigs

\end{minipage}

\newpage

\begin{minipage}[t][9in][s]{6.25in}

\flushright{Certain commercial entities, equipment, or materials may be identified in this \\
document in order to describe an experimental procedure or concept adequately. \\
Such identification is not intended to imply recommendation or endorsement by the \\
National Institute of Standards and Technology, nor is it intended to imply that the \\
entities, materials, or equipment are necessarily the best available for the purpose. \\
}

\vspace{3in}

\large
\flushright{\bf National Institute of Standards and Technology Technical Note XXXX \\
Natl.~Inst.~Stand.~Technol.~Tech.~Note~XX, \pageref{LastPage} pages (February 2016) \\
% http://dx.doi.org/10.6028/NIST.TN.XXXX \\
CODEN: NTNOEF }

\vspace{0.2in}

\begingroup
{\bf This publication is available free of charge from:}
\hypersetup{urlcolor=black}
\href{http://dx.doi.org/10.6028/NIST.TN.1838}{\bf http://dx.doi.org/10.6028/NIST.TN.1838} \\
\endgroup

\vfill

\hspace{1in}

\end{minipage}

\newpage

\frontmatter

\pagestyle{plain}
\pagenumbering{roman}

\cleardoublepage
\phantomsection
\addcontentsline{toc}{chapter}{Contents}
\tableofcontents

\cleardoublepage
\phantomsection
\addcontentsline{toc}{chapter}{List of Figures}
\listoffigures

% \cleardoublepage
% \phantomsection
% \addcontentsline{toc}{chapter}{List of Tables}
% \listoftables

\chapter{List of Acronyms}

\begin{tabbing}
\hspace{1.5in} \= \\
FDS \> Fire Dynamics Simulator \\
HGL \> Hot Gas Layer \\
HRR \> Heat Release Rate \\
HRRPUA \> Heat Release Rate per Unit Area \\
NIST \> National Institute of Standards and Technology \\
\end{tabbing}

\mainmatter

% ================
% = Introduction =
% ================
\chapter{Introduction}
\label{chap:Introduction}

% [Brief Introduction of FRD at NIST, ventilation, etc.]

This report describes eight full-scale fire experiments in which the effects of changing the ventilation patterns within a structure were studied. The experiments were conducted in two structures designed to replicate typical residential dwellings. During each test, propane was provided to three diffusion flame burners at a measured flow rate. Local measurements of temperature, gas velocity, heat flux, and gas concentrations were collected at various locations throughout the structure while ventilation patterns within the structure were changed by opening and closing certain doors and vents.


% ======================
% = EXPERIMENTAL SETUP =
% ======================
\chapter{Experimental Setup}
\label{chap:Experimental_Setup}
The series of field experiments described in this report were conducted in two structures of similar design located at the Delaware County Emergency Services Training Center in Sharon Hill, PA. The fire source for all experiments was generated by three propane burners, and various sensors were used to collect gas temperature, gas velocity, heat flux, and gas concentration measurements throughout the structure.

\section{Test Structures}
\label{sec:Test_Structures}

\subsection{Construction}
\label{sec:construction}
Each test structure was built on a concrete slab as shown in Fig.~\ref{fig:struct_pics}. The East Structure was designed to simulate a single-story residential structure, and the West Structure was designed to simulate a two-story residential structure. The first floor of each structure had an outer wall composed of interlocking concrete blocks with equal side lengths of 0.61~m (2~ft). The joints and gaps between the blocks were filled with high temperature insulation.

\begin{figure}[!ht]
	\includegraphics[width=5.25in]{../../Hose_Stream_Tests/Figures/Pictures/east_structure}
	\\~\\
	\includegraphics[width=5.25in]{../../Hose_Stream_Tests/Figures/Pictures/west_structure}
	\caption[North side of the East and West Structures.]{North side of the East (top) and West (bottom) test structures.}
	\label{fig:struct_pics}
\end{figure}

The interior walls of the first floor of each structure were framed with steel studs set to 0.40~m (16~in) centers and track and were lined with 13~mm (0.5~in) thick cement board. The walls were composed of 16~mm (0.625~in) Type X gypsum board. Additionally, the ceiling was composed of two layers of 13~mm (0.5~in) thick cement board.
\FloatBarrier

The first floor ceiling support of each structure was composed of wood truss joist I-beams (TJIs) with a 0.30~m (11.75~in) depth. Each TJI was composed of laminated veneer lumber flanges with a cross section of 29~mm (1.125~in) x 44~mm (1.75~in) and an 11~mm (0.43~in) thick oriented strand board web as shown in Fig.~\ref{fig:TJI}. Tongue and grove, 18.3~mm (0.72~in) thick, oriented strand board was screwed to the top of the TJIs.

\begin{figure}[!ht]
	\includegraphics[width=6in]{../../Hose_Stream_Tests/Figures/Pictures/TJI_support}
	\caption[Ceiling support of the West Structure.]{First floor ceiling support of the West Structure composed of wood truss joist I-beams. View is of the southeast corner of the structure.}
	\label{fig:TJI}
\end{figure}
\FloatBarrier

%SIG FIGS ON DIMENSIONS - ROUND COMMON DECIMALS? Example: 0.625 -> 0.6; 0.75 -> 0.8? They could also be changed to fractions

The second floor of the West Structure was built on the wood ceiling support described above and was connected to the first floor by a stairwell. The second story walls were wood framed with 51~mm (2~in) by 102~mm (4~in) studs set to 0.40~m (16~in) centers. The interior walls were protected by 16~mm (0.625~in) fire rated gypsum board, 16~mm (0.625~in) durarock board, and a second layer of 16~mm (0.625~in) fire rated gypsum board. The exterior walls were protected with 11~mm (0.44~in) oriented strand board and 8~mm (0.31~in) fiber cement lap siding.

\subsection{Layout}
\label{sec:layout}
Dimensioned floor plans of the East and West Structures are presented in Figures~\ref{fig:east_dimensioned_plan}~and~\ref{fig:west_dimensioned_plan}, respectively.

\begin{figure}[!ht]
	\includegraphics[width=\columnwidth]{../Figures/Floor_Plans/East_Structure_Dimensioned_Full}
	\caption[Dimensioned floor plan of the East Structure.]{Dimensioned floor plan of the East Structure. Structure is symmetric across horizontal centerline.}
	\label{fig:east_dimensioned_plan}
\end{figure}

\begin{figure}[!ht]
	\includegraphics[width=\columnwidth]{../Figures/Floor_Plans/West_Structure_2nd_Floor_Dimensioned_Full}
	\\~\\
	\includegraphics[width=\columnwidth]{../Figures/Floor_Plans/West_Structure_1st_Floor_Dimensioned_Full}
	\caption[Dimensioned floor plans of the West Structure.]{Dimensioned floor plan of the second floor (top) and first floor (bottom) of the West Structure.}
	\label{fig:west_dimensioned_plan}
\end{figure}

The interior dimensions of each structure were approximately 6.1~m (20~ft) wide, 11~m (36~ft) long, and 2.4~m (8~ft) high. The stairs connecting the two floors of the West Structure started 1.6~m (5.25~ft) off the south wall with a width of 1.2~m (4~ft) off the east wall and a 0.18~m (7.25~in) rise and 0.19~m (7.5~in) rise. The exterior doors of both structures, the stairwell door on the second level of the West Structure, and the square roof vent of the East Structure with side length 1.2~m (4~ft) and a depth of 0.32~m (12.75~in) were opened and closed at certain instances during experiments to change the ventilation patterns within the structures.
\FloatBarrier

\section{Instrumentation}
\label{sec:Instrumentation}
The structures were instrumented for temperature, gas velocity, heat flux, and gas concentration measurements. Gas temperatures in the burn rooms were measured with bare-bead, Chromel-Alumel (type K) thermocouples. Additional single thermocouples were installed in conjunction with bi-directional probes for gas velocity measurements. The single thermocouples were bare-bead, Chromel-Alumel (type K) thermocouples with a 1.0~mm (0.04~in) nominal diameter. The thermocouple wire was protected with an 3.2~mm (0.125~in) diameter inconel sheath. Schmidt-Boelter gauges were used to measure both total heat flux and radiant heat flux (radiometer). A radiometer is a total heat flux gauge with a zirconium plate to prevent contributions from convective heat transfer. Calibrated pumps pulled gas samples through a sample conditioning system to eliminate moisture in the sample. Then, the dry gas sample was piped to a series of gas analyzers and the concentrations of oxygen, carbon monoxide, and carbon dioxide were measured. A legend is presented in Fig.~\ref{fig:Instrumentation_Legend} to clarify the instrumentation schematics presented in the follow sections.

\begin{figure}[!ht]
	\includegraphics[width=0.25\columnwidth]{../Figures/Floor_Plans/Instrumentation_Legend}
	\caption[Instrumentation legend.]{Legend used for schematics of instrumentation locations.}
	\label{fig:Instrumentation_Legend}
\end{figure}

Three propane burners, pictured in Fig.~\ref{fig:burners}, were used as the fuel source in each experiment. Each burner had a square opening of side length 0.6~m (2~ft) located 0.14~m (5.5~in) above the floor and were positioned 0.6~m (2~ft) from the south and west walls on the first floor of each structure. Propane was supplied to the burners at a measured flow rate during all experiments.
% INCLUDE HRR CALCULATIONS HERE

\begin{figure}[!ht]
	\includegraphics[width=0.9\columnwidth]{../Figures/Pictures/burners}
	\caption[Three propane burners.]{Three propane burners used as the fuel source for the experiments at a position of 0.6~m (2~ft) off the south and west walls of the East Structure.}
	\label{fig:burners}
\end{figure}

\FloatBarrier

\subsection{East Structure}
The East Structure was instrumented with five bare-bead thermocouple arrays, four bi-directional probe plus solid thermocouple arrays, five total heat flux plus radiometer sensor pairs, and two gas sample inlet pipes at the locations shown in Fig.~\ref{fig:east_instrumentation}.

\begin{figure}[!ht]
	\includegraphics[width=\columnwidth]{../Figures/Floor_Plans/East_Structure_Dimensioned_Instrumentation}
	\caption{Location of instrumentation in the East Structure.}
	\label{fig:east_instrumentation}
\end{figure}
\FloatBarrier

Each bare-bead thermocouple array was composed of eight thermocouples. Three bi-directional probe and solid thermocouple arrays were centered in each exterior doorway of the structure and contained eight probes as shown in Fig.~\ref{fig:BDP_arrays}. The fourth bi-directional probe and solid thermocouple array, also presented in Fig.~\ref{fig:BDP_arrays}, was located at the opening of the roof vent 0.32~m (12.75 in) above the room ceiling and contained three probes centered between the east and west sides of the vent. The position of each probe and thermocouple pair relative to the south wall of the vent are listed in Table~\ref{table:east_channel_list}. The total heat flux gauge/radiometer pairs were set to be aimed to view the ceiling. Gas concentrations were pulled through 9.5~mm (0.375~in) diameter stainless steel tubing. The height of each individual sensor in the sensor arrays are listed in the channel list found in Table~\ref{table:east_channel_list}.   

\begin{figure}[!ht]
	\includegraphics[width=0.45\columnwidth]{../Figures/Pictures/doorway_BDPs}
	\\~\\
	\includegraphics[width=0.65\columnwidth]{../Figures/Pictures/roof_vent_BDPs.png}
	\caption[Bi-directional probe plus solid thermocouple arrays in East Structure.]{Bi-directional probe plus solid thermocouple array in exterior doorway (top) and in roof vent (bottom) of the East Structure.}
	\label{fig:BDP_arrays}
\end{figure}

\clearpage
% MEASUREMENT TYPE FOR BDPs = 'Velocity' OR 'Pressure'?
\begin{longtable}[c]{c|lll}
\caption{East Structure Channel List\label{table:east_channel_list}} \\
\toprule
\begin{tabular}{c} \textbf{Device} \\ \textbf{Location} \end{tabular} & 
\begin{tabular}{c} \textbf{Channel} \\ \textbf{Name} \end{tabular}  & 
\textbf{Channel Location} & 
\textbf{Measurement Type} \\
\midrule
\endhead
\multirow{13}{*}{\large\textbf{A1}} 
 & TC\_A1\_1  & 0.03~m below ceiling & Temperature \\
 & TC\_A1\_2  & 0.30~m below ceiling & Temperature \\
 & TC\_A1\_3  & 0.61~m below ceiling & Temperature \\
 & TC\_A1\_4  & 0.91~m below ceiling & Temperature \\
 & TC\_A1\_5  & 1.22~m below ceiling & Temperature \\
 & TC\_A1\_6  & 1.52~m below ceiling & Temperature \\
 & TC\_A1\_7  & 1.83~m below ceiling & Temperature \\
 & TC\_A1\_8  & 2.13~m below ceiling & Temperature \\
\cline{2-4}
 & HF\_A1	  & 0.15~m above floor   & Total heat flux \\
 & RAD\_A1    & 0.15~m above floor   & Radiative heat flux \\
\cline{2-4}
 & CO\_A      & 1.22~m above floor   & CO concentration \\
 & CO2\_A     & 1.22~m above floor   & CO$_2$ concentration \\
 & O2\_A      & 1.22~m above floor   & O$_2$ concentration \\
\midrule
\multirow{10}{*}{\large{\textbf{A2}}}
 & TC\_A2\_1  & 0.03~m below ceiling & Temperature \\
 & TC\_A2\_2  & 0.30~m below ceiling & Temperature \\
 & TC\_A2\_3  & 0.61~m below ceiling & Temperature \\
 & TC\_A2\_4  & 0.91~m below ceiling & Temperature \\
 & TC\_A2\_5  & 1.22~m below ceiling & Temperature \\
 & TC\_A2\_6  & 1.52~m below ceiling & Temperature \\
 & TC\_A2\_7  & 1.83~m below ceiling & Temperature \\
 & TC\_A2\_8  & 2.13~m below ceiling & Temperature \\
\cline{2-4}
 & HF\_A2	  & 0.15~m above floor   & Total heat flux \\
 & RAD\_A2    & 0.15~m above floor   & Radiative heat flux \\
\midrule
 \multirow{10}{*}{\large{\textbf{A3}}}
 & TC\_A3\_1  & 0.03~m below ceiling & Temperature \\
 & TC\_A3\_2  & 0.30~m below ceiling & Temperature \\
 & TC\_A3\_3  & 0.61~m below ceiling & Temperature \\
 & TC\_A3\_4  & 0.91~m below ceiling & Temperature \\
 & TC\_A3\_5  & 1.22~m below ceiling & Temperature \\
 & TC\_A3\_6  & 1.52~m below ceiling & Temperature \\
 & TC\_A3\_7  & 1.83~m below ceiling & Temperature \\
 & TC\_A3\_8  & 2.13~m below ceiling & Temperature \\
\cline{2-4}
 & HF\_A3	  & 0.15~m above floor   & Total heat flux \\
 & RAD\_A3    & 0.15~m above floor   & Radiative heat flux \\
% \midrule
\bottomrule
\newpage
\multirow{13}{*}{\large\textbf{A4}} 
 & TC\_A4\_1  & 0.03~m below ceiling & Temperature \\
 & TC\_A4\_2  & 0.30~m below ceiling & Temperature \\
 & TC\_A4\_3  & 0.61~m below ceiling & Temperature \\
 & TC\_A4\_4  & 0.91~m below ceiling & Temperature \\
 & TC\_A4\_5  & 1.22~m below ceiling & Temperature \\
 & TC\_A4\_6  & 1.52~m below ceiling & Temperature \\
 & TC\_A4\_7  & 1.83~m below ceiling & Temperature \\
 & TC\_A4\_8  & 2.13~m below ceiling & Temperature \\
\cline{2-4}
 & HF\_A4	  & 0.15~m above floor   & Total heat flux \\
 & RAD\_A4    & 0.15~m above floor   & Radiative heat flux \\
\cline{2-4}
 & CO\_B      & 1.22~m above floor   & CO concentration \\
 & CO2\_B     & 1.22~m above floor   & CO$_2$ concentration \\
 & O2\_B      & 1.22~m above floor   & O$_2$ concentration \\
\midrule
\multirow{10}{*}{\large{\textbf{A5}}}
 & TC\_A5\_1  & 0.03~m below ceiling & Temperature \\
 & TC\_A5\_2  & 0.30~m below ceiling & Temperature \\
 & TC\_A5\_3  & 0.61~m below ceiling & Temperature \\
 & TC\_A5\_4  & 0.91~m below ceiling & Temperature \\
 & TC\_A5\_5  & 1.22~m below ceiling & Temperature \\
 & TC\_A5\_6  & 1.52~m below ceiling & Temperature \\
 & TC\_A5\_7  & 1.83~m below ceiling & Temperature \\
 & TC\_A5\_8  & 2.13~m below ceiling & Temperature \\
\cline{2-4}
 & HF\_A5	  & 0.15~m above floor   & Total heat flux \\
 & RAD\_A5    & 0.15~m above floor   & Radiative heat flux \\
\midrule
\multirow{16}{*}{\large{\textbf{A7}}}
 & TC\_A7\_1  & 0.03~m below soffit  & Temperature \\
 & TC\_A7\_2  & 0.30~m below soffit  & Temperature \\
 & TC\_A7\_3  & 0.61~m below soffit  & Temperature \\
 & TC\_A7\_4  & 0.91~m below soffit  & Temperature \\
 & TC\_A7\_5  & 1.22~m below soffit  & Temperature \\
 & TC\_A7\_6  & 1.52~m below soffit  & Temperature \\
 & TC\_A7\_7  & 1.83~m below soffit  & Temperature \\
 & TC\_A7\_8  & 2.13~m below soffit  & Temperature \\
\cline{2-4}
 & BDP\_A7\_1 & 0.03~m below soffit  & Velocity \\
 & BDP\_A7\_2 & 0.30~m below soffit  & Velocity \\
 & BDP\_A7\_3 & 0.61~m below soffit  & Velocity \\
 & BDP\_A7\_4 & 0.91~m below soffit  & Velocity \\
 & BDP\_A7\_5 & 1.22~m below soffit  & Velocity \\
 & BDP\_A7\_6 & 1.52~m below soffit  & Velocity \\
 & BDP\_A7\_7 & 1.83~m below soffit  & Velocity \\
 & BDP\_A7\_8 & 2.13~m below soffit  & Velocity \\
\bottomrule
\newpage
\multirow{16}{*}{\large{\textbf{A8}}}
 & TC\_A8\_1  & 0.03~m below soffit  & Temperature \\
 & TC\_A8\_2  & 0.30~m below soffit  & Temperature \\
 & TC\_A8\_3  & 0.61~m below soffit  & Temperature \\
 & TC\_A8\_4  & 0.91~m below soffit  & Temperature \\
 & TC\_A8\_5  & 1.22~m below soffit  & Temperature \\
 & TC\_A8\_6  & 1.52~m below soffit  & Temperature \\
 & TC\_A8\_7  & 1.83~m below soffit  & Temperature \\
 & TC\_A8\_8  & 2.13~m below soffit  & Temperature \\
\cline{2-4}
 & BDP\_A8\_1 & 0.03~m below soffit  & Velocity \\
 & BDP\_A8\_2 & 0.30~m below soffit  & Velocity \\
 & BDP\_A8\_3 & 0.61~m below soffit  & Velocity \\
 & BDP\_A8\_4 & 0.91~m below soffit  & Velocity \\
 & BDP\_A8\_5 & 1.22~m below soffit  & Velocity \\
 & BDP\_A8\_6 & 1.52~m below soffit  & Velocity \\
 & BDP\_A8\_7 & 1.83~m below soffit  & Velocity \\
 & BDP\_A8\_8 & 2.13~m below soffit  & Velocity \\
\midrule
\multirow{16}{*}{\large{\textbf{A9}}}
 & TC\_A9\_1  & 0.03~m below soffit  & Temperature \\
 & TC\_A9\_2  & 0.30~m below soffit  & Temperature \\
 & TC\_A9\_3  & 0.61~m below soffit  & Temperature \\
 & TC\_A9\_4  & 0.91~m below soffit  & Temperature \\
 & TC\_A9\_5  & 1.22~m below soffit  & Temperature \\
 & TC\_A9\_6  & 1.52~m below soffit  & Temperature \\
 & TC\_A9\_7  & 1.83~m below soffit  & Temperature \\
 & TC\_A9\_8  & 2.13~m below soffit  & Temperature \\
\cline{2-4}
 & BDP\_A9\_1 & 0.03~m below soffit  & Velocity \\
 & BDP\_A9\_2 & 0.30~m below soffit  & Velocity \\
 & BDP\_A9\_3 & 0.61~m below soffit  & Velocity \\
 & BDP\_A9\_4 & 0.91~m below soffit  & Velocity \\
 & BDP\_A9\_5 & 1.22~m below soffit  & Velocity \\
 & BDP\_A9\_6 & 1.52~m below soffit  & Velocity \\
 & BDP\_A9\_7 & 1.83~m below soffit  & Velocity \\
 & BDP\_A9\_8 & 2.13~m below soffit  & Velocity \\
\midrule
\multirow{6}{*}{\large{\textbf{A10}}}
 & TC\_A10\_1 & 0.91~m from S side of vent & Temperature \\
 & TC\_A10\_2 & 0.61~m from S side of vent & Temperature \\
 & TC\_A10\_3 & 0.30~m from S side of vent & Temperature \\
\cline{2-4}
 & BDP\_A10\_1 & 0.91~m from S side of vent & Velocity \\
 & BDP\_A10\_2 & 0.61~m from S side of vent & Velocity \\
 & BDP\_A10\_3 & 0.30~m from S side of vent & Velocity \\
\bottomrule
\end{longtable}
\clearpage

\subsection{West Structure}
The first floor of the West Structure was instrumented with three bare-bead thermocouple arrays, two bi-directional probe plus solid thermocouple arrays, and one gas sample inlet pipe. The second floor was equipped with three bare-bead thermocouple arrays, four bi-directional probe plus solid thermocouple arrays, two total heat flux sensor pairs (one facing horizontal and one facing vertical), and one gas sample inlet pipe. The location of the instrumentation in the West Structure is shown in Fig.~\ref{fig:west_instrumentation}.

\begin{figure}[!ht]
	\includegraphics[width=\columnwidth]{../Figures/Floor_Plans/West_Structure_2nd_Floor_Dimensioned_Instrumentation}
	\\~\\
	\includegraphics[width=\columnwidth]{../Figures/Floor_Plans/West_Structure_1st_Floor_Dimensioned_Instrumentation}
	\caption[Location of instrumentation in the West Structure.]{Location of instrumentation in the second floor (top) and first floor (bottom) of the West Structure.}
	\label{fig:west_instrumentation}
\end{figure}

Every thermocouple array contained eight bare-bead thermocouples, each bi-directional probe and solid thermocouple array contained 8 probes, and gas concentrations were pulled through 9.5~mm (0.375~in) diameter stainless steel tubing located 1.2~m (4~ft) above the floor.
\clearpage

\begin{longtable}[c]{c|lll}
\caption{West Structure Channel List\label{table:west_channel_list}} \\
\toprule
\begin{tabular}{c} \textbf{Device} \\ \textbf{Location} \end{tabular} & 
\begin{tabular}{c} \textbf{Channel} \\ \textbf{Name} \end{tabular}  & 
\textbf{Channel Location} & 
\textbf{Measurement Type} \\
\midrule
\endhead
\multirow{11}{*}{\large\textbf{A1}} 
 & TC\_A1\_1  & 0.03~m below ceiling & Temperature \\
 & TC\_A1\_2  & 0.30~m below ceiling & Temperature \\
 & TC\_A1\_3  & 0.61~m below ceiling & Temperature \\
 & TC\_A1\_4  & 0.91~m below ceiling & Temperature \\
 & TC\_A1\_5  & 1.22~m below ceiling & Temperature \\
 & TC\_A1\_6  & 1.52~m below ceiling & Temperature \\
 & TC\_A1\_7  & 1.83~m below ceiling & Temperature \\
 & TC\_A1\_8  & 2.13~m below ceiling & Temperature \\
\cline{2-4}
 & CO\_A      & 1.22~m above floor   & CO concentration \\
 & CO2\_A     & 1.22~m above floor   & CO$_2$ concentration \\
 & O2\_A      & 1.22~m above floor   & O$_2$ concentration \\
\midrule
\multirow{10}{*}{\large{\textbf{A2}}}
 & TC\_A2\_1  & 0.03~m below ceiling & Temperature \\
 & TC\_A2\_2  & 0.30~m below ceiling & Temperature \\
 & TC\_A2\_3  & 0.61~m below ceiling & Temperature \\
 & TC\_A2\_4  & 0.91~m below ceiling & Temperature \\
 & TC\_A2\_5  & 1.22~m below ceiling & Temperature \\
 & TC\_A2\_6  & 1.52~m below ceiling & Temperature \\
 & TC\_A2\_7  & 1.83~m below ceiling & Temperature \\
 & TC\_A2\_8  & 2.13~m below ceiling & Temperature \\
 \midrule
 \multirow{10}{*}{\large{\textbf{A3}}}
 & TC\_A3\_1  & 0.03~m below ceiling & Temperature \\
 & TC\_A3\_2  & 0.30~m below ceiling & Temperature \\
 & TC\_A3\_3  & 0.61~m below ceiling & Temperature \\
 & TC\_A3\_4  & 0.91~m below ceiling & Temperature \\
 & TC\_A3\_5  & 1.22~m below ceiling & Temperature \\
 & TC\_A3\_6  & 1.52~m below ceiling & Temperature \\
 & TC\_A3\_7  & 1.83~m below ceiling & Temperature \\
 & TC\_A3\_8  & 2.13~m below ceiling & Temperature \\
\bottomrule
\newpage
\multirow{16}{*}{\large{\textbf{A4}}}
 & TC\_A4\_1  & 0.03~m below soffit  & Temperature \\
 & TC\_A4\_2  & 0.30~m below soffit  & Temperature \\
 & TC\_A4\_3  & 0.61~m below soffit  & Temperature \\
 & TC\_A4\_4  & 0.91~m below soffit  & Temperature \\
 & TC\_A4\_5  & 1.22~m below soffit  & Temperature \\
 & TC\_A4\_6  & 1.52~m below soffit  & Temperature \\
 & TC\_A4\_7  & 1.83~m below soffit  & Temperature \\
 & TC\_A4\_8  & 2.13~m below soffit  & Temperature \\
\cline{2-4}
 & BDP\_A4\_1 & 0.03~m below soffit  & Velocity \\
 & BDP\_A4\_2 & 0.30~m below soffit  & Velocity \\
 & BDP\_A4\_3 & 0.61~m below soffit  & Velocity \\
 & BDP\_A4\_4 & 0.91~m below soffit  & Velocity \\
 & BDP\_A4\_5 & 1.22~m below soffit  & Velocity \\
 & BDP\_A4\_6 & 1.52~m below soffit  & Velocity \\
 & BDP\_A4\_7 & 1.83~m below soffit  & Velocity \\
 & BDP\_A4\_8 & 2.13~m below soffit  & Velocity \\
\midrule
\multirow{16}{*}{\large{\textbf{A5}}}
 & TC\_A5\_1  & 0.03~m below soffit  & Temperature \\
 & TC\_A5\_2  & 0.30~m below soffit  & Temperature \\
 & TC\_A5\_3  & 0.61~m below soffit  & Temperature \\
 & TC\_A5\_4  & 0.91~m below soffit  & Temperature \\
 & TC\_A5\_5  & 1.22~m below soffit  & Temperature \\
 & TC\_A5\_6  & 1.52~m below soffit  & Temperature \\
 & TC\_A5\_7  & 1.83~m below soffit  & Temperature \\
 & TC\_A5\_8  & 2.13~m below soffit  & Temperature \\
\cline{2-4}
 & BDP\_A5\_1 & 0.03~m below soffit  & Velocity \\
 & BDP\_A5\_2 & 0.30~m below soffit  & Velocity \\
 & BDP\_A5\_3 & 0.61~m below soffit  & Velocity \\
 & BDP\_A5\_4 & 0.91~m below soffit  & Velocity \\
 & BDP\_A5\_5 & 1.22~m below soffit  & Velocity \\
 & BDP\_A5\_6 & 1.52~m below soffit  & Velocity \\
 & BDP\_A5\_7 & 1.83~m below soffit  & Velocity \\
 & BDP\_A5\_8 & 2.13~m below soffit  & Velocity \\
\bottomrule
\newpage
\multirow{16}{*}{\large{\textbf{A6}}}
 & TC\_A6\_1  & 0.03~m below soffit  & Temperature \\
 & TC\_A6\_2  & 0.30~m below soffit  & Temperature \\
 & TC\_A6\_3  & 0.61~m below soffit  & Temperature \\
 & TC\_A6\_4  & 0.91~m below soffit  & Temperature \\
 & TC\_A6\_5  & 1.22~m below soffit  & Temperature \\
 & TC\_A6\_6  & 1.52~m below soffit  & Temperature \\
 & TC\_A6\_7  & 1.83~m below soffit  & Temperature \\
 & TC\_A6\_8  & 2.13~m below soffit  & Temperature \\
\cline{2-4}
 & BDP\_A6\_1 & 0.03~m below soffit  & Velocity \\
 & BDP\_A6\_2 & 0.30~m below soffit  & Velocity \\
 & BDP\_A6\_3 & 0.61~m below soffit  & Velocity \\
 & BDP\_A6\_4 & 0.91~m below soffit  & Velocity \\
 & BDP\_A6\_5 & 1.22~m below soffit  & Velocity \\
 & BDP\_A6\_6 & 1.52~m below soffit  & Velocity \\
 & BDP\_A6\_7 & 1.83~m below soffit  & Velocity \\
 & BDP\_A6\_8 & 2.13~m below soffit  & Velocity \\
\midrule
\multirow{8}{*}{\large{\textbf{A7}}}
 & TC\_A7\_1  & 0.03~m below ceiling & Temperature \\
 & TC\_A7\_2  & 0.30~m below ceiling & Temperature \\
 & TC\_A7\_3  & 0.61~m below ceiling & Temperature \\
 & TC\_A7\_4  & 0.91~m below ceiling & Temperature \\
 & TC\_A7\_5  & 1.22~m below ceiling & Temperature \\
 & TC\_A7\_6  & 1.52~m below ceiling & Temperature \\
 & TC\_A7\_7  & 1.83~m below ceiling & Temperature \\
 & TC\_A7\_8  & 2.13~m below ceiling & Temperature \\
\midrule
\multirow{8}{*}{\large{\textbf{A8}}}
 & TC\_A8\_1  & 0.03~m below ceiling & Temperature \\
 & TC\_A8\_2  & 0.30~m below ceiling & Temperature \\
 & TC\_A8\_3  & 0.61~m below ceiling & Temperature \\
 & TC\_A8\_4  & 0.91~m below ceiling & Temperature \\
 & TC\_A8\_5  & 1.22~m below ceiling & Temperature \\
 & TC\_A8\_6  & 1.52~m below ceiling & Temperature \\
 & TC\_A8\_7  & 1.83~m below ceiling & Temperature \\
 & TC\_A8\_8  & 2.13~m below ceiling & Temperature \\
\bottomrule
\newpage
\multirow{8}{*}{\large{\textbf{A9}}}
 & TC\_A9\_1  & 0.03~m below ceiling & Temperature \\
 & TC\_A9\_2  & 0.30~m below ceiling & Temperature \\
 & TC\_A9\_3  & 0.61~m below ceiling & Temperature \\
 & TC\_A9\_4  & 0.91~m below ceiling & Temperature \\
 & TC\_A9\_5  & 1.22~m below ceiling & Temperature \\
 & TC\_A9\_6  & 1.52~m below ceiling & Temperature \\
 & TC\_A9\_7  & 1.83~m below ceiling & Temperature \\
 & TC\_A9\_8  & 2.13~m below ceiling & Temperature \\
\midrule
\multirow{19}{*}{\large\textbf{A10}} 
 & TC\_A10\_1  & 0.03~m below soffit & Temperature \\
 & TC\_A10\_2  & 0.30~m below soffit & Temperature \\
 & TC\_A10\_3  & 0.61~m below soffit & Temperature \\
 & TC\_A10\_4  & 0.91~m below soffit & Temperature \\
 & TC\_A10\_5  & 1.22~m below soffit & Temperature \\
 & TC\_A10\_6  & 1.52~m below soffit & Temperature \\
 & TC\_A10\_7  & 1.83~m below soffit & Temperature \\
 & TC\_A10\_8  & 2.13~m below soffit & Temperature \\
\cline{2-4}
 & BDP\_A10\_1 & 0.03~m below soffit  & Velocity \\
 & BDP\_A10\_2 & 0.30~m below soffit  & Velocity \\
 & BDP\_A10\_3 & 0.61~m below soffit  & Velocity \\
 & BDP\_A10\_4 & 0.91~m below soffit  & Velocity \\
 & BDP\_A10\_5 & 1.22~m below soffit  & Velocity \\
 & BDP\_A10\_6 & 1.52~m below soffit  & Velocity \\
 & BDP\_A10\_7 & 1.83~m below soffit  & Velocity \\
 & BDP\_A10\_8 & 2.13~m below soffit  & Velocity \\
\cline{2-4}
 & CO\_B      & 1.22~m above floor   & CO concentration \\
 & CO2\_B     & 1.22~m above floor   & CO$_2$ concentration \\
 & O2\_B      & 1.22~m above floor   & O$_2$ concentration \\
\bottomrule
\newpage
\multirow{16}{*}{\large{\textbf{A11}}}
 & TC\_A11\_1  & 0.03~m below soffit  & Temperature \\
 & TC\_A11\_2  & 0.30~m below soffit  & Temperature \\
 & TC\_A11\_3  & 0.61~m below soffit  & Temperature \\
 & TC\_A11\_4  & 0.91~m below soffit  & Temperature \\
 & TC\_A11\_5  & 1.22~m below soffit  & Temperature \\
 & TC\_A11\_6  & 1.52~m below soffit  & Temperature \\
 & TC\_A11\_7  & 1.83~m below soffit  & Temperature \\
 & TC\_A11\_8  & 2.13~m below soffit  & Temperature \\
\cline{2-4}
 & BDP\_A11\_1 & 0.03~m below soffit  & Velocity \\
 & BDP\_A11\_2 & 0.30~m below soffit  & Velocity \\
 & BDP\_A11\_3 & 0.61~m below soffit  & Velocity \\
 & BDP\_A11\_4 & 0.91~m below soffit  & Velocity \\
 & BDP\_A11\_5 & 1.22~m below soffit  & Velocity \\
 & BDP\_A11\_6 & 1.52~m below soffit  & Velocity \\
 & BDP\_A11\_7 & 1.83~m below soffit  & Velocity \\
 & BDP\_A11\_8 & 2.13~m below soffit  & Velocity \\
\midrule
\multirow{16}{*}{\large{\textbf{A13}}}
 & TC\_A13\_1  & 0.03~m below soffit  & Temperature \\
 & TC\_A13\_2  & 0.30~m below soffit  & Temperature \\
 & TC\_A13\_3  & 0.61~m below soffit  & Temperature \\
 & TC\_A13\_4  & 0.91~m below soffit  & Temperature \\
 & TC\_A13\_5  & 1.22~m below soffit  & Temperature \\
 & TC\_A13\_6  & 1.52~m below soffit  & Temperature \\
 & TC\_A13\_7  & 1.83~m below soffit  & Temperature \\
 & TC\_A13\_8  & 2.13~m below soffit  & Temperature \\
\cline{2-4}
 & BDP\_A13\_1 & 0.03~m below soffit  & Velocity \\
 & BDP\_A13\_2 & 0.30~m below soffit  & Velocity \\
 & BDP\_A13\_3 & 0.61~m below soffit  & Velocity \\
 & BDP\_A13\_4 & 0.91~m below soffit  & Velocity \\
 & BDP\_A13\_5 & 1.22~m below soffit  & Velocity \\
 & BDP\_A13\_6 & 1.52~m below soffit  & Velocity \\
 & BDP\_A13\_7 & 1.83~m below soffit  & Velocity \\
 & BDP\_A13\_8 & 2.13~m below soffit  & Velocity \\
\bottomrule
\newpage
\multirow{16}{*}{\large{\textbf{A14}}}
 & TC\_A14\_1  & 0.03~m below soffit  & Temperature \\
 & TC\_A14\_2  & 0.30~m below soffit  & Temperature \\
 & TC\_A14\_3  & 0.61~m below soffit  & Temperature \\
 & TC\_A14\_4  & 0.91~m below soffit  & Temperature \\
 & TC\_A14\_5  & 1.22~m below soffit  & Temperature \\
 & TC\_A14\_6  & 1.52~m below soffit  & Temperature \\
 & TC\_A14\_7  & 1.83~m below soffit  & Temperature \\
 & TC\_A14\_8  & 2.13~m below soffit  & Temperature \\
\cline{2-4}
 & BDP\_A14\_1 & 0.03~m below soffit  & Velocity \\
 & BDP\_A14\_2 & 0.30~m below soffit  & Velocity \\
 & BDP\_A14\_3 & 0.61~m below soffit  & Velocity \\
 & BDP\_A14\_4 & 0.91~m below soffit  & Velocity \\
 & BDP\_A14\_5 & 1.22~m below soffit  & Velocity \\
 & BDP\_A14\_6 & 1.52~m below soffit  & Velocity \\
 & BDP\_A14\_7 & 1.83~m below soffit  & Velocity \\
 & BDP\_A14\_8 & 2.13~m below soffit  & Velocity \\
\midrule
\multirow{3}{*}{\large{\textbf{A16}}}
 & HF\_2\_H	  & \begin{tabular}{l} 1~m above floor, \\ facing doorway (horizontal) \end{tabular} & Total heat flux \\
 & HF\_2\_V   & \begin{tabular}{l} 1~m above floor, \\ facing ceiling (vertical) \end{tabular} 	   & Total heat flux \\
\midrule
\multirow{3}{*}{\large{\textbf{A17}}}
 & HF\_1\_H	  & \begin{tabular}{l} 1~m above floor, \\ facing doorway (horizontal) \end{tabular} & Total heat flux \\
 & HF\_1\_V	  & \begin{tabular}{l} 1~m above floor, \\ facing ceiling (vertical) \end{tabular} 	   & Total heat flux \\
\bottomrule
\end{longtable}
\clearpage
% FIX 'CHANNEL LOCATION' CENTERING FOR A16/17 

\subsection{Measurement Uncertainty}
\label{sec:Uncertainty}
There are different components of uncertainty in the reported length, mass, temperature, heat flux, gas concentration, differential pressure, and gas velocity measurements. Uncertainties are grouped into two categories according to the method used to estimate them. Type A uncertainties are those which are evaluated by statistical methods, and Type B are those which are evaluated by other means~\cite{Taylor&Kuyatt:1994}. Type B analysis of systematic uncertainties involves estimating the upper (+a) and lower (-a) limits for the quantity in question such that the probability that the value would be in the interval ($\pm$a) is essentially 100\%. After estimating uncertainties by either Type A or B analysis, the uncertainties are combined in quadrature to yield the combined standard uncertainty. Then the combined standard uncertainty is multiplied by a coverage factor of two, which results in the expanded uncertainty with a 95\% confidence interval (2$\sigma$). For some of these components, such as the zero and calibration elements, uncertainties are derived from referenced instrument specifications. For other components, referenced research results and past experience with the instruments provided input in the uncertainty determination.

\subsubsection{Compartment Dimensions}
Each length measurement was taken carefully. Length measurements such as the room dimensions and instrumentation array locations were made with a hand held laser measurement device, which has an accuracy of $\pm$6.0~mm (0.25~in) over a range of 0.61~m (2.0~ft) to 15.3~m (50.0~ft)~\cite{StanleyTools}. However, conditions affecting the measurement, such as levelness of the device, yields an estimated uncertainty of $\pm$0.5\% for measurements in the 2.0~m (6.6~ft) to 10.0~m (32.8~ft) range. Steel measuring tapes with a resolution of $\pm$0.5~mm (0.02~in) were used to locate individual sensors within a measurement array and to measure and position the furniture. The steel measuring tapes were manufactured in compliance with NIST Manual 44, which specifies a tolerance of $\pm$1.6~mm (0.06~in) for 9.1~m (30~ft) tapes and $\pm$6.4~mm (0.25~in) for 30.5~m (100~ft) tapes~\cite{Butcher:2012}. Some issues, such as ``soft'' edges on the upholstered furniture, result in an estimated total expanded uncertainty of $\pm$1.0\%.

\subsubsection{Thermocouples}
The standard uncertainty in the temperature of the thermocouple wire itself is $\pm$2.2$^{\circ}$C at $277^{\circ}$C and increases to $\pm$9.5$^{\circ}$C at $871^{\circ}$C as determined by the wire manufacturer~\cite{Omega:2004}. The variation of the temperature in the environment surrounding the thermocouple is known to be much greater than that of the wire uncertainty~\cite{Blevins:1999,Pitts:2003}. Small diameter thermocouples were used to limit the impact of radiative heating and cooling. The estimated total expanded uncertainty for temperature in these experiments is $\pm$15\%.

\subsubsection{Heat Flux Gauges}
Total heat flux measurements were made with water-cooled Schimidt-Bolter gauges. The manufacturer reports a $\pm$3\% calibration expanded uncertainty for these devices~\cite{Medtherm:2003}. Results from an international study on total heat flux gauge calibration and response demonstrated that the uncertainty of a Schmidt-Boelter gauge is typically $\pm$8\%~\cite{Pitts:2006}.

\subsubsection{Gas Sampling}
The gas measurement instruments and sampling system used in this series of experiments have demonstrated an expanded (k = 2) relative uncertainty of $\pm$1\% when compared with span gas volume fractions~\cite{Bundy:2007}. Given the non-uniformities and movement of the fire gas environment and the limited set of sampling points in these experiments, an estimated uncertainty of $\pm$12\% is associated with gas concentration measurements~\cite{Lock:1}.

\subsubsection{Pressure Transducers}
Differential pressure reading uncertainty components were derived from pressure transducer instrument specifications and previous experience with pressure transducers. The transducers were factory calibrated and the zero and span of each was checked in the laboratory prior to the experiments yielding an accuracy of $\pm$1\%~\cite{Setra:2002}. The total expanded uncertainty was estimated at 10\%.

\subsubsection{Bi-Directional Probes}
Bi-directional probes and single thermocouples were used to measure the velocity. The bi-directional probes used similar pressure transducers as those used for the differential pressure measurements discussed above. A single bare-bead Chromel-Alumel (type K) thermocouple with a 1.0~mm (0.04~in) nominal diameter was co-located with each probe. The thermocouple wire was protected with a 3.2~mm (0.125~in) diameter inconel sheath. A gas velocity measurement study examining the doorway flow of pre-flashover compartment fires yielded expanded uncertainty measurements ranging from $\pm$0.14 to $\pm$0.22 for bi-directional probes similar to the ones described here~\cite{Bryant:FSJ2009}. The total expanded uncertainty for gas velocity in these experiments was estimated to be $\pm$18\%.

\clearpage

% ==========================
% = EXPERIMENTAL PROCEDURE =
% ==========================
\chapter{Experimental Procedure}
\label{chap:Experimental_Procedure}
A similar procedure was followed for all experiments described in this report. First, three propane burners were ignited in sequential order. Next, different doors and vents in the structure were opened and closed to change the ventilation pattern within the structure. Finally, the burners were turned off in sequential order and the fire was extinguished.

\section{East Structure Tests}
\label{sec:east_procedure}
Four different tests---Test 2, Test 3, Test 4, and Test 5---were conducted in the East Structure. Each series was composed of three experiments that used an identical procedure to change the ventilation in the structure. The experiments conducted during Series 4 followed a procedure in which each exterior door was opened to change the ventilation pattern. During the experiments in Series 5, each double door was opened and the roof vent was opened and closed. Both the roof vent and west double door were opened during each experiment in Series 6. Figs.~\ref{fig:east_test_4}--\ref{fig:east_test_6} include a schematic and additional details about the procedures used for the experiments during Series 4--6.

% \begin{table}[!ht]
% \caption{Event times for Tests 2-4}
% \begin{tabular}{lccc}
% \toprule
%  & \multicolumn{3}{c}{\textbf{\underline{Event Time (mm:ss)}}} \\
% \textbf{Event} 				& \textbf{Test 2} & \textbf{Test 3} & \textbf{Test 4} \\
% \midrule
% Corner burner on 			& 	21:00		  &		4:00		&		4:00	  \\
% Middle burner on 			&   24:00		  &		7:00		&		7:00	  \\
% Center burner on 			&   27:00		  &	   10:00		&	   10:00	  \\
% West double door opened 	&   28:00		  &    11:00		&	   11:00	  \\
% East double door opened 	&   30:00		  &    13:00		&	   13:00	  \\
% South exterior door opened 	&   31:00		  &    14:00		&	   14:00	  \\
% Center burner off 			&   33:00		  &    17:00		&	   17:00	  \\
% Middle burner off 			&   35:00		  &    19:00		&	   19:00	  \\
% Corner burner off 			&   37:00		  &    21:00		&	   21:00	  \\
% \bottomrule
% \end{tabular}
% \label{table:Tests_2-4_times}
% \end{table}
% RAN FAN AFTER BURNER WAS OFF, BUT DIDN'T INCLUDE BECAUSE BURNERS WERE OFF

\begin{figure}[!ht]
\begin{minipage}[b]{0.8\columnwidth}
	\begin{flushleft}
	\small
	\begin{tabular}{lccc}
\multicolumn{4}{c}{\normalsize Event Times (mm:ss) for Test Data File} \\
\toprule
\multicolumn{1}{c}{\textbf{Event}} & \textbf{Test 2} & \textbf{Test 3} & \textbf{Test 4} \\
\midrule
(1) Corner burner on 			& 	21:00		  &		4:00		&		4:00	  \\
(2) Middle burner on 			&   24:00		  &		7:00		&		7:00	  \\
(3) Center burner on 			&   27:00		  &	   10:00		&	   10:00	  \\
(4) West double door opened 	&   28:00		  &    11:00		&	   11:00	  \\
(5) East double door opened 	&   30:00		  &    13:00		&	   13:00	  \\
(6) South exterior door opened 	&   31:00		  &    14:00		&	   14:00	  \\
(7) Center burner off 			&   33:00		  &    17:00		&	   17:00	  \\
(8) Middle burner off 			&   35:00		  &    19:00		&	   19:00	  \\
(9) Corner burner off 			&   37:00		  &    21:00		&	   21:00	  \\
\bottomrule
\end{tabular}
	% \begin{tabular}[b]{cc}
 % 	\toprule
 % 	\textbf{Number} & \textbf{Event Description} \\
 % 	\midrule
 % 	1-3  & Burners ignited \\
 % 	4	 & West double door opened \\
 % 	5 	 & East double door opened \\
 % 	6	 & South exterior door opened \\
 % 	7-9  & Burners extinguished \\
	% \bottomrule
	% \end{tabular}
	\end{flushleft}
\end{minipage}
\begin{minipage}[b]{0.9\columnwidth}
	\vspace{15pt}
	\centering
	\includegraphics[width=\columnwidth]{../Figures/Floor_Plans/East_Structure_Test_4}
\end{minipage}
\caption{Test 2-4 layout and sequence of events.}
\label{fig:Tests_2-4_layout}
\end{figure}

\begin{figure}[!ht]
\begin{minipage}[b]{0.8\columnwidth}
	\begin{flushleft}
	\small
\begin{figure}[!ht]
\begin{minipage}[b]{0.8\columnwidth}
	\begin{flushleft}
	\small
	\begin{tabular}{lccc}
\multicolumn{4}{c}{\normalsize Event Times (mm:ss) for Test Data File} \\
\toprule
\multicolumn{1}{c}{\textbf{Event}} & \textbf{Test 2} & \textbf{Test 3} & \textbf{Test 4} \\
\midrule
(1) Corner burner on 			& 	14:00		  &	   34:30		&	   54:30	\\
(2) Middle burner on 			&   14:15		  &	   34:45		&	   54:45	\\
(3) Center burner on 			&   14:30		  &	   35:00		&	   55:00	\\
(4) Roof vent opened 			&   16:38		  &    36:30		&	   56:30	\\
(5) West double door opened 	&	17:03		  &	   38:00 		&	   58:00 	\\
(6) East double door opened 	&   20:03		  &    39:30		&	   59:31	\\
(7) Roof vent closed		 	&   21:30		  &    42:45		&	   61:43	\\
(7) Center burner off 			&   23:00		  &    44:15		&	   63:30	\\
(8) Middle burner off 			&   23:20		  &    44:30		&	   63:45	\\
(9) Corner burner off 			&   23:40		  &    44:45		&	   64:00	\\
(10) Roof vent opened			& 	26:05 		  &	   45:30		&	   65:30	\\
(11) East double door closed	& 	33:17 		  &	   52:38		&	   N/A	\\
(12) West double door closed	& 	33:30 		  &	   52:55		&	   N/A	\\
(13) Roof vent closed		 	&   33:45		  &    53:54		&	   N/A	\\
\bottomrule
\end{tabular}

	% \begin{tabular}[b]{cc}
 % 	\toprule
 % 	\textbf{Number} & \textbf{Event Description} \\
 % 	\midrule
 % 	1-3  & Burners ignited \\
 % 	4	 & Roof vent opened \\
 % 	5 	 & West double door opened \\
 % 	6 	 & East double door opened \\	
 % 	7	 & Roof vent closed \\
 % 	8-10  & Burners extinguished \\
	% \bottomrule
	% \end{tabular}
	\end{flushleft}
\end{minipage}
\begin{minipage}[b]{0.9\columnwidth}
	\vspace{15pt}
	\centering
	\includegraphics[width=\columnwidth]{../Figures/Floor_Plans/East_Structure_Test_5}
\end{minipage}
\caption{Test Series 5 layout and sequence of events.}
\label{fig:east_test_5}
\end{figure}

\begin{figure}[!ht]
\begin{minipage}[b]{0.8\columnwidth}
	\begin{flushleft}
	\small
	\begin{tabular}[b]{cc}
 	\toprule
 	\textbf{Number} & \textbf{Event Description} \\
 	\midrule
 	1-3  & Burners ignited \\
 	4	 & Roof vent opened \\
 	5 	 & West double door opened \\
 	6-8  & Burners extinguished \\
	\bottomrule
	\end{tabular}
	\end{flushleft}
\end{minipage}
\begin{minipage}[b]{0.9\columnwidth}
	\vspace{15pt}
	\centering
	\includegraphics[width=\columnwidth]{../Figures/Floor_Plans/East_Structure_Test_6}
\end{minipage}
\caption{Test Series 6 layout and sequence of events.}
\label{fig:east_test_6}
\end{figure}
\FloatBarrier

\section{West Structure Tests}
\label{sec:west_procedure}
Two different test series---Series 22 and Series 24---were conducted in the West Structure. Each series was composed of two experiments that used an identical procedure to change the ventilation in the structure. Experiments conducted during Series 22 followed a procedure in which each double door on the first and second floors was opened. Experiments from Series 24 followed a sequence of events that involved opening the interior stairwell door, opening the west double door on both the first and second level, and opening the south side exterior door on the second level of the structure. Figs.~\ref{fig:west_test_22}~and~\ref{fig:west_test_24} include a schematic and additional details about the procedures used for the experiments during Series 22 and 24.

\begin{figure}[!ht]
\begin{minipage}[b]{0.8\columnwidth}
	\begin{flushleft}
	\small
	\begin{tabular}[b]{cc}
 	\toprule
 	\textbf{Number} & \textbf{Event Description} \\
 	\midrule
 	1-3  & Burners ignited \\
 	4	 & 2nd floor west double door opened \\
 	5 	 & 1st floor west double door opened \\
 	6	 & 1st floor east double door opened \\
 	7 	 & 2nd floor east double door opened \\
 	8-10  & Burners extinguished \\
	\bottomrule
	\end{tabular}
	\end{flushleft}
\end{minipage}
\begin{minipage}[b]{0.9\columnwidth}
	\vspace{15pt}
	\centering
	\includegraphics[width=\columnwidth]{../Figures/Floor_Plans/West_Structure_2nd_Floor_Test_22}
	\includegraphics[width=\columnwidth]{../Figures/Floor_Plans/West_Structure_1st_Floor_Test_22}
\end{minipage}
\caption{Test Series 22 layout and sequence of events.}
\label{fig:west_test_22}
\end{figure}

\begin{figure}[!ht]
\begin{minipage}[b]{0.8\columnwidth}
	\begin{flushleft}
	\small
	\begin{tabular}[b]{cc}
 	\toprule
 	\textbf{Number} & \textbf{Event Description} \\
 	\midrule
 	1-3  & Burners ignited \\
 	4	 & Interior stairwell door opened \\
 	5 	 & 1st floor west double door opened \\
 	6	 & 2nd floor west double door opened \\
 	7 	 & 2nd floor south exterior door opened \\
 	8-10 & Burners extinguished \\
	\bottomrule
	\end{tabular}
	\end{flushleft}
\end{minipage}
\begin{minipage}[b]{0.9\columnwidth}
	\vspace{15pt}
	\centering
	\includegraphics[width=\columnwidth]{../Figures/Floor_Plans/West_Structure_2nd_Floor_Test_24}
	\includegraphics[width=\columnwidth]{../Figures/Floor_Plans/West_Structure_1st_Floor_Test_24}
\end{minipage}
\caption{Test Series 24 layout and sequence of events.}
\label{fig:west_test_24}
\end{figure}
\FloatBarrier

\clearpage

\chapter{Summary}
\label{chap:Summary}

\bibliography{../../../../../Bibliography/FDS_refs,../../../../../Bibliography/FDS_general}

\appendix

% \begin{landscape}
% \begin{table}[!ht]
% \caption{East Structure experimental event times (mm:ss)}
% \begin{tabular}{lcccccccccccccc}
%  \toprule
% \textbf{Test} &
% \multicolumn{2}{c}{\textbf{\underline{Corner Burner}}} &
% \multicolumn{2}{c}{\textbf{\underline{Middle Burner}}} &
% \multicolumn{2}{c}{\textbf{\underline{Center Burner}}} &
% \multicolumn{2}{c}{\textbf{\underline{W Double Door}}} &
% \multicolumn{2}{c}{\textbf{\underline{E Double Door}}} &
% \multicolumn{2}{c}{\textbf{\underline{Single Door}}} &
% \multicolumn{2}{c}{\textbf{\underline{Roof Vent}}}
% \\
% \textbf{Number} &
% \textbf{On} & \textbf{Off} & \textbf{On} & \textbf{Off} & \textbf{On} & \textbf{Off} &
% \textbf{Close} & \textbf{Open} & \textbf{Close} & \textbf{Open} &
% \textbf{Close} & \textbf{Open} & \textbf{Close} & \textbf{Open}
% \\
% \midrule
% % Test 2
% 2 & 0:20 & 16:20 & 3:20 & 14:20 & 6:20 & 12:20 &
% N/A & 7:20 & N/A & 9:20 & N/A & 10:30 & N/A & N/A \\
% % Test 3
% 3 & 0:20 & 17:20 & 3:20 & 15:20 & 6:20 & 13:20 &
% N/A & 7:20 & N/A & 9:20 & N/A & 10:20 & N/A & N/A \\
% % Test 4
% 4 & 0:20 & 17:20 & 3:20 & 15:20 & 6:20 & 13:20 &
% N/A & 7:20 & N/A & 9:20 & N/A & 10:20 & N/A & N/A
% \\ \multicolumn{15}{c}{ } \\
% % Test 5a
% 5a & 0:15 & 9:55 & 0:30 & 9:35 & 0:45 & 9:15 &
% N/A & 3:18 & N/A & 6:18 & N/A & N/A & 7:45 & 2:53 \\
% % Test 5b
% 5b & 0:15 & 10:30 & 0:30 & 10:15 & 0:45 & 10:00 &
% N/A & 3:45 & N/A & 5:15 & N/A & N/A & 8:30 & 2:15 \\
% % Test 5c
% 5c & 0:15 & 9:45 & 0:30 & 9:30 & 0:45 & 9:15 &
% N/A & 3:45 & N/A & 5:16 & N/A & N/A & 7:28 & 2:15
% \\ \multicolumn{15}{c}{ } \\
% % Test 6a
% 6a & 0:15 & 9:55 & 0:30 & 9:35 & 0:45 & 9:15 &
% N/A & 3:18 & N/A & N/A & N/A & N/A & N/A & 2:53 \\
% % Test 6b
% 6b & 0:15 & 9:55 & 0:30 & 9:35 & 0:45 & 9:15 &
% N/A & 3:18 & N/A & N/A & N/A & N/A & N/A & 2:53 \\
% % Test 6c
% 6c & 0:15 & 9:55 & 0:30 & 9:35 & 0:45 & 9:15 &
% N/A & 3:18 & N/A & N/A & N/A & N/A & N/A & 2:53 \\
% \bottomrule
% \end{tabular}
% \label{table:east_exp_times}
% \end{table}
% \end{landscape}


\end{document}
