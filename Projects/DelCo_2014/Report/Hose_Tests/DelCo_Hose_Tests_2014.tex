\documentclass[12pt,oneside]{book}

%%%%%%%%%%%%%%%%%%%%%%%%%%%%%%%%%%%%%%%%%%%%%%%%%%%%%%%%%%%%%%%%%%%%%%%%%%%%%%%%%%%%%%%%%%%%%%%%%%%
%                                                                                                 %
% The mathematical style of these documents follows                                               %
%                                                                                                 %
% A. Thompson and B.N. Taylor. The NIST Guide for the Use of the International System of Units.   %
%    NIST Special Publication 881, 2008.                                                          %
%                                                                                                 %
% http://www.nist.gov/pml/pubs/sp811/index.cfm                                                    %
%                                                                                                 %
%%%%%%%%%%%%%%%%%%%%%%%%%%%%%%%%%%%%%%%%%%%%%%%%%%%%%%%%%%%%%%%%%%%%%%%%%%%%%%%%%%%%%%%%%%%%%%%%%%%

\input{../../../../Bibliography/commoncommands}

% Rename chapter headings
\renewcommand{\chaptername}{Section}
\renewcommand{\bibname}{References}

% Math shortcuts
\renewcommand{\sb}[1]{_\mathrm{#1}}
\renewcommand{\C}{\mbox{C}}
\renewcommand{\H}{\mbox{H}}
\renewcommand{\O}{\mbox{O}}
\newcommand{\N}{\mbox{N}}

% Center all figures
\makeatletter
\g@addto@macro\@floatboxreset\centering
\makeatother

\begin{document}

\bibliographystyle{unsrt}
\pagestyle{empty}

\begin{minipage}[t][9in][s]{6.25in}

\headerB{
Delco 2014
}

\headerC{
\flushright{
Daniel Madrzykowski \\
Craig G. Weinschenk \\
Kristopher J. Overholt \\
{\em Fire Research Division \\
Engineering Laboratory \\
Gaithersburg, Maryland, USA} \\ }
}

\flushright{\today \\
}

\vfill

\flushright{
\includegraphics[width=2.in]{../../../../Bibliography/nistident_flright_vec} \\[.3in]
}

\titlesigs

\end{minipage}

\newpage

\frontmatter

\pagestyle{plain}
\pagenumbering{roman}

\cleardoublepage
\phantomsection
\addcontentsline{toc}{chapter}{Contents}
\tableofcontents

\cleardoublepage
\phantomsection
\addcontentsline{toc}{chapter}{List of Figures}
\listoffigures

\cleardoublepage
\phantomsection
\addcontentsline{toc}{chapter}{List of Tables}
\listoftables

\chapter{List of Acronyms}

\begin{tabbing}
\hspace{1.5in} \= \\
FDS \> Fire Dynamics Simulator \\
HGL \> Hot Gas Layer \\
HRR \> Heat Release Rate \\
HRRPUA \> Heat Release Rate per Unit Area \\
NIST \> National Institute of Standards and Technology \\
\end{tabbing}

\mainmatter

\chapter{Introduction}
\label{chap:Introduction}

\chapter{Experimental Setup}
\label{chap:Experimental_Setup}

\section{Experimental Structure}
\label{sec:Experimental_Structure}

\subsection{Construction}
\label{sec:Construction}
Two structures, shown in Fig.~\ref{fig:struct_pics}, were used for the experiments. Each structure contained a ground level floor composed of concrete walls. The West Structure contained a second level.

\begin{figure}[!ht]
\includegraphics[width=6in]{../../Figures/east_structure}
\\~\\
\includegraphics[width=6in]{../../Figures/west_structure}
\caption{East (top) and West (bottom) Test Structures}
\label{fig:struct_pics}
\end{figure}

\clearpage

\begin{figure}[!ht]
\includegraphics[width=6in]{../../Drawings/PDFs/Without_Intrumentation/West_Structure_1st_Floor_Metric_Simple}
\\
\includegraphics[width=6in]{../../Drawings/PDFs/Without_Intrumentation/West_Structure_2nd_Floor_Metric_Simple}
\caption{West Structure first floor (top) and second floor (bottom) layouts.}
\label{fig:west_general_plan}
\end{figure}

\clearpage
\section{Fuel Load}

\section{Instrumentation}

\section{Uncertainty}

\section{Experimental Procedure}

\subsection{Tests 18 and 19}
Tests 18 and 19 varied in that the first floor south side door was open for Test 18 (Fig. \ref{fig:test_18_plan}) and closed for Test 19 (Fig. \ref{fig:test_19_plan}).

Tests 18 and 19 followed identical procedures that involved flowing water into the West Structure through the first floor, south side double doors using a 1.75 in. hose with a combination nozzle. The nozzle was used to produce three types of hose streams: a straight stream, a narrow fog stream, and a wide fog stream. Tests 18 and 19 involved three sets of experiments, one for each type of hose stream. For each set of experiments, four different application patterns were tested. First, water was applied while the hose was in a fixed position. Next, the hoseline was moved from side-to-side across the open room, creating a sweeping application pattern. Finally, the hoseline was rotated in both the clockwise and counterclockwise directions, creating clockwise and counterclockwise application patterns. 

To begin each set of experiments, the combination nozzle was adjusted to produce the desired hose stream pattern. Once the nozzle was adjusted, water was applied while the hoseline was in the fixed position. After 60 seconds of water flow, the stairwell door was opened. 60 seconds after the stairwell door was opened, the north side, west double door on the second floor was opened, and water continued to flow for 60 more seconds. Then, the hoseline and two doors were closed, and the procedure was repeated for the sweeping, clockwise, and counterclockwise application patterns.

\begin{figure}[!ht]
\includegraphics[width=6in]{../../Drawings/PDFs/Without_Intrumentation/West_Structure_Hose_Test_18_1st_Floor}
\\
\includegraphics[width=6in]{../../Drawings/PDFs/Without_Intrumentation/West_Structure_Hose_Test_18_2nd_Floor}
\caption{West Structure first floor (top) and second floor (bottom) layouts for Test 18. Both double doors and the south side door were open on the first floor for the duration of the test. On the second floor, the stairwell door and north side, west double door were opened and closed during the experiments, while the north side, east double door and south side door were in the closed position throughout the entire test.}
\label{fig:test_18_plan}
\end{figure}

\clearpage

\begin{figure}[!ht]
\includegraphics[width=6in]{../../Drawings/PDFs/Without_Intrumentation/West_Structure_Hose_Test_19_1st_Floor}
\\
\includegraphics[width=6in]{../../Drawings/PDFs/Without_Intrumentation/West_Structure_Hose_Test_19_2nd_Floor}
\caption{West Structure first floor (top) and second floor (bottom) layouts for Test 19. Both double doors were open and the south side door was closed on the first floor for the duration of the test. On the second floor, the stairwell door and north side, west double door were opened and closed during the experiments, while the north side, east double door and south side door were in the closed position throughout the entire test.}
\label{fig:test_19_plan}
\end{figure}

\clearpage

\chapter{Results}
\label{chap:Results}

\section{Hose Flow Tests}

\subsection{Tests 18 and 19}


\begin{figure}[!ht]
\includegraphics[width=6in]{../../Figures/Hose_Test_Figures/Test_18_West_063014BDP_A10_8_All_Streams}
\caption{Velocity at Ceiling Level of Stairwell Door, Test 18, All Streams}
\label{fig:Test_18_West_063014_SS_BDP_A10}
\end{figure}

\begin{figure}[!ht]
\includegraphics[width=6in]{../../Figures/Hose_Test_Figures/Test_19_West_063014BDP_A10_8_All_Streams}
\caption{Velocity at Ceiling Level of Stairwell Door, Test 19, All Streams}
\label{fig:Test_19_West_063014_SS_BDP_A10}
\end{figure}

\clearpage

\subsubsection{Clockwise vs. Counterclockwise Rotation}

\begin{figure}[!ht]
\includegraphics[width=6in]{../../Figures/Hose_Test_Figures/Test_18_West_063014BDP_A10_8_CW_vs_CCW}
\caption{Velocity at Ceiling Level of Stairwell Door, Test 18, All Streams, CW vs. CCW}
\label{fig:Test_18_West_063014_SS_BDP_A10}
\end{figure}

\begin{figure}[!ht]
\includegraphics[width=6in]{../../Figures/Hose_Test_Figures/Test_19_West_063014BDP_A10_8_CW_vs_CCW}
\caption{Velocity at Ceiling Level of Stairwell Door, Test 19, All Streams, CW vs. CCW}
\label{fig:Test_19_West_063014_SS_BDP_A10}
\end{figure}

\clearpage



\chapter{Conclusions}
\label{chap:Conclusions}

\chapter{Future Work}
\label{chap:Future_Work}

\chapter{Acknowledgments}
\label{chap:Acknowledgments}

\bibliography{../../../Bibliography/FDS_refs,../../../Bibliography/FDS_general}

\appendix

\chapter{Appendix A}

Placeholder


\end{document}
