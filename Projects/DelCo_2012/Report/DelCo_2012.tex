\documentclass[12pt,oneside]{book}

%%%%%%%%%%%%%%%%%%%%%%%%%%%%%%%%%%%%%%%%%%%%%%%%%%%%%%%%%%%%%%%%%%%%%%%%%%%%%%%%%%%%%%%%%%%%%%%%%%%
%                                                                                                 %
% The mathematical style of these documents follows                                               %
%                                                                                                 %
% A. Thompson and B.N. Taylor. The NIST Guide for the Use of the International System of Units.   %
%    NIST Special Publication 881, 2008.                                                          %
%                                                                                                 %
% http://www.nist.gov/pml/pubs/sp811/index.cfm                                                    %
%                                                                                                 %
%%%%%%%%%%%%%%%%%%%%%%%%%%%%%%%%%%%%%%%%%%%%%%%%%%%%%%%%%%%%%%%%%%%%%%%%%%%%%%%%%%%%%%%%%%%%%%%%%%%

\input{../../../Bibliography/commoncommands}

% Rename chapter headings
\renewcommand{\chaptername}{Section}
\renewcommand{\bibname}{References}

% Math shortcuts
\renewcommand{\sb}[1]{_\mathrm{#1}}
\renewcommand{\C}{\mbox{C}}
\renewcommand{\H}{\mbox{H}}
\renewcommand{\O}{\mbox{O}}
\newcommand{\N}{\mbox{N}}

% Center all figures
\makeatletter
\g@addto@macro\@floatboxreset\centering
\makeatother

% Extra packages
\usepackage{xfrac}

\begin{document}
	
\bibliographystyle{unsrt}
\pagestyle{empty}
	
\begin{minipage}[t][9in][s]{6.25in}
	
\begin{flushright}
\fontsize{20}{24}\selectfont
\bf{NIST Technical Note XXXX}
\end{flushright}
		
\headerB{
Delaware County, PA Fire Suppression Tests \\
Spring 2012 \\
}
		
\normalsize
		
\headerC{
{
\flushright{
Daniel Madrzykowski \\
Keith M. Stakes \\
					
\vspace*{2\baselineskip}
				
\begingroup
This publication is available free of charge from:
\hypersetup{urlcolor=black}
\href{http://dx.doi.org/10.6028/NIST.TN.XXXX}{http://dx.doi.org/10.6028/NIST.TN.XXXX}
\endgroup
}
				
\vfill
				
\flushright{
		
\includegraphics[width=2.in]{../../../Bibliography/nistident_flright_vec} \\[.3in]
}
}
}
		
\end{minipage}

\newpage
\hspace{5in}
\newpage
	
\frontmatter
	
\pagenumbering{roman}
	
\begin{minipage}[t][9in][s]{6.25in}
		
\begin{flushright}
\fontsize{20}{24}\selectfont
\bf{NIST Technical Note XXXX}
\end{flushright}
		
\headerB{
Delaware County, PA Fire Suppression Tests \\
Spring 2012 \\
}
		
\headerC{
\flushright{
Daniel Madrzykowski \\
Keith M. Stakes \\
{\em Fire Research Division \\
Engineering Laboratory} \\
				
\vspace*{2\baselineskip}
				
\begingroup
This publication is available free of charge from:
\hypersetup{urlcolor=black}
\href{http://dx.doi.org/10.6028/NIST.TN.XXXX}{http://dx.doi.org/10.6028/NIST.TN.XXXX} \\
\endgroup
				
\vspace*{2\baselineskip}
August 2014}}
		
\vfill
		
\flushright{\includegraphics[width=1in]{../../../Bibliography/doc} }
	
\titlesigs
		
\end{minipage}
	
\newpage
	
\begin{minipage}[t][9in][s]{6.25in}
		
\flushright{Certain commercial entities, equipment, or materials may be identified in this \\
document in order to describe an experimental procedure or concept adequately. \\
Such identification is not intended to imply recommendation or endorsement by the \\
National Institute of Standards and Technology, nor is it intended to imply that the \\
entities, materials, or equipment are necessarily the best available for the purpose. \\
}
		
\vspace{3in}
		
\large
\flushright{\bf National Institute of Standards and Technology Technical Note XXXX \\
Natl.~Inst.~Stand.~Technol.~Tech.~Note~XX, \pageref{LastPage} pages (August 2014) \\
% http://dx.doi.org/10.6028/NIST.TN.XXXX \\
CODEN: NTNOEF }

\vspace{0.2in}

\begingroup
{\bf This publication is available free of charge from:}
\hypersetup{urlcolor=black}
\href{http://dx.doi.org/10.6028/NIST.TN.1838}{\bf http://dx.doi.org/10.6028/NIST.TN.1838} \\
\endgroup		

\vfill
		
\hspace{1in}
		
\end{minipage}
	
\newpage
	
\frontmatter
	
\pagestyle{plain}
\pagenumbering{roman}
	
\cleardoublepage
\phantomsection
\addcontentsline{toc}{chapter}{Contents}
\tableofcontents
	
\cleardoublepage
\phantomsection
\addcontentsline{toc}{chapter}{List of Figures}
\listoffigures
	
\cleardoublepage
\phantomsection
\addcontentsline{toc}{chapter}{List of Tables}
\listoftables
	
\chapter{List of Acronyms}
	
\begin{tabbing}
\hspace{1.5in} \= \\
FDS \> Fire Dynamics Simulator \\
HGL \> Hot Gas Layer \\
HRR \> Heat Release Rate \\
HRRPUA \> Heat Release Rate per Unit Area \\
NIST \> National Institute of Standards and Technology \\
\end{tabbing}
	
\mainmatter
	
\chapter{Abstract}
\label{chap:Abstract} 

\chapter{Introduction}
\label{chap:Introduction}

\chapter{Experimental Setup}
\label{chap:Experimental_Setup}

\section{Experimental Facility}
\label{sec:Experimental_Facility}

Two identical concrete structures were built on a concrete slab as shown in Figure 1. They were designed to simulate a single floor of a residential structure.  The outer wall of each structure was composed of interlocking concrete blocks 0.61 m (2 ft) wide, 0.61 m (2 ft) high and 1.22 m (4 ft) long.  The interior dimensions of teach structure were 6.1 m (20 ft) wide, 11 m (36 ft) long and 2.4 m (8 ft) high.  The joints and gaps between the blocks were filled with high temperature insulation.

	\begin{figure}[!ht]
		\includegraphics[width=6in]{../Figures/Pictures/DelCo_Structures}
		\caption{Delaware County, PA Fire Test Structures}
		\label{fig:Delaware_County,_PA_Fire_Test_Structures}
	\end{figure}

Each structure had burn room, approximately 6.0 m (19.6 ft) by 4.4 m (14.3 ft) and 2.74 m (9.0 ft) in height and a hallway connecting the burn room to the front of the structure via a small entry foyer.  The hallway is approximately 0.95 m (3.1 ft) wide and 5.4 m (17.6 ft) and the entry foyer is 1.2 m (4.1 ft) by 1.9 m (6.1 ft).  The ceiling height in the hallway and entry foyer was 2.4 m (8 ft).  The open door on north side from the exterior to the entry foyer was 2.0 m (6.5 ft) high and 0.9 m (2.9 ft) wide.  The opening on the south face of the structure from the exterior to the burn room was 2.4 m (7.75 ft) high and 1.09 m (3.6 ft) wide.   A schematic plan view of the structure is given in Figure     .    

The floor of the structure was a concrete pad.  The interior walls of the structure were framed with steel studs and track.  The studs were set to 0.40 m (16 in) centers.  The ceiling support was composed of wood truss joist I-beams (TJIs) with a 299 mm (11.75 in) depth.  The TJI was composed of laminated veneer lumber flanges with a cross section of 29 mm (1.125 in) x 44 mm (1.75 in) and an 11 mm (0.43 in) thick oriented strand board web as shown in .  Tongue and grove, 18.3 mm (0.72 in) thick, oriented strand board was attached to the top of the TJIs.     

The interior walls of the burn room were lined with 13 mm (0.5 in) thick cement board.  The ceiling of the burn room was the exposed “floor assembly”.  The walls of the hallway and entry foyer were composed of 16 mm (0.625 in) Type X gypsum room. The ceiling of the hallway and entry foyer was composed of two layers of 13 mm (0.5 in) thick cement board.    

Both structures were similar and were used to run comparative experiments on a given day.

\section{Instrumentation}
\label{sec:Instrumentation}

The structures were instrumented for temperature, heat flux, and gas velocity measurements.  A schematic plan view of the instrumentation arrangement is show in ????  

\subsection{Temperature}
\label{subsec:Temperature}

Gas temperatures in the burn room were measured with bare-bead, Chromel-Alumel (type K) thermocouples, with a 0.5 mm (0.02 in) nominal diameter.  Thermocouple arrays were installed 2.0 m (6.7 ft) from both the east and the west walls of the burn room and 2.1 m (6.9 ft) from the south wall as shown in Figure ..  The vertical arrays had thermocouples located 0.03, 0.3, 0.61, 0.91, 1.22, 1.52, 1.83, 2.13 m below the bottom edge of the joist, where a ceiling would be installed.   

Additional single thermocouples were installed in conjunction with the bi-directional probes at the exterior vents and in the hallway.  The single thermocouples were bare-bead, Chromel-Alumel (type K) thermocouples, with a 1.0 mm (0.04 in) nominal diameter. Just behind the bead, the thermocouple wire is protected with an 3.2 mm (0.125 in) diameter inconel sheath.  
In the north doorway, there were seven thermocouples were located at: 0.15 m (0.5 ft), 0.30 m (1 ft), 0.61 m (2 ft), 0.91 m (3 ft), 1.22 m (4 ft), 1.51 m (5.0 ft), and 1.83 (6.0 ft) below the soffit along the vertical centerline of the doorway opening.  In the hallway, there were seven thermocouples positioned vertically at: 0.15 m (0.5 ft), 0.30 m (1 ft), 0.61 m (2 ft), 0.91 m (3 ft), 1.22 m (4 ft), 1.51 m (5.0 ft), and 1.83 (6.0 ft)   In the south window opening, there were four thermocouples are located 0.28 m (0.95 ft), 0.58 (1.90 ft), 0.84 m (2.85 ft), and 1.16 m (3.8 ft) below the soffit.  

\subsection{Heat Flux}
\label{subsec:Heat_Flux}

Total heat flux was measured with Schmidt Boelter gauges.  Radiant heat flux was measured with Schmidt Boelter gauges with sapphire lens.  One of each was located near the thermocouple arrays in the burn room.  The gauges were pointed at the ceiling and positioned 0.15 m (6 in) above the floor.  

In the northwest corner of the burn room two total heat flux gauges are located approximately 0.61 m (2 ft) out of the corner and 1.52 m (5 ft) below the ceiling, a position chosen to be representative of the height of a crawling firefighter’s head.  One of the gauge’s sensing surface is facing the ceiling and the other gauge is facing the pallets.  A similar heat flux gauge installation is located in the hallway, approximately 3.81 m (12.5 ft) from the burn room and 0.3 m (1ft) from the east wall of the hallway as shown in figure xx.  Again on gauge was facing the ceiling and one was facing the burn room.  

\subsection{Gas Velocity}
\label{subsec:Gas_Velocity}

Gas velocity was determined utilizing differential pressure transducers connected to bidirectional velocity probes[ ] in conjunction with a temperature measurement.  These probes were co-located with the sheathed thermocouples in the north doorway, the south window opening and the hallway.  The locations are shown in ??????

\subsection{Mass}
\label{subsec:Mass}

\subsection{Moisture Content}
\label{subsec:Moisture_Content}

\subsection{Infrared Radiation}
\label{subsec:Infrared_Radiation}

\section{Fuel Load: Wood Experiments}
\label{sec:Fuel_Load:_Wood_Experiments} 

The fuel load for the experiments consisted of one bale of hay, twelve pallets, eleven and a half sheets plywood on the walls and the wood “floor assembly”.
Two stacks of pallets with hay were used as the first items ignited in each experiment.  Each stack was composed of six pallets with a half bale of hay.  The hay was layered between each pallet.  The pallets and hay were weighed prior to each experiment.  For the four experiments, the average mass of the two stacks of pallets and the bale of hay was 232.4 kg (511.3 lbs).  A complete listing of the mass of each pallet and hay bales is given in Appendix ????????
  
The pallet stacks were located 0.3 m (1ft) away from the east and south walls of the burn room.  The pallets were 1.2 m (48 in) long, 1.0 m (40 in ) and 123 mm (4.8 in) high.  The stacks were 150 mm (6 in) apart.  One layer of 13 mm (0.5 in) thick gypsum board panels were laid on the concrete floor under the wood pallets to form a protective layer to minimize thermal damage to the concrete floor.
The east and south wall of the burn room was covered with 15 mm (0.59 in) thick, sheets of plywood.  The “wood flooring assembly”, composed of 12 TGIs and 9 sheets of OSB, served as the ceiling of the burn room.  The open doorway to the burn room on the south side was covered with a 2.4 m (8 ft) tall by 1.2 m (4 ft) wide piece of medium density fiberboard paneling that was 5 mm (0.19 in) thick.  The southeast corner of the burn room with the fuel load installed is shown in Figure 3.

\begin{figure}[!ht]
	\includegraphics[width=6in]{../Figures/Pictures/Wood_Fuel_Package}
	\caption{Photograph of Southeast corner of burn room with wood fuel load.}
	\label{fig:Wood_Fuel_Load}
\end{figure}

\section{Fuel Load: Furniture Experiments}
\label{sec:Fuel_Load:_Furniture_Experiments}

\begin{sidewaystable}[!ht]
	\centering
	\caption{Fuel masses}
	\begin{tabular}{llcc}
		\hline\noalign{\smallskip}
		Item                         &  Material Description             &  Dimensions (m)            &  Mass (kg)  \\
		\noalign{\smallskip}\hline\noalign{\smallskip}
		% 3-seat Legacy Sofa         &  491 Brawley                      &  1.96 m x 0.86 m x 0.89 m  &  56.2 kg    \\
		%   Cushion (left)           &  Cotton w/ inner springs          &                            &  4.06 kg    \\
		%   Cushions (center)        &  Cotton w/ inner springs          &                            &  4.56 kg    \\
		%   Cushions (rear)          &  Cotton w/ inner springs          &                            &  4.16 kg    \\
		% 2-seat New Sofa            &  491 Brawley                      &  1.59 m x 0.91 m x 0.86 m  &  33.3 kg    \\
		%   Cushion (seat)           &                                   &  0.56 m x 0.61 m x 0.14 m  &  1.6 kg     \\
		%   Cushion (rear)           &  0.71 m at top, 0.58 m at bottom  &  Varies x 0.53 m x 0.20 m  &  2.0 kg     \\
		3-seat Brown Sofa            &  302 College (front room)         &  2.24 m x 1.02 m x 0.91 m  &  61.0 kg    \\
		2-seat Purple Stripe Sofa    &  302 College (front room)         &  1.57 m x 0.97 m x 0.86 m  &  82.0 kg    \\
		3-seat Black Pine Leaf Sofa  &  302 College (front room)         &  2.08 m x 0.86 m x 0.86 m  &  51.0 kg    \\
		Blue/Brown Chair             &  302 College (front room)         &  0.97 m x 0.81 m x 0.97 m  &  46.4 kg    \\
		2-seat Floral Sofa           &  302 College (rear room)          &  1.68 m x 0.91 m x 0.91 m  &  34.5 kg    \\
		2-seat Blue Plaid Sofa       &  302 College (rear room)          &  1.63 m x 0.97 m x 0.91 m  &  39.8 kg    \\
		3-seat Yellow Sofa           &  302 College (rear room)          &  2.13 m x 0.91 m x 0.71 m  &  48.3 kg    \\
		3-seat Floral Sofa           &  302 College (rear room)          &  2.29 m x 0.91 m x 0.91 m  &  48.0 kg    \\
		3-seat New Sofa              &  289 College (front room)         &  2.13 m x 0.91 m x 0.91 m  &  43.0 kg    \\
		Blue Chair                   &  289 College (middle room)        &  0.91 m x 0.91 m x 0.91 m  &  25.3 kg    \\
		3-seat Floral Sofa           &  289 College (sprinkler room)     &  2.24 m x 1.02 m x 1.02 m  &  81.2 kg    \\
		Floral Chair                 &  289 College (sprinkler room)     &  1.02 m x 1.02 m x 1.02 m  &  36.3 kg    \\
		\noalign{\smallskip}\hline
	\end{tabular}
	\label{tab:Fuel_Masses}
\end{sidewaystable}

\chapter{Experiments and Results}
\label{chap:Experiments_and_Results}

\section{Gas Cooling}
\label{sec:Gas_Cooling}

\subsection{Spray Density}
\label{subsec:Spray_Density}

\section{Fire Suppression}
\label{sec:Fire_Suppression}	
	
\subsection{Wood Fuel Package}
\label{subsec:Wood_Fuel_Package}

\subsection{Furniture Fuel Package}
\label{subsec:Furniture_Fuel_Package}

\subsection{Spray Density}
\label{subsec:Spray_Density}
	
\chapter{Discussion and Tactical Considerations}
\label{chap:Discussion_and_Tactical_Considerations}

\chapter{Summary}
\label{chap:Summary}

\chapter{References}
\label{chap:References}

\chapter{Appendix}
\label{chap:Appendix}


\clearpage

\bibliography{../../../Bibliography/FDS_refs,../../../Bibliography/FDS_general,references}

	
\end{document}

