\documentclass[12pt,oneside]{book}

%%%%%%%%%%%%%%%%%%%%%%%%%%%%%%%%%%%%%%%%%%%%%%%%%%%%%%%%%%%%%%%%%%%%%%%%%%%%%%%%%%%%%%%%%%%%%%%%%%%
%                                                                                                 %
% The mathematical style of these documents follows                                               %
%                                                                                                 %
% A. Thompson and B.N. Taylor. The NIST Guide for the Use of the International System of Units.   %
%    NIST Special Publication 881, 2008.                                                          %
%                                                                                                 %
% http://www.nist.gov/pml/pubs/sp811/index.cfm                                                    %
%                                                                                                 %
%%%%%%%%%%%%%%%%%%%%%%%%%%%%%%%%%%%%%%%%%%%%%%%%%%%%%%%%%%%%%%%%%%%%%%%%%%%%%%%%%%%%%%%%%%%%%%%%%%%

\input{../../../Bibliography/commoncommands}

% Rename chapter headings
\renewcommand{\chaptername}{Section}
\renewcommand{\bibname}{References}

% Math shortcuts
\renewcommand{\sb}[1]{_\mathrm{#1}}
\renewcommand{\C}{\mbox{C}}
\renewcommand{\H}{\mbox{H}}
\renewcommand{\O}{\mbox{O}}
\newcommand{\N}{\mbox{N}}

% Center all figures
\makeatletter
\g@addto@macro\@floatboxreset\centering
\makeatother

% Extra packages
\usepackage{xfrac}

\begin{document}
	
\bibliographystyle{unsrt}
\pagestyle{empty}
	
\begin{minipage}[t][9in][s]{6.25in}
	
\begin{flushright}
\fontsize{20}{24}\selectfont
\bf{NIST Technical Note XXXX}
\end{flushright}
		
\headerB{
Examination of Compressed Air Foam (CAF) for Interior Firefighting \\
Spring 2012 \\
}
		
\normalsize
		
\headerC{
{
\flushright{
Daniel Madrzykowski \\
Keith M. Stakes \\
					
\vspace*{2\baselineskip}
				
\begingroup
This publication is available free of charge from:
\hypersetup{urlcolor=black}
\href{http://dx.doi.org/10.6028/NIST.TN.XXXX}{http://dx.doi.org/10.6028/NIST.TN.XXXX}
\endgroup
}
				
\vfill
				
\flushright{
		
\includegraphics[width=2.in]{../../../Bibliography/nistident_flright_vec} \\[.3in]
}
}
}
		
\end{minipage}

\newpage
\hspace{5in}
\newpage
	
\frontmatter
	
\pagenumbering{roman}
	
\begin{minipage}[t][9in][s]{6.25in}
		
\begin{flushright}
\fontsize{20}{24}\selectfont
\bf{NIST Technical Note XXXX}
\end{flushright}
		
\headerB{
Examination of Compressed Air Foam (CAF) for Interior Firefighting \\
Spring 2012 \\
}
		
\headerC{
\flushright{
Daniel Madrzykowski \\
Keith M. Stakes \\
{\em Fire Research Division \\
Engineering Laboratory} \\
				
\vspace*{2\baselineskip}
				
\begingroup
This publication is available free of charge from:
\hypersetup{urlcolor=black}
\href{http://dx.doi.org/10.6028/NIST.TN.XXXX}{http://dx.doi.org/10.6028/NIST.TN.XXXX} \\
\endgroup
				
\vspace*{2\baselineskip}
August 2014}}
		
\vfill
		
\flushright{\includegraphics[width=1in]{../../../Bibliography/doc} }
	
\titlesigs
		
\end{minipage}
	
\newpage
	
\begin{minipage}[t][9in][s]{6.25in}
		
\flushright{Certain commercial entities, equipment, or materials may be identified in this \\
document in order to describe an experimental procedure or concept adequately. \\
Such identification is not intended to imply recommendation or endorsement by the \\
National Institute of Standards and Technology, nor is it intended to imply that the \\
entities, materials, or equipment are necessarily the best available for the purpose. \\
}
		
\vspace{3in}
		
\large
\flushright{\bf National Institute of Standards and Technology Technical Note XXXX \\
Natl.~Inst.~Stand.~Technol.~Tech.~Note~XX, \pageref{LastPage} pages (August 2014) \\
% http://dx.doi.org/10.6028/NIST.TN.XXXX \\
CODEN: NTNOEF }

\vspace{0.2in}

\begingroup
{\bf This publication is available free of charge from:}
\hypersetup{urlcolor=black}
\href{http://dx.doi.org/10.6028/NIST.TN.1838}{\bf http://dx.doi.org/10.6028/NIST.TN.1838} \\
\endgroup		

\vfill
		
\hspace{1in}
		
\end{minipage}
	
\newpage
	
\frontmatter
	
\pagestyle{plain}
\pagenumbering{roman}
	
\cleardoublepage
\phantomsection
\addcontentsline{toc}{chapter}{Contents}
\tableofcontents
	
\cleardoublepage
\phantomsection
\addcontentsline{toc}{chapter}{List of Figures}
\listoffigures
	
\cleardoublepage
\phantomsection
\addcontentsline{toc}{chapter}{List of Tables}
\listoftables
	
\chapter{List of Acronyms}
	
\begin{tabbing}
\hspace{1.5in} \= \\
FDS \> Fire Dynamics Simulator \\
HGL \> Hot Gas Layer \\
HRR \> Heat Release Rate \\
HRRPUA \> Heat Release Rate per Unit Area \\
NIST \> National Institute of Standards and Technology \\
\end{tabbing}
	
\mainmatter
	
\chapter{Abstract}
\label{chap:Abstract} 

Traditionally, fire fighting suppression operations have been conducted from the interior of the structure as a means to reduce water damage, manage the “thermal balance,” and limit fire damage to structures.  In current fire fighter training manuals, it is stated that these operations must be coordinated with the ventilation operations.  However guidance on that coordination has been limited.  Previous research and examinations of line of duty deaths have shown that ventilation events occurring with fire fighters in the structure prior to suppression have led to tragic results.  One means of reducing the possibilities of this occurrence would be to begin to extinguish the fire prior to entering the structure.  In effect, an offensive, exterior fire attack, in which water is directed into the structure from the exterior to cool the fire gases and reduce the heat release rate of the fire, prior to the fire fighters entering the building.  The major concern with this type of operation is the potential harm that might occur to people trapped in the structure or the amount of water damage to the structure.  Therefore, measurements are needed to document the changes of the thermal environment within the structure due to interior and exterior fire suppression tactics.  This report will focus on water distribution and fire suppression experiments conducted to develop a baseline for the interior fire suppression.

\chapter{Introduction}
\label{chap:Introduction}
     
NIST FFTG

Objective: To improve the safety and effectiveness of firefighters through measurement science to advance suppression tactics, examine non-traditional means of fire suppression, and transfer of the results to the fire service. 

Aid the fire service across the country in the most effective and safe way to fight fires.

The goal is to improve the safety of not only the responding firefighters but the public as well through the use of modern fire suppression tactics for modern fires.

Approach: Real scale, instrumented fire suppression experiments. 

This is an STRS project that has been complemented with OA studies funding from: 
Cal Poly 
FDNY 
ISFSI 
NYU 

NIST has conducted a significant amount of research examining how ventilation affects the growth and spread of fire within structures and how the air flow to the fire may be controlled to limit or delay the growth of the fire.  The studies have resulted in guidance to the fire service regarding ventilation tactics.  However ventilation tactics alone will not result in the complete extinguishment of the fire, fire suppression with hose streams are needed.

Fire suppression tactics using hose streams also affect the ventilation in a structure and can impact the movement of smoke and heat through a structure as vents are made to advance the line or if ventilation inducing hand line tactics are in practice.  This research addressing the coordination of suppression tactics and the impact on ventilation is needed to complete recommendations on fire control tactics to appropriate standards, education, and training documents.

Test Objectives:

1)	Examine the water distribution for solid streams, straight streams, and fog streams.

2)	Examine the impact of solid streams, straight streams, and fog streams on the cooling of fire gases inside a structure with various flow path configurations, structure configurations, and structure volumes.

3)  Examine the impact of solid streams, straight streams, and fog streams on the 

Constants: 
Building Geometry
Fuel Load
Fire Room Conditions at time of water flow
Water Flow Rate – 120 gpm

Variable: 
Smooth Bore vs Narrow Fog

The Engineering Laboratory at National Institute of Standards and Technology (NIST) has a research program aimed at improving the safety and effectiveness of firefighters through improved knowledge of fire behavior and firefighting tactics.  One of the objectives of this program is to examine non-traditional means of fire suppression.   This series of experiments conducted by NIST for the California Polytechnic State University (Cal Poly), San Luis Obispo in collaboration with the Montgomery County (MD) Fire and Rescue Service complements the NIST research program on firefighting tactics.  These experiments are part of a larger research project led by Cal Poly with the purpose of investigating the capabilities and limitations of compressed air foam systems (CAFS) for structural firefighting with the objective of enhancing the scientific knowledge base regarding the effectiveness and safety implications associated with the use of CAFS for structural firefighting.
  
NIST conducted several series of experiments to provide scientific measurements and comparisons between CAFS and water-only hose streams for different compartment fire scenarios.  

Three types of experiments were conducted; 

1)            spray density measures under non-fire conditions 
2)            gas layer cooling 
3)            fire suppression in an experimental structure

The key measures were to understand the cooling and suppression capabilities of CAFS compared to water.   The experimental designs were developed based on a project planning workshop which included the discussion of incidents where firefighters had concerns or experienced minor burns while using CAFS during interior attack operations in residential occupancies.  [REF-  PROJECT ON “CAPABILITIES AND LIMITATIONS OF COMPRESSED AIR FOAM SYSTEMS (CAFS) FOR STRUCTURAL FIREFIGHTING” WORKSHOP SUMMARY, December, 13-14, 2011.   Casey C. Grant, P.E., Fire Protection Research Foundation, One Batterymarch Park, Quincy, MA USA 02169-7471
17 January 2012]

Since Montgomery County (MD) Fire and Rescue Service provided the CAFS equipped engines and the staffing needed to provide both the water and CAF for the experiments.  Baseline operating conditions were agreed upon at the workshop for these experiments. 

*Nozzles – Metro 1, fixed gallonage fog/straight/7/8”solid stream  
100 feet of hose line – 1-3/4”
Foam concentration – 0.3%
CAF Flowrate – 120 gpm / 60 cfm 
Water / air pressure (100-120 psi)

2.3. Study objective
2.3.1. FF exposure – usability / safety
2.3.1.1. Steam blowback
2.3.1.2. Knockdown time
2.3.1.3. Smoke scrubbing
2.3.1.4. Rekindle time / prevention
2.3.1.5. Overhaul / exposure time
2.3.2. Other Issues
2.3.2.1. Structural fire testing
2.3.2.2. Room / attic configuration

These experiments were conducted at the Delaware County Emergency Services Training Center (ESTC) in Sharon Hill, PA.  

\chapter{Experimental Setup}
\label{chap:Experimental_Setup}

\section{Experiment Descriptions}
\label{sec:Experiment_Descriptions}

The experimental set-up for this test series is described by the three types of experiments conducted: spray density measures, gas layer cooling, and fire suppression.

\subsection{Spray Density Measures}
\label{sec:Spray_Density_Measures}

Thirty two tests were conducted to determine spray density measures based on various nozzle patterns and locations. The purpose of these tests was to determine the appropriate nozzle locations for the gas cooling and fire suppression tests to follow by gaining an understanding of the water distribution throughout the structures. The gas cooling and fire suppression tests simulated an interior fire attack through the deployment of hose lines inside the structures with fixed nozzles. This set-up is representative of a firefighter operating on a hose line within the structure. Therefore, it is critical to ensure proper nozzle location for the tests to follow to provide the most realistic test conditions.     

The nozzles chosen for these experiments were as a result of those available through the MCFRS provided engine and decided upon during the CAFS workshop.  This test series used Task Force Tips (TFT) Metro 1, Fixed Gallonage nozzles with a flow rate of 120 gpm. These are automatic nozzles which have a 7/8 inch slug-tip and can be varied from a straight stream to a wide fog for application. The set-up also included 100 ft of 1-3/4 inch hose.  The nozzles were mounted to a "blitz-fire" monitor device and placed inside the structures.  The nozzles were controlled and operated via a gate valve from the exterior. 

The tests conducted for spray density measurement were divided between the available structures at the Delaware County ESTC. Twenty one of the tests were conducted in the concrete burn building where the gas cooling tests would take place, and eleven of the tests were conducted in the experimental structures to be used for the fire suppression tests. The tests varied by changing the nozzle height and location within the structure. Additionally, the nozzle pattern was varied as well. A description and drawings of these facilities can be seen in the Experimental Facility section below.

\begin{figure}[!ht]
	\includegraphics[width=6in]{../Figures/Pictures/Flows}
	\caption{Spray Density Nozzle Patterns}
	\label{fig:Spray_Density_Nozzle_Patterns}
\end{figure}

The experiments were conducted by placing interlocking collector bins across the floor surface area of the burn rooms within each structure. Each collector bin was a 2.5 square foot poly box. The building was sheathed in plastic for the spray density tests in order to protect and preserve as much of the structure as possible before any fire suppression tests were conducted. The nozzles were flowed for a set duration and once the test was concluded, the bins were removed from the structure.  The amount of water in each bin was measured in order to determine how much of the water ended up in each part of the room dependent on the nozzle location and pattern. The results can be seen in the sections to follow.

\begin{figure}[!ht]
	\includegraphics[width=6in]{../Figures/Pictures/Spray_Distribution}
	\caption{Collector Bins and Plastic Sheathing in Experimental Structures}
	\label{fig:Spray_Density_Set_Up}
\end{figure}

\subsection{Gas Cooling}
\label{sec:Gas_Cooling}

The purpose built fire training structure was used for a total of eighty eight experiments studying the effectiveness of cooling the upper gas layer in an adjacent space to the area of fire involvement. These tests examined five different hose stream patterns through two nozzle types for both compressed air foam and typical water suppression. The area of the purpose built structure used for these experiments was located on the ground floor and involved two adjoining rooms.  The fire was ignited and allowed to grow and fill the compartment with heated fire products and flame.  The developing hot gas layer spread to the adjacent room through a single doorway.  The adjacent room developed a hot gas layer and was the focus of this series of tests.  

\begin{figure}[!ht]
	\includegraphics[width=6in]{../Figures/Pictures/DelCoBurnBuildingSpray}
	\caption{Nozzle Location for Gas Cooling Tests}
	\label{fig:Nozzle_Location_Gas_Cooling_Tests}
\end{figure}



\subsection{Fire Suppression}
\label{sec:Fire_Suppression}

The fire suppression analysis included a series of 14 tests in two single story experimental structures, also located on the grounds of the Delaware County ESTC. Six tests were conducted in the East structure and six tests were conducted in the West structure, each including both water and CAF application as a suppression agent. Additionally, two tests were conducted, one in each structure, on suppression tactics for attic fires in residential dwellings. The nozzle patterns were varied from straight stream to narrow fog, and in the case of the two attic tests, a smooth bore nozzle was used. Several other factors including the fuel load and available ventilation were varied during these tests.

Several of the tests used a wood fuel load comprised of pallets and the remaining tests used a more realistic fuel load comprised of common household furnishings to include wooden wall paneling, two sofas, two overstuffed chairs, carpet, and padding.  The burn room was located to the rear of the single story structures and the available ventilation was both through the front doorway and from a rear window, dependent on the test in question.  The fires were intended to replicate room and contents fires typically seen within residential structures.  The hose lines were pre-deployed with fixed nozzles set in two locations.  One in the hallway on the approach to the burn room (hallway nozzle), and one just inside the burn room to the right side of the hallway (room nozzle). The hallway nozzle was intended to simulate a crew advancing into a structure towards the burn room. Suppression from this location would be in the form of gas cooling as the stream was not aimed at the seat of the fire, representing an indirect interior attack. The room nozzle was intended to simulate a crew which had advanced into a structure, entered the fire room, positioned just to the side of the entry door, and began suppression on the seat of the fire.  This position represented a direct interior attack.

\begin{figure}[!ht]
	\includegraphics[width=6in]{../Figures/Pictures/Spray_Hall}
	\caption{Hallway Nozzle Location for Fire Suppression Tests}
	\label{fig:Hallway_Nozzle_Location_Fire_Suppression_Tests}
\end{figure}

\begin{figure}[!ht]
	\includegraphics[width=6in]{../Figures/Pictures/Spray_Room}
	\caption{Room Nozzle Location for Fire Suppression Tests}
	\label{fig:Room_Nozzle_Location_Fire_Suppression_Tests}
\end{figure}

Due to the limited time and resources for testing, the extremes were chosen: smooth bore and narrow fog.

Straight stream from a fog nozzle was the medium between the two.

\section{Experimental Facility}
\label{sec:Experimental_Facility}

The series of tests described within this report were conducted at the Delaware County Emergency Services Training Center, located in Sharon Hill, PA. A burn building as well as two purpose built concrete structures are located on the grounds of the ESTC. The burn building was used for the gas cooling experiments and the purpose built concrete structures were used for the fire suppression experiments.   

\subsection{Purpose Built Fire Training Structure - Gas Cooling}
\label{sec:Burn_Building}

The burn building is a purpose-built live fire training structure comprised of both a two story and three story section. The building is supported with reinforced concrete beams and columns.  The floors and ceilings are also concrete with the interior walls constructed of cement block. The walls and ceiling of the burn rooms within the structure are protected with a 25 mm thick layer of calcium silicate insulation, which is covered by a 50 mm thick concrete tile. The floor of the burn rooms is protected with fire brick.

\begin{figure}[!ht]
	\includegraphics[width=6in]{../Figures/Pictures/burnbuilding}
	\caption{Delaware County, PA Burn Building}
	\label{fig:Delaware_County,_PA_Burn_Building}
\end{figure}

\begin{figure}[!ht]
	\includegraphics[width=6in]{../Figures/Pictures/DelCoBurnBuildingDimensions}
	\caption{Delaware County, PA Burn Building Layout}
	\label{fig:Delaware_County,_PA_Burn_Building_Layout}
\end{figure}

The two room configuration within the burn building had overall dimensions of 6.5 m (21.3 ft) by 9.2 m (30.2 ft). The burn room where the fuel load was located measured 3.8 m (12.5 ft) by 5.7 m (18.7 ft) with a ceiling height of 3.35 m (11.0 ft). The adjacent room where the water was applied for gas cooling measured 4.1 m (13.4 ft) by 5.7 m (18.7 ft), also with a ceiling height of 3.35 m (11.0 ft). The open door from the adjacent room to the exterior measured 2.0 m (6.5 ft) high and 0.9 m (2.9 ft) wide. 

\subsection{Concrete Structures - Fire Suppression}
\label{sec:Experimental Structures}

Two identical concrete structures were built on a concrete slab as shown in Fig.~\ref{sec:Experimental_Facility}. They were designed to simulate a single floor of a residential structure.  The outer wall of each structure was composed of interlocking concrete blocks 0.61 m (2 ft) wide, 0.61 m (2 ft) high and 1.22 m (4 ft) long.  The interior dimensions of each structure were 6.1 m (20 ft) wide, 11 m (36 ft) long and 2.4 m (8 ft) high.  The joints and gaps between the blocks were filled with high temperature insulation.

\begin{figure}[!ht]
	\includegraphics[width=6in]{../Figures/Pictures/DelCo_Structures}
	\caption{Delaware County, PA Fire Test Structures}
	\label{fig:Delaware_County,_PA_Fire_Test_Structures}
\end{figure}

Each structure had burn room, approximately 6.0 m (19.6 ft) by 4.4 m (14.3 ft) and 2.74 m (9.0 ft) in height and a hallway connecting the burn room to the front of the structure via a small entry foyer.  The hallway is approximately 0.95 m (3.1 ft) wide and 5.4 m (17.6 ft) and the entry foyer is 1.2 m (4.1 ft) by 1.9 m (6.1 ft).  The ceiling height in the hallway and entry foyer was 2.4 m (8 ft).  The open door on north side from the exterior to the entry foyer was 2.0 m (6.5 ft) high and 0.9 m (2.9 ft) wide.  The opening on the south face of the structure from the exterior to the burn room was 2.4 m (7.75 ft) high and 1.09 m (3.6 ft) wide.   A schematic plan view of the structure is given in Figure     .    

\begin{figure}[!ht]
	\includegraphics[width=6in]{../Figures/Pictures/DelCoSingleStory}
	\caption{Delaware County, PA Fire Test Structure}
	\label{fig:DelCoSingleStory}
\end{figure}

\begin{figure}[!ht]
	\includegraphics[width=6in]{../Figures/Pictures/DelCoSingleStoryDimensionsMetric}
	\caption{Floor Plan of the Test Structure.}
	\label{fig:Test_Structure_Floor_Plan}
\end{figure}
    
The floor of the structure was a concrete pad.  The interior walls of the structure were framed with steel studs and track.  The studs were set to 0.40 m (16 in) centers.  The ceiling support was composed of wood truss joist I-beams (TJIs) with a 299 mm (11.75 in) depth.  The TJI was composed of laminated veneer lumber flanges with a cross section of 29 mm (1.125 in) x 44 mm (1.75 in) and an 11 mm (0.43 in) thick oriented strand board web as shown in .  Tongue and grove, 18.3 mm (0.72 in) thick, oriented strand board was attached to the top of the TJIs.     

The interior walls of the burn room were lined with 13 mm (0.5 in) thick cement board.  The ceiling of the burn room was the exposed ``floor assembly".  The walls of the hallway and entry foyer were composed of 16 mm (0.625 in) Type X gypsum room. The ceiling of the hallway and entry foyer was composed of two layers of 13 mm (0.5 in) thick cement board.    

Both structures were similar and were used to run comparative experiments on a given day.

\section{Instrumentation}
\label{sec:Instrumentation}

\begin{figure}[!ht]
	\includegraphics[width=6in]{../Figures/Pictures/Legend2}
	\caption{Instrumentation Legend}
	\label{fig:Instrumentation_Legend}
\end{figure}

\subsection{Gas Cooling Instrumentation} 
\label{subsec:Gas_Cooling_Instrumentation}

\begin{figure}[!ht]
	\includegraphics[width=6in]{../Figures/Pictures/DelCoBurnBuildingInstrumentation}
	\caption{Photograph of the Instrumentation Dimensions in the Gas Cooling Experiments}
	\label{fig:Gas_Cooling_Instrumentation_Dimensions}
\end{figure}

\subsection{Fire Suppression Instrumentation}
\label{subsec:Fire_Suppression_Instrumentation}

The structures were instrumented for temperature, heat flux, and gas velocity measurements.  A schematic plan view of the instrumentation arrangement is show in ???? 

\begin{figure}[!ht]
	\includegraphics[width=6in]{../Figures/Pictures/DelCoSingleStoryInstrumentationDimensions}
	\caption{Photograph of the Instrumentation Dimensions in the Fire Suppression Experiments.}
	\label{fig:Fire_Suppression_Instrumentation_Dimensions}
\end{figure}

\begin{figure}[!ht]
	\includegraphics[width=6in]{../Figures/Pictures/Fog_Instrumentation}
	\caption{Photograph of the Instrumentation in the Fog Stream Tests.}
	\label{fig:Fog_Instrumentation}
\end{figure}

\begin{figure}[!ht]
	\includegraphics[width=6in]{../Figures/Pictures/Straight_Stream_Instrumentation}
	\caption{Photograph of the Instrumentation in the Straight Stream Tests.}
	\label{fig:Straigh_Stream_Instrumentation}
\end{figure}

\begin{figure}[!ht]
	\includegraphics[width=6in]{../Figures/Pictures/Single_Story_Instrumentation}
	\caption{Photograph of the Instrumentation in the Fire Suppression Tests.}
	\label{fig:Fire_Suppression_Instrumentation}
\end{figure}

\subsection{Temperature}
\label{subsec:Temperature}

Thermocouples:
Work through the Seebeck effect.  Two dissimilar metals are joined together with a weld bead.  Each metal generates a voltage, different from one another, when exposed to a thermal gradient.  This voltage difference increases with an increase in energy to the thermocouple and through a correlation, a corresponding temperature is found.  

\subsubsection{Gas Cooling Experiments}
\label{subsubsec:Gas_Cooling_Temperature_Instrumentation}

\subsubsection{Fire Suppression Experiments}
\label{subsubsec:Fire_Suppression_Temperature_Instrumentation}

Gas temperatures in the burn room were measured with bare-bead, Chromel-Alumel (type K) thermocouples, with a 0.5 mm (0.02 in) nominal diameter.  Thermocouple arrays were installed 2.0 m (6.7 ft) from both the east and the west walls of the burn room and 2.1 m (6.9 ft) from the south wall as shown in Figure ..  The vertical arrays had thermocouples located 0.03, 0.3, 0.61, 0.91, 1.22, 1.52, 1.83, 2.13 m below the bottom edge of the joist, where a ceiling would be installed.   

Additional single thermocouples were installed in conjunction with the bi-directional probes at the exterior vents and in the hallway.  The single thermocouples were bare-bead, Chromel-Alumel (type K) thermocouples, with a 1.0 mm (0.04 in) nominal diameter. Just behind the bead, the thermocouple wire is protected with an 3.2 mm (0.125 in) diameter inconnel sheath.  
In the north doorway, there were seven thermocouples were located at: 0.15 m (0.5 ft), 0.30 m (1 ft), 0.61 m (2 ft), 0.91 m (3 ft), 1.22 m (4 ft), 1.51 m (5.0 ft), and 1.83 (6.0 ft) below the soffit along the vertical centerline of the doorway opening.  In the hallway, there were seven thermocouples positioned vertically at: 0.15 m (0.5 ft), 0.30 m (1 ft), 0.61 m (2 ft), 0.91 m (3 ft), 1.22 m (4 ft), 1.51 m (5.0 ft), and 1.83 (6.0 ft)   In the south window opening, there were four thermocouples are located 0.28 m (0.95 ft), 0.58 (1.90 ft), 0.84 m (2.85 ft), and 1.16 m (3.8 ft) below the soffit.  

\subsection{Heat Flux}
\label{subsec:Heat_Flux}

Total Heat Flux:
Water cooled heat flux gauges measure both the convective and radiative heat transfer to the sensor.  Hot and cold side of a plate simulating the hot and cold side of a thermocouple.  The temperature gradient generates a voltage and through correlations, a total heat flux measurement is obtained.  The sensor is water cooled to ensure a thermal gradient remains and the sensor does not reach uniform temperature.  The sensor top is painted black to assist with radiation absorption.

Radiometer:
The radiometer is simply a total heat flux gauge with Zirconium plate atop the sensor preventing the device from recording any heat due to convective transfer.

\subsubsection{Gas Cooling Experiments}
\label{subsubsec:Gas_Cooling_Heat_Flux_Instrumentation}

\subsubsection{Fire Suppression Experiments}
\label{subsubsec:Fire_Suppression_Heat_Flux_Instrumentation}

Total heat flux was measured with Schmidt-Boelter gauges.  Radiant heat flux was measured with Schmidt Boelter gauges with sapphire lens.  One of each was located near the thermocouple arrays in the burn room.  The gauges were pointed at the ceiling and positioned 0.15 m (6 in) above the floor.  

In the northwest corner of the burn room two total heat flux gauges are located approximately 0.61 m (2 ft) out of the corner and 1.52 m (5 ft) below the ceiling, a position chosen to be representative of the height of a crawling firefighter's head.  One of the gauge's sensing surface is facing the ceiling and the other gauge is facing the pallets.  A similar heat flux gauge installation is located in the hallway, approximately 3.81 m (12.5 ft) from the burn room and 0.3 m (1ft) from the east wall of the hallway as shown in figure xx.  Again on gauge was facing the ceiling and one was facing the burn room.  

\subsection{Gas Velocity}
\label{subsec:Gas_Velocity}

Bi-directional Probes:
- Fire flow measurements are taken using these devices which have a high and low pressure side.  These flows then travel to a pressure transducer.  Through a simple conversion, the voltages are converted to pressure and finally a velocity based on the pressure and corresponding temperature measurements.

\subsubsection{Gas Cooling Experiments}
\label{subsubsec:Gas_Cooling_Gas_Velocity_Instrumentation}

\subsubsection{Fire Suppression Experiments}
\label{subsubsec:Fire_Suppression_Gas_Velocity_Instrumentation}

Gas velocity was determined utilizing differential pressure transducers connected to bidirectional velocity probes[ ] in conjunction with a temperature measurement.  These probes were co-located with the sheathed thermocouples in the north doorway, the south window opening and the hallway.  The locations are shown in ??????

\subsection{Mass}
\label{subsec:Mass}

\subsection{Moisture Content}
\label{subsec:Moisture_Content}

\subsection{Infrared Radiation}
\label{subsec:Infrared_Radiation}

\subsection{Uncertainty}
\label{subsec:Uncertainty}

There are different components of uncertainty in the length, mass, temperature, heat flux, gas concentration, differential pressure, gas velocity and heat release rate reported here. Uncertainties are grouped into two categories according to the method used to estimate them. Type A uncertainties are those which are evaluated by statistical methods, and Type B are those which are evaluated by other means  [ ]. Type B analysis of systematic uncertainties involves estimating the upper (+ a) and lower (- a) limits for the quantity in question such that the probability that the value would be in the interval (± a) is essentially 100 %. After estimating uncertainties by either Type A or B analysis, the uncertainties are combined in quadrature to yield the combined standard uncertainty. Then the combined standard uncertainty is multiplied by a coverage factor of two, which results in the expanded uncertainty with a 95 % confidence interval (2σ).  For some of these components, such as the zero and calibration elements, uncertainties are derived from referenced instrument specifications. For other components, referenced research results and past experience with the instruments provided input in the uncertainty determination. 

Each length measurement was taken carefully. Length measurements such as the room dimensions, instrumentation array locations and fire apparatus (for example nozzle, sprinkler, or fan) placement were made with a hand held laser measurement device which is has an accuracy of ± 6.0 mm (0.25) over a range of 0.61 m (2.00 ft) to 15.3 m (50.0 ft) [ ] .  However, conditions affecting the measurement, such as levelness of the device, yields an estimated uncertainty of ± 0.5 % for measurements in the 2.0 m (6.6 ft) to 10.0 m (32.8 ft) range.  Steel measuring tapes with a resolution of  ± 0.5 mm (0.02 in) were used to locate individual sensors within a measurement array and to measure and position the furniture. The steel measuring tapes were manufactured in compliance with NIST Manual 44, which specifies a tolerance of ±1.6 mm (0.06 in) for 9.1 m (30 ft) tapes and ±6.4 mm (0.25 in) for 30.5 m (100 ft) tapes [ ] .  Some issues, such as “soft” edges on the upholstered furniture, result in an estimated total expanded uncertainty of  ± 1.0 %. 

The load cell used to weigh the fuels prior to the experiments had a range of 0 kg (0 lbs) to 200 kg (440lbs) with a resolution of a 0.05 kg (0.11 lb) and a calibration uncertainty within 1% [] . The expanded uncertainty is estimated to be less than  ± 5 %. 
The standard uncertainty in temperature of the thermocouple wire itself is  ± 2.2 °C at 277 °C and increases to ± 9.5 °C at 871 °C as determined by the wire manufacturer [ ] .  The variation of the temperature in the environment surrounding the thermocouple is known to be much greater than that of the wire uncertainty [ ,  ] .  Small diameter thermocouples were used to limit the impact of radiative heating and cooling.  The estimated total expanded uncertainty for temperature in these experiments is ± 15 %.
In this study, total heat flux measurements were made with water-cooled Schimidt-Bolter gauges.  The manufacturer reports a ± 3 % calibration expanded uncertainty for these devices [ ] . Results from an international study on total heat flux gauge calibration and response demonstrated that the uncertainty of a Schmidt-Boelter gauge is typically ± 8 % [] .    

The gas measurement instruments and sampling system used in this series of experiments have been demonstrated an expanded (k = 2) relative uncertainty of ± 1 % when compared with span gas volume fractions [].   Given the non-uniformities and movement of the fire gas environment and the limited set of sampling points in these experiments an estimated uncertainty of ± 12 % is being applied to the results [ ]. 

Differential pressure reading uncertainty components were derived from pressure transducer
instrument specifications and previous experience with pressure transducers. The transducers
were factory calibrated and the zero and span of each was checked in the laboratory prior to the
experiments yielding an accuracy of ± 1 % [ ] . The total expanded uncertainty was estimated at 10 %.

Bi-directional probes and single thermocouples were used to measure the velocity.  The bi-directional probes used similar pressure transducers as those used for the differential pressure measurements discussed above.   Bare-bead Type K thermocouple are co-located with the probe. A gas velocity measurement study, examining the doorway flow of pre-flashover compartment fires, yielded expanded uncertainty measurements ranging from ± 0.14 to ± 0.22 for bi-directional probes of similar design[] .  The total expanded uncertainty for gas velocity in these experiments w estimated to be  ± 18 %.   
Water Flowrate
Water flowrate was measured with a pressure and flow meter combination shown in Figure x.x. The meter consists of a section of 6.35 cm (2.5 inch) cast aluminum pipe with a 0 – 4.1 MPa (0 - 600 psi) pressure transducer and a paddlewheel type flow sensor with a range of 0 to 4800 lpm (1250 gpm). The pressure transducer and paddlewheel both connect to the battery operated control box where the pressure transducer voltage is converted to a pressure and the paddlewheel pulse count is converted to a volumetric flow rate.  The manufacturer reports a ± 5 % calibration expanded uncertainty for the flow sensor and ± 3 %  for the pressure sensor [ ] . The pressure transducer was calibrated with a known analog pressure gauge. The flow meter was calibrated by capturing water over time and measuring that mass of water to determine the flowrate. The total expanded uncertainty was estimated at ± 10 %.

\section{Fuel Load: Gas Cooling}
\label{sec:Fuel_Load:_Gas_Cooling} 

Wood pallets were used to provide the fuel for the fire used in the gas cooling experiments.  The pallets were approximately   1.2 m (4.0 ft) by 1.0 m (3.3 ft) by 0.13 m (0.42 ft) thick.  Forty eight of the pallets used to fuel the gas cooling experiments were weighed.  The pallets ranged in mass from 13.6 kg (29.9 lbs) to 26.4 kg (58.1 lbs) with an average of 18.4 kg (40.5 lbs).  The initial fuel load consisted of 10 pallets, arranged in two stacks of five as shown in Fig xx.  Approximately one half of a bale of excelsior, 13.0  kg (28.6 lbs), was mixed with the pallets to aid with ignition.  

As the pallets burned burn away and the hot gas layer temperatures decreased, the steel shutters on the window to the fire room were opened and additional pallets were added to the piles of burning pallets until the flames from the piles reached the ceiling of the burn room again.  The number of pallets added each time varied.

\begin{figure}[!ht]
	\includegraphics[width=6in]{../Figures/Pictures/DelCoBurnBuildingFuelLoad}
	\caption{Burn Building Fuel Load.}
	\label{fig:Burn_Building_Fuel_Load}
\end{figure}

\section{Fuel Load: Wood Experiments}
\label{sec:Fuel_Load:_Wood_Experiments} 

The fuel load for the experiments consisted of one bale of hay, twelve pallets, eleven and a half sheets plywood on the walls and the wood ``floor assembly". Two stacks of pallets with hay were used as the first items ignited in each experiment.  Each stack was composed of six pallets with a half bale of hay.  The hay was layered between each pallet.  The pallets and hay were weighed prior to each experiment.  For the four experiments, the average mass of the two stacks of pallets and the bale of hay was 232.4~kg (511.3~lb).  A complete listing of the mass of each pallet and hay bales is given in Appendix ????????
  
The pallet stacks were located 0.3~m (12~in) away from the east and south walls of the burn room.  The pallets were 1.2~m (48~in) long, 1.0~m (40~in) and 123~mm (4.8~in) high.  The stacks were 150~mm (6~in) apart.  One layer of 13 mm (0.5 in) thick gypsum board panels were laid on the concrete floor under the wood pallets to form a protective layer to minimize thermal damage to the concrete floor.
The east and south wall of the burn room was covered with 15~mm (0.59~in) thick, sheets of plywood.  The ``wood flooring assembly", composed of 12 TGIs and 9 sheets of OSB, served as the ceiling of the burn room.  The open doorway to the burn room on the south side was covered with a 2.4~m (8~ft) tall by 1.2~m (4~ft) wide piece of medium density fiberboard paneling that was 5~mm (0.19~in) thick.  The southeast corner of the burn room with the fuel load installed is shown in Fig.~\ref{fig:Wood_Fuel_Load}.

\begin{figure}[!ht]
	\includegraphics[width=6in]{../Figures/Pictures/DelCoSingleStoryWoodFuelLoad}
	\caption{Wood Fuel Load Dimensions.}
	\label{fig:Wood_Fuel_Load_Dimensions}
\end{figure}

\begin{figure}[!ht]
	\includegraphics[width=6in]{../Figures/Pictures/Wood_Fuel_Package}
	\caption{Photograph of Southeast corner of burn room with wood fuel load.}
	\label{fig:Wood_Fuel_Load}
\end{figure}

\section{Fuel Load: Furniture Experiments}
\label{sec:Fuel_Load:_Furniture_Experiments}

\begin{figure}[!ht]
	\includegraphics[width=6in]{../Figures/Pictures/DelCoSingleStoryFurnitureFuelLoad}
	\caption{Furniture Fuel Load Dimensions.}
	\label{fig:Furniture_Fuel_Load_Dimensions}
\end{figure}

\begin{sidewaystable}[!ht]
	\centering
	\caption{Fuel masses}
	\begin{tabular}{llcc}
		\hline\noalign{\smallskip}
		Item                         &  Material Description             &  Dimensions (m)            &  Mass (kg)  \\
		\noalign{\smallskip}\hline\noalign{\smallskip}
		3-seat Legacy Sofa         &  491 Brawley                      &  1.96 m x 0.86 m x 0.89 m  &  56.2 kg    \\
		Cushion (left)           &  Cotton w/ inner springs          &                            &  4.06 kg    \\
		Cushions (center)        &  Cotton w/ inner springs          &                            &  4.56 kg    \\
		Cushions (rear)          &  Cotton w/ inner springs          &                            &  4.16 kg    \\
		2-seat New Sofa            &  491 Brawley                      &  1.59 m x 0.91 m x 0.86 m  &  33.3 kg    \\
		Cushion (seat)           &                                   &  0.56 m x 0.61 m x 0.14 m  &  1.6 kg     \\
		Cushion (rear)           &  0.71 m at top, 0.58 m at bottom  &  Varies x 0.53 m x 0.20 m  &  2.0 kg     \\
		3-seat Brown Sofa            &  302 College (front room)         &  2.24 m x 1.02 m x 0.91 m  &  61.0 kg    \\
		2-seat Purple Stripe Sofa    &  302 College (front room)         &  1.57 m x 0.97 m x 0.86 m  &  82.0 kg    \\
		3-seat Black Pine Leaf Sofa  &  302 College (front room)         &  2.08 m x 0.86 m x 0.86 m  &  51.0 kg    \\
		Blue/Brown Chair             &  302 College (front room)         &  0.97 m x 0.81 m x 0.97 m  &  46.4 kg    \\
		2-seat Floral Sofa           &  302 College (rear room)          &  1.68 m x 0.91 m x 0.91 m  &  34.5 kg    \\
		2-seat Blue Plaid Sofa       &  302 College (rear room)          &  1.63 m x 0.97 m x 0.91 m  &  39.8 kg    \\
		3-seat Yellow Sofa           &  302 College (rear room)          &  2.13 m x 0.91 m x 0.71 m  &  48.3 kg    \\
		3-seat Floral Sofa           &  302 College (rear room)          &  2.29 m x 0.91 m x 0.91 m  &  48.0 kg    \\
		3-seat New Sofa              &  289 College (front room)         &  2.13 m x 0.91 m x 0.91 m  &  43.0 kg    \\
		Blue Chair                   &  289 College (middle room)        &  0.91 m x 0.91 m x 0.91 m  &  25.3 kg    \\
		3-seat Floral Sofa           &  289 College (sprinkler room)     &  2.24 m x 1.02 m x 1.02 m  &  81.2 kg    \\
		Floral Chair                 &  289 College (sprinkler room)     &  1.02 m x 1.02 m x 1.02 m  &  36.3 kg    \\
		\noalign{\smallskip}\hline
	\end{tabular}
	\label{tab:Fuel_Masses}
\end{sidewaystable}

\chapter{Experiments and Results}
\label{chap:Experiments_and_Results}

\begin{sidewaystable}[!ht]
	\centering
	\caption{Temperature Comparison}
	\begin{tabular}{ll}
		\hline\noalign{\smallskip}
		Temperature				 & Response		\\
		\noalign{\smallskip}\hline\noalign{\smallskip}
		37 deg C (98.6 deg F)    &  Normal human oral/body temperature      \\
		44 deg C (111 deg F)   	 & Human skin begins to feel pain     \\
		48 deg C (118 deg F) 	 & Human skin receives a first degree burn injury      \\
		55 deg C (131 deg F)     & Human skin receives a second degree burn injury    \\
		62 deg C (140 deg F)     & A phase where burned human tissue becomes numb    \\
		72 deg C (162 deg F)     & Human skin instantly destroyed     \\
		100 deg C (212 deg F)    & Water boils and produces steam    \\
		140 deg C (284 deg F)    & Glass transition temperature of polycarbonate (SCBA Facepieces)      \\
		230 deg C (446 deg F)    & Melting temperature of polycarbonate (SCBA Facepieces)    \\
		250 deg C (482 deg F)    & Charring of natural cotton begins     \\
		>300 deg C (>572 deg F)  & Charring of modern protective clothing fabrics begins    \\
		>600 deg C (>1112 deg F) & Temperatures inside a post-flashover compartment fire     \\
		\noalign{\smallskip}\hline
	\end{tabular}
	\label{tab:Temperature_Comparison}
\end{sidewaystable}

\begin{sidewaystable}[!ht]
	\centering
	\caption{Heat Flux Comparison}
	\begin{tabular}{ll}
		\hline\noalign{\smallskip}
		Heat Flux	  				   & Response		\\
		\noalign{\smallskip}\hline\noalign{\smallskip}
		1 kW/m\textsuperscript{2}      & Sunny day      \\
		2.5 kW/m\textsuperscript{2}    & Typical firefighter exposure     \\
		3-5 kW/m\textsuperscript{2}    & Pain to skin within seconds      \\
		20 kW/m\textsuperscript{2}     & Threshold flux to floor at flashover onset    \\
		84 kW/m\textsuperscript{2}     & Thermal Protection Performance (TPP) test    \\
		60-200 kW/m\textsuperscript{2} & Flames over surface     \\
		\noalign{\smallskip}\hline
	\end{tabular}
	\label{tab:Heat_Flux_Comparison}
\end{sidewaystable}

\subsection{Spray Density}
\label{subsec:Spray_Density}

12 Experiments:
4 SS from room
2 Fog from room
2 SS from Hall
2 Fog from Hall
1 SB from Hall
1 SB from Room

Spray Distribution Summary:
- Experiments provided nozzle placement for gas cooling and fire suppression experiments.
- Provide a sense of the distribution of the water throughout the test space.

Hallway nozzle = gas cooling

Room nozzle = seat of fire

\begin{figure}[!ht]
	\includegraphics[width=6in]{../Figures/Bars/BB1}
	\caption{Burn Building Test 1}
	\label{fig:Burn_Building_Test_1}
\end{figure}

\begin{figure}[!ht]
	\includegraphics[width=6in]{../Figures/Bars/BB2}
	\caption{Burn Building Test 2}
	\label{fig:Burn_Building_Test_2}
\end{figure}

\begin{figure}[!ht]
	\includegraphics[width=6in]{../Figures/Bars/BB3}
	\caption{Burn Building Test 3}
	\label{fig:Burn_Building_Test_3}
\end{figure}

\begin{figure}[!ht]
	\includegraphics[width=6in]{../Figures/Bars/BB4}
	\caption{Burn Building Test 4}
	\label{fig:Burn_Building_Test_4}
\end{figure}

\clearpage

\begin{figure}[!ht]
	\includegraphics[width=6in]{../Figures/Bars/BB5}
	\caption{Burn Building Test 5}
	\label{fig:Burn_Building_Test_5}
\end{figure}

\begin{figure}[!ht]
	\includegraphics[width=6in]{../Figures/Bars/BB6}
	\caption{Burn Building Test 6}
	\label{fig:Burn_Building_Test_6}
\end{figure}

\begin{figure}[!ht]
	\includegraphics[width=6in]{../Figures/Bars/BB7}
	\caption{Burn Building Test 7}
	\label{fig:Burn_Building_Test_7}
\end{figure}

\begin{figure}[!ht]
	\includegraphics[width=6in]{../Figures/Bars/BB8}
	\caption{Burn Building Test 8}
	\label{fig:Burn_Building_Test_8}
\end{figure}

\clearpage

\begin{figure}[!ht]
	\includegraphics[width=6in]{../Figures/Bars/BB9}
	\caption{Burn Building Test 9}
	\label{fig:Burn_Building_Test_9}
\end{figure}

\begin{figure}[!ht]
	\includegraphics[width=6in]{../Figures/Bars/BB10}
	\caption{Burn Building Test 10}
	\label{fig:Burn_Building_Test_10}
\end{figure}

\begin{figure}[!ht]
	\includegraphics[width=6in]{../Figures/Bars/BB11}
	\caption{Burn Building Test 11}
	\label{fig:Burn_Building_Test_11}
\end{figure}

\begin{figure}[!ht]
	\includegraphics[width=6in]{../Figures/Bars/BB12}
	\caption{Burn Building Test 12}
	\label{fig:Burn_Building_Test_12}
\end{figure}

\clearpage

\begin{figure}[!ht]
	\includegraphics[width=6in]{../Figures/Bars/BB13}
	\caption{Burn Building Test 13}
	\label{fig:Burn_Building_Test_13}
\end{figure}

\begin{figure}[!ht]
	\includegraphics[width=6in]{../Figures/Bars/BB14}
	\caption{Burn Building Test 14}
	\label{fig:Burn_Building_Test_14}
\end{figure}

\begin{figure}[!ht]
	\includegraphics[width=6in]{../Figures/Bars/BB15}
	\caption{Burn Building Test 15}
	\label{fig:Burn_Building_Test_15}
\end{figure}

\begin{figure}[!ht]
	\includegraphics[width=6in]{../Figures/Bars/BB16}
	\caption{Burn Building Test 16}
	\label{fig:Burn_Building_Test_16}
\end{figure}

\clearpage

\begin{figure}[!ht]
	\includegraphics[width=6in]{../Figures/Bars/BB17}
	\caption{Burn Building Test 17}
	\label{fig:Burn_Building_Test_17}
\end{figure}

\begin{figure}[!ht]
	\includegraphics[width=6in]{../Figures/Bars/BB18}
	\caption{Burn Building Test 18}
	\label{fig:Burn_Building_Test_18}
\end{figure}

\begin{figure}[!ht]
	\includegraphics[width=6in]{../Figures/Bars/BB19}
	\caption{Burn Building Test 19}
	\label{fig:Burn_Building_Test_19}
\end{figure}

\begin{figure}[!ht]
	\includegraphics[width=6in]{../Figures/Bars/BB20}
	\caption{Burn Building Test 20}
	\label{fig:Burn_Building_Test_20}
\end{figure}

\begin{figure}[!ht]
	\includegraphics[width=6in]{../Figures/Bars/BB21}
	\caption{Burn Building Test 21}
	\label{fig:Burn_Building_Test_21}
\end{figure}

\clearpage

\begin{figure}[!ht]
	\includegraphics[width=6in]{../Figures/Bars/ES1}
	\caption{Experimental Structures Test 1}
	\label{fig:Experimental_Structures_Test_1}
\end{figure}

\begin{figure}[!ht]
	\includegraphics[width=6in]{../Figures/Bars/ES2}
	\caption{Experimental Structures Test 2}
	\label{fig:Experimental_Structures_Test_2}
\end{figure}

\begin{figure}[!ht]
	\includegraphics[width=6in]{../Figures/Bars/ES3}
	\caption{Experimental Structures Test 3}
	\label{fig:Experimental_Structures_Test_3}
\end{figure}

\begin{figure}[!ht]
	\includegraphics[width=6in]{../Figures/Bars/ES4}
	\caption{Experimental Structures Test 4}
	\label{fig:Experimental_Structures_Test_4}
\end{figure}

\clearpage

\begin{figure}[!ht]
	\includegraphics[width=6in]{../Figures/Bars/ES5}
	\caption{Experimental Structures Test 5}
	\label{fig:Experimental_Structures_Test_5}
\end{figure}

\begin{figure}[!ht]
	\includegraphics[width=6in]{../Figures/Bars/ES6}
	\caption{Experimental Structures Test 6}
	\label{fig:Experimental_Structures_Test_6}
\end{figure}

\begin{figure}[!ht]
	\includegraphics[width=6in]{../Figures/Bars/ES7}
	\caption{Experimental Structures Test 7}
	\label{fig:Experimental_Structures_Test_7}
\end{figure}

\begin{figure}[!ht]
	\includegraphics[width=6in]{../Figures/Bars/ES72}
	\caption{Experimental Structures Test 7.2}
	\label{fig:Experimental_Structures_Test_7.2}
\end{figure}

\clearpage

\begin{figure}[!ht]
	\includegraphics[width=6in]{../Figures/Bars/ES8}
	\caption{Experimental Structures Test 8}
	\label{fig:Experimental_Structures_Test_8}
\end{figure}

\begin{figure}[!ht]
	\includegraphics[width=6in]{../Figures/Bars/ES9}
	\caption{Experimental Structures Test 9}
	\label{fig:Experimental_Structures_Test_9}
\end{figure}

\begin{figure}[!ht]
	\includegraphics[width=6in]{../Figures/Bars/ES10}
	\caption{Experimental Structures Test 10}
	\label{fig:Experimental_Structures_Test_10}
\end{figure}

\begin{figure}[!ht]
	\includegraphics[width=6in]{../Figures/Bars/ES11}
	\caption{Experimental Structures Test 11}
	\label{fig:Experimental_Structures_Test_11}
\end{figure}

\clearpage

\section{Gas Cooling}
\label{sec:Gas_Cooling}

\section{Fire Suppression}
\label{sec:Fire_Suppression}	

\begin{sidewaystable}[!ht]
	\centering
	\caption{Test Descriptions}
	\begin{tabular}{lllll}
		\hline\noalign{\smallskip}
		Test		& Agent		& Stream				& Flow Rate			& Comments \\
		\noalign{\smallskip}\hline\noalign{\smallskip}
		1 (FSE1)    &  CAF      &  30 deg Fog  			&  120 gpm/60 cfm   &	Furniture Fuel Package  \\
		2 (FSW2)   	& Water     &  30 deg Fog  			&  120 gpm    		&	Furniture Fuel Package	\\
		3 (FSW3) 	&  CAF      &  7/8" Smooth Bore  	&  120 gpm/60 cfm   &	Furniture Fuel Package  \\
		4 (FSE4)    &  Water    &  7/8" Smooth Bore  	&  120 gpm    		&	Furniture Fuel Package	\\
		5 (FSE5)    &  Water    &  30 deg Fog  			&  120 gpm    		&	Furniture Fuel Package	\\
		6 (FSW6)    &  CAF      &  30 deg Fog  			&  120 gpm/60 cfm   &	Furniture Fuel Package  \\
		7 (FSW7)    &  Water    &  30 deg Fog  			&  120 gpm    		&	Furniture Fuel Package	\\
		8 (FSE8)    &  CAF      &  30 deg Fog  			&  120 gpm/60 cfm   &	Furniture Fuel Package  \\
		9 (Attic 1) &  Water    &  Straight Stream  	&  120 gpm    		&	Furniture Fuel Package	\\
		10 (Attic 2)&  CAF      &  Straight Stream  	&  120 gpm/60 cfm   &	Furniture Fuel Package  \\
		11 (FSE1)   &  Water    &  30 deg Fog  			&  120 gpm   		&	Wood Fuel Package, Panel Window	 \\
		12 (FSW1)   &  CAF      &  30 deg Fog  			&  120 gpm/60 cfm   &	Wood Fuel Package, Panel Window  \\
		13 (FSE2)   &  Water    &  30 deg Fog  			&  120 gpm    		&	Wood Fuel Package, Panel Window  \\
		14 (FSW2)   &  CAF      &  30 deg Fog  			&  120 gpm/60 cfm   &	Wood Fuel Package, Panel Window  \\
		\noalign{\smallskip}\hline
	\end{tabular}
	\label{tab:Test_Descriptions}
\end{sidewaystable}

\begin{sidewaystable}[!ht]
	\centering
	\caption{Nozzle Activation Times}
	\begin{tabular}{lllllll}
		\hline\noalign{\smallskip}
		Test		& Agent		& Stream				& Nozzle 1 Activation	& Nozzle 1 Shut Off		& Nozzle 2 Activation	& Nozzle 2 Shut Off \\
		\noalign{\smallskip}\hline\noalign{\smallskip}
		1 (FSE1)    &  CAF      &  30 deg Fog  	    	& 233 					& 248					& 269					& 284					\\
		2 (FSW2)   	& Water     &  30 deg Fog  	    	& 235					& 250					& 272					& 287					\\
		3 (FSW3) 	&  CAF      &  7/8" Smooth Bore     & 228					& 243 					& 255					& 270					\\
		4 (FSE4)    &  Water    &  7/8" Smooth Bore     & 233					& 248					& 268					& 283					\\
		5 (FSE5)    &  Water    &  30 deg Fog  	    	& 232					& 247					& 257					& 272					\\
		6 (FSW6)    &  CAF      &  30 deg Fog     		& 239					& 254					& 268					& 283					\\
		7 (FSW7)    &  Water    &  30 deg Fog     		& 276					& 291					& 308					& 323					\\
		8 (FSE8)    &  CAF      &  30 deg Fog     		& 245					& 260					& 275					& 290					\\
		9 (Attic 1) &  Water    &  Straight Stream     	&						&						&						&					\\
		10 (Attic 2)&  CAF      &  Straight Stream     	& 						&						&						&					\\
		11 (FSE1)   &  Water    &  30 deg Fog  	   		& 315					& 330					& 345					& 360					\\
		12 (FSW1)   &  CAF      &  30 deg Fog  	   		& 233					& 248					& 263					& 278					\\
		13 (FSE2)   &  Water    &  30 deg Fog  	   		& 330					& 390					& 406					& 466					\\
		14 (FSW2)   &  CAF      &  30 deg Fog  	  		& 435					& 495					& 510					& 570					\\
		\noalign{\smallskip}\hline
	\end{tabular}
	\label{tab:Nozzle_Activation_Times}
\end{sidewaystable}
	
\subsection{Wood Fuel Package}
\label{subsec:Wood_Fuel_Package}

\begin{figure}[!ht]
	\includegraphics[width=6in]{../Figures/Pictures/Metric/DelCoFogTest11FirstSuppression}
	\caption{Test 11: Wood Fuel Package, Panel Window, Water, 30 deg Fog, 120 gpm, First Suppression}
	\label{fig:Test_11_First_Suppression}
\end{figure}

\begin{figure}[!ht]
	\includegraphics[width=6in]{../Figures/Pictures/Metric/DelCoFogTest11SecondSuppression}
	\caption{Test 11: Wood Fuel Package, Panel Window, Water, 30 deg Fog, 120 gpm, Second Suppression}
	\label{fig:Test_11_Second_Suppression}
\end{figure}

\begin{figure}[!ht]
	\includegraphics[width=6in]{../Figures/Pictures/Metric/DelCoFogTest12FirstSuppression}
	\caption{Test 12: Wood Fuel Package, Panel Window, CAFS, 30 deg Fog, 120 gpm/60cfm, First Suppression}
	\label{fig:Test_12_First_Suppression}
\end{figure}

\begin{figure}[!ht]
	\includegraphics[width=6in]{../Figures/Pictures/Metric/DelCoFogTest12SecondSuppression}
	\caption{Test 12: Wood Fuel Package, Panel Window, CAFS, 30 deg Fog, 120 gpm/60cfm, Second Suppression}
	\label{fig:Test_12_Second_Suppression}
\end{figure}

\clearpage

\begin{figure}[!ht]
	\includegraphics[width=6in]{../Figures/Pictures/Metric/DelCoFogTest13FirstSuppression}
	\caption{Test 13: Wood Fuel Package, Panel Window, Water, 30 deg Fog, 120 gpm, First Suppression}
	\label{fig:Test_13_First_Suppression}
\end{figure}

\begin{figure}[!ht]
	\includegraphics[width=6in]{../Figures/Pictures/Metric/DelCoFogTest13SecondSuppression}
	\caption{Test 13: Wood Fuel Package, Panel Window, Water, 30 deg Fog, 120 gpm, Second Suppression}
	\label{fig:Test_13_Second_Suppression}
\end{figure}

\begin{figure}[!ht]
	\includegraphics[width=6in]{../Figures/Pictures/Metric/DelCoFogTest14FirstSuppression}
	\caption{Test 14: Wood Fuel Package, Panel Window, CAFS, 30 deg Fog, 120 gpm/60cfm, First Suppression}
	\label{fig:Test_14_First_Suppression}
\end{figure}

\begin{figure}[!ht]
	\includegraphics[width=6in]{../Figures/Pictures/Metric/DelCoFogTest14SecondSuppression}
	\caption{Test 14: Wood Fuel Package, Panel Window, CAFS, 30 deg Fog, 120 gpm/60cfm, Second Suppression}
	\label{fig:Test_14_Second_Suppression}
\end{figure}

\clearpage

\subsection{Furniture Fuel Package}
\label{subsec:Furniture_Fuel_Package}

\begin{figure}[!ht]
	\includegraphics[width=6in]{../Figures/Pictures/Metric/DelCoFogTest1FirstSuppression}
	\caption{Test 1: Furniture Fuel Package, CAFS, 30 deg Fog, 120 gpm/60cfm, First Suppression}
	\label{fig:Test_1_First_Suppression}
\end{figure}

\begin{figure}[!ht]
	\includegraphics[width=6in]{../Figures/Pictures/Metric/DelCoFogTest1SecondSuppression}
	\caption{Test 1: Furniture Fuel Package, CAFS, 30 deg Fog, 120 gpm/60cfm, Second Suppression}
	\label{fig:Test_1_Second_Suppression}
\end{figure}

\begin{figure}[!ht]
	\includegraphics[width=6in]{../Figures/Pictures/Metric/DelCoFogTest2FirstSuppression}
	\caption{Test 2: Furniture Fuel Package, Water, 30 deg Fog, 120 gpm, First Suppression}
	\label{fig:Test_2_First_Suppression}
\end{figure}

\begin{figure}[!ht]
	\includegraphics[width=6in]{../Figures/Pictures/Metric/DelCoFogTest2SecondSuppression}
	\caption{Test 2: Furniture Fuel Package, Water, 30 deg Fog, 120 gpm, Second Suppression}
	\label{fig:Test_2_Second_Suppression}
\end{figure}

\clearpage

\begin{figure}[!ht]
	\includegraphics[width=6in]{../Figures/Pictures/Metric/DelCoSSTest3FirstSuppression}
	\caption{Test 3: Furniture Fuel Package, CAFS, 7/8" Smooth Bore, 120 gpm/60 cfm, First Suppression}
	\label{fig:Test_3_First_Suppression}
\end{figure}

\begin{figure}[!ht]
	\includegraphics[width=6in]{../Figures/Pictures/Metric/DelCoSSTest3SecondSuppression}
	\caption{Test 3: Furniture Fuel Package, CAFS, 7/8" Smooth Bore, 120 gpm/60 cfm, Second Suppression}
	\label{fig:Test_3_Second_Suppression}
\end{figure}

\begin{figure}[!ht]
	\includegraphics[width=6in]{../Figures/Pictures/Metric/DelCoSSTest4FirstSuppression}
	\caption{Test 4: Furniture Fuel Package, Water, 7/8" Smooth Bore, 120 gpm, First Suppression}
	\label{fig:Test_4_First_Suppression}
\end{figure}

\begin{figure}[!ht]
	\includegraphics[width=6in]{../Figures/Pictures/Metric/DelCoSSTest4SecondSuppression}
	\caption{Test 4: Furniture Fuel Package, Water, 7/8" Smooth Bore, 120 gpm, Second Suppression}
	\label{fig:Test_4_Second_Suppression}
\end{figure}

\clearpage

\begin{figure}[!ht]
	\includegraphics[width=6in]{../Figures/Pictures/Metric/DelCoFogTest5FirstSuppression}
	\caption{Test 5: Furniture Fuel Package, Water, 30 deg Fog, 120 gpm, First Suppression}
	\label{fig:Test_5_First_Suppression}
\end{figure}

\begin{figure}[!ht]
	\includegraphics[width=6in]{../Figures/Pictures/Metric/DelCoFogTest5SecondSuppression}
	\caption{Test 5: Furniture Fuel Package, Water, 30 deg Fog, 120 gpm, Second Suppression}
	\label{fig:Test_5_Second_Suppression}
\end{figure}

\begin{figure}[!ht]
	\includegraphics[width=6in]{../Figures/Pictures/Metric/DelCoFogTest6FirstSuppression}
	\caption{Test 6: Furniture Fuel Package, CAFS, 30 deg Fog, 120 gpm/60cfm, First Suppression}
	\label{fig:Test_6_First_Suppression}
\end{figure}

\begin{figure}[!ht]
	\includegraphics[width=6in]{../Figures/Pictures/Metric/DelCoFogTest6SecondSuppression}
	\caption{Test 6: Furniture Fuel Package, CAFS, 30 deg Fog, 120 gpm/60cfm, Second Suppression}
	\label{fig:Test_6_Second_Suppression}
\end{figure}

\clearpage

\begin{figure}[!ht]
	\includegraphics[width=6in]{../Figures/Pictures/Metric/DelCoFogTest7FirstSuppression}
	\caption{Test 7: Furniture Fuel Package, Water, 30 deg Fog, 120 gpm, First Suppression}
	\label{fig:Test_7_First_Suppression}
\end{figure}

\begin{figure}[!ht]
	\includegraphics[width=6in]{../Figures/Pictures/Metric/DelCoFogTest7SecondSuppression}
	\caption{Test 7: Furniture Fuel Package, Water, 30 deg Fog, 120 gpm, Second Suppression}
	\label{fig:Test_7_Second_Suppression}
\end{figure}

\begin{figure}[!ht]
	\includegraphics[width=6in]{../Figures/Pictures/Metric/DelCoFogTest8FirstSuppression}
	\caption{Test 8: Furniture Fuel Package, CAFS, 30 deg Fog, 120 gpm/60cfm, First Suppression}
	\label{fig:Test_8_First_Suppression}
\end{figure}

\begin{figure}[!ht]
	\includegraphics[width=6in]{../Figures/Pictures/Metric/DelCoFogTest8SecondSuppression}
	\caption{Test 8: Furniture Fuel Package, CAFS, 30 deg Fog, 120 gpm/60cfm, Second Suppression}
	\label{fig:Test_8_Second_Suppression}
\end{figure}

\clearpage

\subsection{Spray Density}
\label{subsec:Spray_Density}
	
\chapter{Discussion and Tactical Considerations}
\label{chap:Discussion_and_Tactical_Considerations}

\chapter{Summary}
\label{chap:Summary}

The National Institute of Standards and Technology, with the support of the Fire Protection Research
Foundation, the U.S. Department of Homeland Security, and the U.S. Fire Administration conducted a
series of fire experiments to examine the impact of wind on fire spread through a multi-room structure
and examine the capabilities of wind-control devices (WCD) and externally applied water to mitigate the
hazard. The measurements used to examine the impact of the WCDs and the external water application
tactics were heat release rate, temperature, heat flux, and gas velocity inside the structure. Oxygen,
carbon dioxide, carbon monoxide, total hydrocarbons and differential pressures were also measured.
Each of the experiments was recorded with video and thermal imaging cameras. Some of these
measurements are not practical or affordable to make in an acquired structure, hence the need to build a
structure and conduct the experiments within the confines of the NIST Large Fire Facility. These 354
experiments also provided visual documentation of fire phenomena that are not typically observable on
the fire ground.
A limited series of heat release rate experiments were conducted to characterize the fuel load packages
used in wind driven structure experiments. Both the bedroom and the living room contained a fuel load
composed of furnishings with an average peak heat release rate of 7.8 MW with a total heat release of at
least 1700 MJ, not accounting for any of the wooden furniture or interior finish materials.
The experiments were designed to expose a public corridor area to a wind driven, post-flashover
apartment fire. The door from the apartment to the corridor was open for each of the experiments. The
conditions in the corridor were of critical importance because that is the portion of the building that
firefighters would use to approach the fire apartment or that occupants from an adjoining apartment
would use to exit the building.
The fires were ignited in the bedroom of the apartment. Prior to the failure or venting of the bedroom
window, which was on the upwind side of the experimental apartment, the heat release rate from the fire
was on the order of 1 MW. Prior to implementing either of the mitigating tactics, the heat release rates
from the post-flashover structure fire were typically between 15 MW and 20 MW. When the door from
the apartment to the corridor was open, temperatures in the corridor area near the open doorway, 1.52 m
(5.00 ft) below the ceiling, were in excess of 600 °C (1112 °F) for each of the experiments. The heat
fluxes measured in the same location, during the same experiments, were in excess of 70 kW/m². These
extreme thermal conditions are not tenable, even for a firefighter in full protective gear. These
conditions were attained within 30 s of the window failure.
Experiment 1 was conducted without any external wind. This experiment provided valuable baseline
data and demonstrated several important points relevant to fire fighting:
Smoke is Fuel. A ventilation limited (fuel rich) condition had developed prior to the failure of
the window. Oxygen depleted combustion products, containing carbon dioxide, carbon monoxide and
unburned hydrocarbons filled the rooms of the structure. Once the window failed, the fresh air provided
the oxygen needed to sustain the transition through flashover, which caused a significant increase in heat
release rate.
Venting does not always equal cooling. In this experiment, post ventilation temperatures and
heat fluxes all increased, due to the ventilation induced flashover.
Fire induced flows. Velocities within the structure exceeded 5 m/s (11 mph), just due to the fire
growth and the flow path that was set-up between the window opening and the corridor vent.
Avoid the flow path. The directional nature of the fire gas flow was demonstrated with thermal
conditions, both temperature and heat flux, which were twice as high in the “flow” portion of the
corridor as opposed to the “static” portion of the corridor in Experiment 1. Thermal conditions in the
flow path were not consistent with firefighter survival.
Experiments 2 through 8 all used a mechanically generated wind, ranging from 3 m/s to 9 m/s (7 mph to
20 mph). 355
The fuel load in the structure was the same for all of the experiments. Each of these experiments
demonstrated a rapid transition to untenable conditions in the corridor, even for a firefighter in full PPE,
after the window failed.
Experiments 2 through 5 focused on the impact of WCDS. In these experiments, the WCDs reduced
the temperatures in the corridor outside the doorway by more than 50 % within 60 s of deployment. The
heat fluxes were reduced by at least 70 % during this same time period. The WCDs also completely
mitigated any gas velocity due to the external wind.
Experiments 6 through 8 focused on the impact of externally applied water. In these experiments, the
externally applied water streams were implemented in three different ways; a fog stream across the face
of the window opening, a fog stream into the window opening, and a solid water stream into the window
opening. The fog stream across the window was not effective at reducing the thermal conditions in the
corridor. The fog stream in the window decreased the corridor temperature by at least 20 % and the
corresponding heat flux measured by at least 30 %. The solid stream experiments resulted in corridor
temperature and heat flux reductions of at least 40 % within 60 s of application. None of the water
applications reduced the gas velocities in the structure. In some cases, the gas velocity increased during
water application, due to momentum imparted from the water.
These experiments demonstrated the “extreme” thermal conditions that can be generated by a “simple
room and contents” fire and how these conditions can be extended along a flow path within a structure
when wind and an open vent are present. Two potential tactics which could be implemented from either
the floor above the fire in the case of a WCD, or from the floor below the fire in the case of the external
water application were demonstrated to be effective in reducing the thermal hazard in the corridor.
However, these experimental results also indicate that the post deployment thermal conditions for any
single tactic were still of a level which could pose a hazard to firefighters in full PPE.
The experiments also provided potential guidance for firefighters as a part of a fire size up and approach
to the room of fire origin: note wind conditions in the area of the fire, look for “pulsing flames”,
examine smoke conditions around closed doors in the potential flow path, and maintain control of doors
in the flow path.
Further research in actual buildings is required to fully understand the ability of firefighters to
implement these tactics, to examine the thermal conditions throughout the structure such as in stairways,
and to examine the interaction of these tactics with building ventilation strategies both natural and with
positive pressure ventilation.
If the demonstrated technologies continue to prove effective in the field trials and pilot programs, the
next step may be to examine the need for standards and standardized test methods to define a minimum
level of acceptable performance of these devices.

\chapter{Appendix}
\label{chap:Appendix}

\section{Test 1 Figures}
\label{subsec:Test_1_Figures}

\subsection{Temperature}
\label{subsec:Temperature}

\begin{figure}[!ht]
	\includegraphics[width=6in]{../Figures/Temperature/FSE1_Eastside_Array}
	\caption{Test 1 (FSE1): Eastside Array}
	\label{fig:Test_1_Eastside_Array}
\end{figure}

\begin{figure}[!ht]
	\includegraphics[width=6in]{../Figures/Temperature/FSE1_Westside_Array}
	\caption{Test 1 (FSE1): Westside Array}
	\label{fig:Test_1_Westside_Array}
\end{figure}

\begin{figure}[!ht]
	\includegraphics[width=6in]{../Figures/Temperature/FSE1_Hallway_Array}
	\caption{Test 1 (FSE1): Hallway Array}
	\label{fig:Test_1_Hallway_Array}
\end{figure}

\begin{figure}[!ht]
	\includegraphics[width=6in]{../Figures/Temperature/FSE1_Doorway_Array}
	\caption{Test 1 (FSE1): Doorway Array}
	\label{fig:Test_1_Doorway_Array}
\end{figure}

\begin{figure}[!ht]
	\includegraphics[width=6in]{../Figures/Temperature/Suppression_FSE1_Eastside_Array}
	\caption{Test 1 (FSE1): Suppression Eastside Array}
	\label{fig:Test_1_Suppression_Eastside_Array}
\end{figure}

\begin{figure}[!ht]
	\includegraphics[width=6in]{../Figures/Temperature/Suppression_FSE1_Westside_Array}
	\caption{Test 1 (FSE1): Suppression Westside Array}
	\label{fig:Test_1_Suppression_Westside_Array}
\end{figure}

\begin{figure}[!ht]
	\includegraphics[width=6in]{../Figures/Temperature/Suppression_FSE1_Hallway_Array}
	\caption{Test 1 (FSE1): Suppression Hallway Array}
	\label{fig:Test_1_Suppression_Hallway_Array}
\end{figure}

\begin{figure}[!ht]
	\includegraphics[width=6in]{../Figures/Temperature/Suppression_FSE1_Doorway_Array}
	\caption{Test 1 (FSE1): Suppression Doorway Array}
	\label{fig:Test_1_Suppression_Doorway_Array}
\end{figure}

\subsection{Heat Flux}
\label{subsec:Heat_Flux}

\begin{figure}[!ht]
	\includegraphics[width=6in]{../Figures/Heat_Flux/FSE_Test_1_Heat_Flux_Eastside}
	\caption{Test 1 (FSE1): Eastside Heat Flux}
	\label{fig:Test_1_Eastside_Heat_Flux}
\end{figure}

\begin{figure}[!ht]
	\includegraphics[width=6in]{../Figures/Heat_Flux/FSE_Test_1_Heat_Flux_Westside}
	\caption{Test 1 (FSE1): Westside Heat Flux}
	\label{fig:Test_1_Westside_Heat_Flux}
\end{figure}

\begin{figure}[!ht]
	\includegraphics[width=6in]{../Figures/Heat_Flux/FSE_Test_1_Heat_Flux_Hallway}
	\caption{Test 1 (FSE1): Hallway Heat Flux}
	\label{fig:Test_1_Hallway_Heat_Flux}
\end{figure}

\begin{figure}[!ht]
	\includegraphics[width=6in]{../Figures/Heat_Flux/FSE_Test_1_Heat_Flux_Near_Fire_Room}
	\caption{Test 1 (FSE1): Heat Flux Near Fire Room}
	\label{fig:Test_1_Heat_Flux_Near_Fire_Room}
\end{figure}

\subsection{Velocity}
\label{subsec:Velocity}

\begin{figure}[!ht]
	\includegraphics[width=6in]{../Figures/Velocity/FSE_Test_1_Hallway_Velocity}
	\caption{Test 1 (FSE1): Hallway Velocity}
	\label{fig:Test_1_Hallway_Velocity}
\end{figure}

\begin{figure}[!ht]
	\includegraphics[width=6in]{../Figures/Velocity/FSE_Test_1_Doorway_Velocity}
	\caption{Test 1 (FSE1): Doorway Velocity}
	\label{fig:Test_1_Doorway_Velocity}
\end{figure}

\clearpage

\section{Test 2 Figures}
\label{subsec:Test_2_Figures}

\subsection{Temperature}
\label{subsec:Temperature}

\begin{figure}[!ht]
	\includegraphics[width=6in]{../Figures/Temperature/FSW2_Eastside_Array}
	\caption{Test 2 (FSW2): Eastside Array}
	\label{fig:Test_2_Eastside_Array}
\end{figure}

\begin{figure}[!ht]
	\includegraphics[width=6in]{../Figures/Temperature/FSW2_Westside_Array}
	\caption{Test 2 (FSW2: Westside Array}
	\label{fig:Test_2_Westside_Array}
\end{figure}

\begin{figure}[!ht]
	\includegraphics[width=6in]{../Figures/Temperature/FSW2_Hallway_Array}
	\caption{Test 2 (FSW2): Hallway Array}
	\label{fig:Test_2_Hallway_Array}
\end{figure}

\begin{figure}[!ht]
	\includegraphics[width=6in]{../Figures/Temperature/FSW2_Doorway_Array}
	\caption{Test 2 (FSW2): Doorway Array}
	\label{fig:Test_2_Doorway_Array}
\end{figure}

\begin{figure}[!ht]
	\includegraphics[width=6in]{../Figures/Temperature/Suppression_FSW2_Eastside_Array}
	\caption{Test 2 (FSW2): Suppression Eastside Array}
	\label{fig:Test_2_Suppression_Eastside_Array}
\end{figure}

\begin{figure}[!ht]
	\includegraphics[width=6in]{../Figures/Temperature/Suppression_FSW2_Westside_Array}
	\caption{Test 2 (FSW2): Suppression Westside Array}
	\label{fig:Test_2_Suppression_Westside_Array}
\end{figure}

\begin{figure}[!ht]
	\includegraphics[width=6in]{../Figures/Temperature/Suppression_FSW2_Hallway_Array}
	\caption{Test 2 (FSW2): Suppression Hallway Array}
	\label{fig:Test_2_Suppression_Hallway_Array}
\end{figure}

\begin{figure}[!ht]
	\includegraphics[width=6in]{../Figures/Temperature/Suppression_FSW2_Doorway_Array}
	\caption{Test 2 (FSW2): Suppression Doorway Array}
	\label{fig:Test_2_Suppression_Doorway_Array}
\end{figure}

\subsection{Heat Flux}
\label{subsec:Heat_Flux}

\begin{figure}[!ht]
	\includegraphics[width=6in]{../Figures/Heat_Flux/FSW_Test_2_Heat_Flux_Eastside}
	\caption{Test 2 (FSW2): Eastside Heat Flux}
	\label{fig:Test_2_Eastside_Heat_Flux}
\end{figure}

\begin{figure}[!ht]
	\includegraphics[width=6in]{../Figures/Heat_Flux/FSW_Test_2_Heat_Flux_Westside}
	\caption{Test 2 (FSW2): Westside Heat Flux}
	\label{fig:Test_2_Westside_Heat_Flux}
\end{figure}

\begin{figure}[!ht]
	\includegraphics[width=6in]{../Figures/Heat_Flux/FSW_Test_2_Heat_Flux_Hallway}
	\caption{Test 2 (FSW2): Hallway Heat Flux}
	\label{fig:Test_2_Hallway_Heat_Flux}
\end{figure}

\begin{figure}[!ht]
	\includegraphics[width=6in]{../Figures/Heat_Flux/FSW_Test_2_Heat_Flux_Near_Fire_Room}
	\caption{Test 2 (FSW2): Heat Flux Near Fire Room}
	\label{fig:Test_2_Heat_Flux_Near_Fire_Room}
\end{figure}

\subsection{Velocity}
\label{subsec:Velocity}

\begin{figure}[!ht]
	\includegraphics[width=6in]{../Figures/Velocity/FSW_Test_2_Hallway_Velocity}
	\caption{Test 2 (FSW2): Hallway Velocity}
	\label{fig:Test_2_Hallway_Velocity}
\end{figure}

\begin{figure}[!ht]
	\includegraphics[width=6in]{../Figures/Velocity/FSW_Test_2_Doorway_Velocity}
	\caption{Test 2 (FSW2): Doorway Velocity}
	\label{fig:Test_2_Doorway_Velocity}
\end{figure}

\clearpage

\section{Test 3 Figures}
\label{subsec:Test_3_Figures}

\subsection{Temperature}
\label{subsec:Temperature}

\begin{figure}[!ht]
	\includegraphics[width=6in]{../Figures/Temperature/FSW3_Eastside_Array}
	\caption{Test 3 (FSW3): Eastside Array}
	\label{fig:Test_3_Eastside_Array}
\end{figure}

\begin{figure}[!ht]
	\includegraphics[width=6in]{../Figures/Temperature/FSW3_Westside_Array}
	\caption{Test 3 (FSW3: Westside Array}
	\label{fig:Test_3_Westside_Array}
\end{figure}

\begin{figure}[!ht]
	\includegraphics[width=6in]{../Figures/Temperature/FSW3_Hallway_Array}
	\caption{Test 3 (FSW3): Hallway Array}
	\label{fig:Test_3_Hallway_Array}
\end{figure}

\begin{figure}[!ht]
	\includegraphics[width=6in]{../Figures/Temperature/FSW3_Doorway_Array}
	\caption{Test 3 (FSW3): Doorway Array}
	\label{fig:Test_3_Doorway_Array}
\end{figure}

\begin{figure}[!ht]
	\includegraphics[width=6in]{../Figures/Temperature/Suppression_FSW3_Eastside_Array}
	\caption{Test 3 (FSW3): Suppression Eastside Array}
	\label{fig:Test_3_Suppression_Eastside_Array}
\end{figure}

\begin{figure}[!ht]
	\includegraphics[width=6in]{../Figures/Temperature/Suppression_FSW3_Westside_Array}
	\caption{Test 3 (FSW3): Suppression Westside Array}
	\label{fig:Test_3_Suppression_Westside_Array}
\end{figure}

\begin{figure}[!ht]
	\includegraphics[width=6in]{../Figures/Temperature/Suppression_FSW3_Hallway_Array}
	\caption{Test 3 (FSW3): Suppression Hallway Array}
	\label{fig:Test_3_Suppression_Hallway_Array}
\end{figure}

\begin{figure}[!ht]
	\includegraphics[width=6in]{../Figures/Temperature/Suppression_FSW3_Doorway_Array}
	\caption{Test 3 (FSW3): Suppression Doorway Array}
	\label{fig:Test_3_Suppression_Doorway_Array}
\end{figure}

\subsection{Heat Flux}
\label{subsec:Heat_Flux}

\begin{figure}[!ht]
	\includegraphics[width=6in]{../Figures/Heat_Flux/FSW_Test_3_Heat_Flux_Eastside}
	\caption{Test 3 (FSW3): Eastside Heat Flux}
	\label{fig:Test_3_Eastside_Heat_Flux}
\end{figure}

\begin{figure}[!ht]
	\includegraphics[width=6in]{../Figures/Heat_Flux/FSW_Test_3_Heat_Flux_Westside}
	\caption{Test 3 (FSW3): Westside Heat Flux}
	\label{fig:Test_3_Westside_Heat_Flux}
\end{figure}

\begin{figure}[!ht]
	\includegraphics[width=6in]{../Figures/Heat_Flux/FSW_Test_3_Heat_Flux_Hallway}
	\caption{Test 3 (FSW3): Hallway Heat Flux}
	\label{fig:Test_3_Hallway_Heat_Flux}
\end{figure}

\begin{figure}[!ht]
	\includegraphics[width=6in]{../Figures/Heat_Flux/FSW_Test_3_Heat_Flux_Near_Fire_Room}
	\caption{Test 3 (FSW3): Heat Flux Near Fire Room}
	\label{fig:Test_3_Heat_Flux_Near_Fire_Room}
\end{figure}

\subsection{Velocity}
\label{subsec:Velocity}

\begin{figure}[!ht]
	\includegraphics[width=6in]{../Figures/Velocity/FSW_Test_3_Hallway_Velocity}
	\caption{Test 3 (FSW3): Hallway Velocity}
	\label{fig:Test_3_Hallway_Velocity}
\end{figure}

\begin{figure}[!ht]
	\includegraphics[width=6in]{../Figures/Velocity/FSW_Test_3_Doorway_Velocity}
	\caption{Test 3 (FSW3): Doorway Velocity}
	\label{fig:Test_3_Doorway_Velocity}
\end{figure}

\clearpage

\section{Test 4 Figures}
\label{subsec:Test_4_Figures}

\subsection{Temperature}
\label{subsec:Temperature}

\begin{figure}[!ht]
	\includegraphics[width=6in]{../Figures/Temperature/FSE4_Eastside_Array}
	\caption{Test 4 (FSE4): Eastside Array}
	\label{fig:Test_4_Eastside_Array}
\end{figure}

\begin{figure}[!ht]
	\includegraphics[width=6in]{../Figures/Temperature/FSE4_Westside_Array}
	\caption{Test 4 (FSE4): Westside Array}
	\label{fig:Test_4_Westside_Array}
\end{figure}

\begin{figure}[!ht]
	\includegraphics[width=6in]{../Figures/Temperature/FSE4_Hallway_Array}
	\caption{Test 4 (FSE4): Hallway Array}
	\label{fig:Test_4_Hallway_Array}
\end{figure}

\begin{figure}[!ht]
	\includegraphics[width=6in]{../Figures/Temperature/FSE4_Doorway_Array}
	\caption{Test 4 (FSE4): Doorway Array}
	\label{fig:Test_4_Doorway_Array}
\end{figure}

\begin{figure}[!ht]
	\includegraphics[width=6in]{../Figures/Temperature/Suppression_FSE4_Eastside_Array}
	\caption{Test 4 (FSE4): Suppression Eastside Array}
	\label{fig:Test_4_Suppression_Eastside_Array}
\end{figure}

\begin{figure}[!ht]
	\includegraphics[width=6in]{../Figures/Temperature/Suppression_FSE4_Westside_Array}
	\caption{Test 4 (FSE4): Suppression Westside Array}
	\label{fig:Test_4_Suppression_Westside_Array}
\end{figure}

\begin{figure}[!ht]
	\includegraphics[width=6in]{../Figures/Temperature/Suppression_FSE4_Hallway_Array}
	\caption{Test 4 (FSE4): Suppression Hallway Array}
	\label{fig:Test_4_Suppression_Hallway_Array}
\end{figure}

\begin{figure}[!ht]
	\includegraphics[width=6in]{../Figures/Temperature/Suppression_FSE4_Doorway_Array}
	\caption{Test 4 (FSE4): Suppression Doorway Array}
	\label{fig:Test_4_Suppression_Doorway_Array}
\end{figure}

\subsection{Heat Flux}
\label{subsec:Heat_Flux}

\begin{figure}[!ht]
	\includegraphics[width=6in]{../Figures/Heat_Flux/FSE_Test_4_Heat_Flux_Eastside}
	\caption{Test 4 (FSE4): Eastside Heat Flux}
	\label{fig:Test_4_Eastside_Heat_Flux}
\end{figure}

\begin{figure}[!ht]
	\includegraphics[width=6in]{../Figures/Heat_Flux/FSE_Test_4_Heat_Flux_Westside}
	\caption{Test 4 (FSE4): Westside Heat Flux}
	\label{fig:Test_4_Westside_Heat_Flux}
\end{figure}

\begin{figure}[!ht]
	\includegraphics[width=6in]{../Figures/Heat_Flux/FSE_Test_4_Heat_Flux_Hallway}
	\caption{Test 4 (FSE4): Hallway Heat Flux}
	\label{fig:Test_4_Hallway_Heat_Flux}
\end{figure}

\begin{figure}[!ht]
	\includegraphics[width=6in]{../Figures/Heat_Flux/FSE_Test_4_Heat_Flux_Near_Fire_Room}
	\caption{Test 4 (FSE4): Heat Flux Near Fire Room}
	\label{fig:Test_4_Heat_Flux_Near_Fire_Room}
\end{figure}

\subsection{Velocity}
\label{subsec:Velocity}

\begin{figure}[!ht]
	\includegraphics[width=6in]{../Figures/Velocity/FSE_Test_4_Hallway_Velocity}
	\caption{Test 4 (FSE4): Hallway Velocity}
	\label{fig:Test_4_Hallway_Velocity}
\end{figure}

\begin{figure}[!ht]
	\includegraphics[width=6in]{../Figures/Velocity/FSE_Test_4_Doorway_Velocity}
	\caption{Test 4 (FSE4): Doorway Velocity}
	\label{fig:Test_4_Doorway_Velocity}
\end{figure}

\clearpage

\section{Test 5 Figures}
\label{subsec:Test_5_Figures}

\subsection{Temperature}
\label{subsec:Temperature}

\begin{figure}[!ht]
	\includegraphics[width=6in]{../Figures/Temperature/FSE5_Eastside_Array}
	\caption{Test 5 (FSE5): Eastside Array}
	\label{fig:Test_5_Eastside_Array}
\end{figure}

\begin{figure}[!ht]
	\includegraphics[width=6in]{../Figures/Temperature/FSE5_Westside_Array}
	\caption{Test 5 (FSE5): Westside Array}
	\label{fig:Test_5_Westside_Array}
\end{figure}

\begin{figure}[!ht]
	\includegraphics[width=6in]{../Figures/Temperature/FSE5_Hallway_Array}
	\caption{Test 5 (FSE5): Hallway Array}
	\label{fig:Test_5_Hallway_Array}
\end{figure}

\begin{figure}[!ht]
	\includegraphics[width=6in]{../Figures/Temperature/FSE5_Doorway_Array}
	\caption{Test 5 (FSE5): Doorway Array}
	\label{fig:Test_5_Doorway_Array}
\end{figure}

\begin{figure}[!ht]
	\includegraphics[width=6in]{../Figures/Temperature/Suppression_FSE5_Eastside_Array}
	\caption{Test 5 (FSE5): Suppression Eastside Array}
	\label{fig:Test_5_Suppression_Eastside_Array}
\end{figure}

\begin{figure}[!ht]
	\includegraphics[width=6in]{../Figures/Temperature/Suppression_FSE5_Westside_Array}
	\caption{Test 5 (FSE5): Suppression Westside Array}
	\label{fig:Test_5_Suppression_Westside_Array}
\end{figure}

\begin{figure}[!ht]
	\includegraphics[width=6in]{../Figures/Temperature/Suppression_FSE5_Hallway_Array}
	\caption{Test 5 (FSE5): Suppression Hallway Array}
	\label{fig:Test_5_Suppression_Hallway_Array}
\end{figure}

\begin{figure}[!ht]
	\includegraphics[width=6in]{../Figures/Temperature/Suppression_FSE5_Doorway_Array}
	\caption{Test 5 (FSE5): Suppression Doorway Array}
	\label{fig:Test_5_Suppression_Doorway_Array}
\end{figure}

\subsection{Heat Flux}
\label{subsec:Heat_Flux}

\begin{figure}[!ht]
	\includegraphics[width=6in]{../Figures/Heat_Flux/FSE_Test_5_Heat_Flux_Eastside}
	\caption{Test 5 (FSE5): Eastside Heat Flux}
	\label{fig:Test_5_Eastside_Heat_Flux}
\end{figure}

\begin{figure}[!ht]
	\includegraphics[width=6in]{../Figures/Heat_Flux/FSE_Test_5_Heat_Flux_Westside}
	\caption{Test 5 (FSE5): Westside Heat Flux}
	\label{fig:Test_5_Westside_Heat_Flux}
\end{figure}

\begin{figure}[!ht]
	\includegraphics[width=6in]{../Figures/Heat_Flux/FSE_Test_5_Heat_Flux_Hallway}
	\caption{Test 5 (FSE5): Hallway Heat Flux}
	\label{fig:Test_5_Hallway_Heat_Flux}
\end{figure}

\begin{figure}[!ht]
	\includegraphics[width=6in]{../Figures/Heat_Flux/FSE_Test_5_Heat_Flux_Near_Fire_Room}
	\caption{Test 5 (FSE5): Heat Flux Near Fire Room}
	\label{fig:Test_5_Heat_Flux_Near_Fire_Room}
\end{figure}

\subsection{Velocity}
\label{subsec:Velocity}

\begin{figure}[!ht]
	\includegraphics[width=6in]{../Figures/Velocity/FSE_Test_5_Hallway_Velocity}
	\caption{Test 5 (FSE5): Hallway Velocity}
	\label{fig:Test_5_Hallway_Velocity}
\end{figure}

\begin{figure}[!ht]
	\includegraphics[width=6in]{../Figures/Velocity/FSE_Test_5_Doorway_Velocity}
	\caption{Test 5 (FSE5): Doorway Velocity}
	\label{fig:Test_5_Doorway_Velocity}
\end{figure}

\clearpage

\section{Test 6 Figures}
\label{subsec:Test_6_Figures}

\subsection{Temperature}
\label{subsec:Temperature}

\begin{figure}[!ht]
	\includegraphics[width=6in]{../Figures/Temperature/FSW6_Eastside_Array}
	\caption{Test 6 (FSW6): Eastside Array}
	\label{fig:Test_6_Eastside_Array}
\end{figure}

\begin{figure}[!ht]
	\includegraphics[width=6in]{../Figures/Temperature/FSW6_Westside_Array}
	\caption{Test 6 (FSW6: Westside Array}
	\label{fig:Test_6_Westside_Array}
\end{figure}

\begin{figure}[!ht]
	\includegraphics[width=6in]{../Figures/Temperature/FSW6_Hallway_Array}
	\caption{Test 6 (FSW6): Hallway Array}
	\label{fig:Test_6_Hallway_Array}
\end{figure}

\begin{figure}[!ht]
	\includegraphics[width=6in]{../Figures/Temperature/FSW6_Doorway_Array}
	\caption{Test 6 (FSW6): Doorway Array}
	\label{fig:Test_6_Doorway_Array}
\end{figure}

\begin{figure}[!ht]
	\includegraphics[width=6in]{../Figures/Temperature/Suppression_FSW6_Eastside_Array}
	\caption{Test 6 (FSW6): Suppression Eastside Array}
	\label{fig:Test_6_Suppression_Eastside_Array}
\end{figure}

\begin{figure}[!ht]
	\includegraphics[width=6in]{../Figures/Temperature/Suppression_FSW6_Westside_Array}
	\caption{Test 6 (FSW6): Suppression Westside Array}
	\label{fig:Test_6_Suppression_Westside_Array}
\end{figure}

\begin{figure}[!ht]
	\includegraphics[width=6in]{../Figures/Temperature/Suppression_FSW6_Hallway_Array}
	\caption{Test 6 (FSW6): Suppression Hallway Array}
	\label{fig:Test_6_Suppression_Hallway_Array}
\end{figure}

\begin{figure}[!ht]
	\includegraphics[width=6in]{../Figures/Temperature/Suppression_FSW6_Doorway_Array}
	\caption{Test 6 (FSW6): Suppression Doorway Array}
	\label{fig:Test_6_Suppression_Doorway_Array}
\end{figure}

\subsection{Heat Flux}
\label{subsec:Heat_Flux}

\begin{figure}[!ht]
	\includegraphics[width=6in]{../Figures/Heat_Flux/FSW_Test_6_Heat_Flux_Eastside}
	\caption{Test 6 (FSW6): Eastside Heat Flux}
	\label{fig:Test_6_Eastside_Heat_Flux}
\end{figure}

\begin{figure}[!ht]
	\includegraphics[width=6in]{../Figures/Heat_Flux/FSW_Test_6_Heat_Flux_Westside}
	\caption{Test 6 (FSW6): Westside Heat Flux}
	\label{fig:Test_6_Westside_Heat_Flux}
\end{figure}

\begin{figure}[!ht]
	\includegraphics[width=6in]{../Figures/Heat_Flux/FSW_Test_6_Heat_Flux_Hallway}
	\caption{Test 6 (FSW6): Hallway Heat Flux}
	\label{fig:Test_6_Hallway_Heat_Flux}
\end{figure}

\begin{figure}[!ht]
	\includegraphics[width=6in]{../Figures/Heat_Flux/FSW_Test_6_Heat_Flux_Near_Fire_Room}
	\caption{Test 6 (FSW6): Heat Flux Near Fire Room}
	\label{fig:Test_6_Heat_Flux_Near_Fire_Room}
\end{figure}

\subsection{Velocity}
\label{subsec:Velocity}

\begin{figure}[!ht]
	\includegraphics[width=6in]{../Figures/Velocity/FSW_Test_6_Hallway_Velocity}
	\caption{Test 6 (FSW6): Hallway Velocity}
	\label{fig:Test_6_Hallway_Velocity}
\end{figure}

\begin{figure}[!ht]
	\includegraphics[width=6in]{../Figures/Velocity/FSW_Test_6_Doorway_Velocity}
	\caption{Test 6 (FSW6): Doorway Velocity}
	\label{fig:Test_6_Doorway_Velocity}
\end{figure}

\clearpage

\section{Test 7 Figures}
\label{subsec:Test_7_Figures}

\subsection{Temperature}
\label{subsec:Temperature}

\begin{figure}[!ht]
	\includegraphics[width=6in]{../Figures/Temperature/FSW7_Eastside_Array}
	\caption{Test 7 (FSW7): Eastside Array}
	\label{fig:Test_7_Eastside_Array}
\end{figure}

\begin{figure}[!ht]
	\includegraphics[width=6in]{../Figures/Temperature/FSW7_Westside_Array}
	\caption{Test 7 (FSW7: Westside Array}
	\label{fig:Test_7_Westside_Array}
\end{figure}

\begin{figure}[!ht]
	\includegraphics[width=6in]{../Figures/Temperature/FSW7_Hallway_Array}
	\caption{Test 7 (FSW7): Hallway Array}
	\label{fig:Test_7_Hallway_Array}
\end{figure}

\begin{figure}[!ht]
	\includegraphics[width=6in]{../Figures/Temperature/FSW7_Doorway_Array}
	\caption{Test 7 (FSW7): Doorway Array}
	\label{fig:Test_7_Doorway_Array}
\end{figure}

\begin{figure}[!ht]
	\includegraphics[width=6in]{../Figures/Temperature/Suppression_FSW7_Eastside_Array}
	\caption{Test 7 (FSW7): Suppression Eastside Array}
	\label{fig:Test_7_Suppression_Eastside_Array}
\end{figure}

\begin{figure}[!ht]
	\includegraphics[width=6in]{../Figures/Temperature/Suppression_FSW7_Westside_Array}
	\caption{Test 7 (FSW7): Suppression Westside Array}
	\label{fig:Test_7_Suppression_Westside_Array}
\end{figure}

\begin{figure}[!ht]
	\includegraphics[width=6in]{../Figures/Temperature/Suppression_FSW7_Hallway_Array}
	\caption{Test 7 (FSW7): Suppression Hallway Array}
	\label{fig:Test_7_Suppression_Hallway_Array}
\end{figure}

\begin{figure}[!ht]
	\includegraphics[width=6in]{../Figures/Temperature/Suppression_FSW7_Doorway_Array}
	\caption{Test 7 (FSW7): Suppression Doorway Array}
	\label{fig:Test_7_Suppression_Doorway_Array}
\end{figure}

\subsection{Heat Flux}
\label{subsec:Heat_Flux}

\begin{figure}[!ht]
	\includegraphics[width=6in]{../Figures/Heat_Flux/FSW_Test_7_Heat_Flux_Eastside}
	\caption{Test 7 (FSW7): Eastside Heat Flux}
	\label{fig:Test_7_Eastside_Heat_Flux}
\end{figure}

\begin{figure}[!ht]
	\includegraphics[width=6in]{../Figures/Heat_Flux/FSW_Test_7_Heat_Flux_Westside}
	\caption{Test 7 (FSW7): Westside Heat Flux}
	\label{fig:Test_7_Westside_Heat_Flux}
\end{figure}

\begin{figure}[!ht]
	\includegraphics[width=6in]{../Figures/Heat_Flux/FSW_Test_7_Heat_Flux_Hallway}
	\caption{Test 7 (FSW7): Hallway Heat Flux}
	\label{fig:Test_7_Hallway_Heat_Flux}
\end{figure}

\begin{figure}[!ht]
	\includegraphics[width=6in]{../Figures/Heat_Flux/FSW_Test_7_Heat_Flux_Near_Fire_Room}
	\caption{Test 7 (FSW7): Heat Flux Near Fire Room}
	\label{fig:Test_7_Heat_Flux_Near_Fire_Room}
\end{figure}

\subsection{Velocity}
\label{subsec:Velocity}

\begin{figure}[!ht]
	\includegraphics[width=6in]{../Figures/Velocity/FSW_Test_7_Hallway_Velocity}
	\caption{Test 7 (FSW7): Hallway Velocity}
	\label{fig:Test_7_Hallway_Velocity}
\end{figure}

\begin{figure}[!ht]
	\includegraphics[width=6in]{../Figures/Velocity/FSW_Test_7_Doorway_Velocity}
	\caption{Test 7 (FSW7): Doorway Velocity}
	\label{fig:Test_7_Doorway_Velocity}
\end{figure}

\clearpage

\section{Test 8 Figures}
\label{subsec:Test_8_Figures}

\subsection{Temperature}
\label{subsec:Temperature}

\begin{figure}[!ht]
	\includegraphics[width=6in]{../Figures/Temperature/FSE8_Eastside_Array}
	\caption{Test 8 (FSE8): Eastside Array}
	\label{fig:Test_8_Eastside_Array}
\end{figure}

\begin{figure}[!ht]
	\includegraphics[width=6in]{../Figures/Temperature/FSE8_Westside_Array}
	\caption{Test 8 (FSE8): Westside Array}
	\label{fig:Test_8_Westside_Array}
\end{figure}

\begin{figure}[!ht]
	\includegraphics[width=6in]{../Figures/Temperature/FSE8_Hallway_Array}
	\caption{Test 8 (FSE8): Hallway Array}
	\label{fig:Test_8_Hallway_Array}
\end{figure}

\begin{figure}[!ht]
	\includegraphics[width=6in]{../Figures/Temperature/FSE8_Doorway_Array}
	\caption{Test 8 (FSE8): Doorway Array}
	\label{fig:Test_8_Doorway_Array}
\end{figure}

\begin{figure}[!ht]
	\includegraphics[width=6in]{../Figures/Temperature/Suppression_FSE8_Eastside_Array}
	\caption{Test 8 (FSE8): Suppression Eastside Array}
	\label{fig:Test_8_Suppression_Eastside_Array}
\end{figure}

\begin{figure}[!ht]
	\includegraphics[width=6in]{../Figures/Temperature/Suppression_FSE8_Westside_Array}
	\caption{Test 8 (FSE8): Suppression Westside Array}
	\label{fig:Test_8_Suppression_Westside_Array}
\end{figure}

\begin{figure}[!ht]
	\includegraphics[width=6in]{../Figures/Temperature/Suppression_FSE8_Hallway_Array}
	\caption{Test 8 (FSE8): Suppression Hallway Array}
	\label{fig:Test_8_Suppression_Hallway_Array}
\end{figure}

\begin{figure}[!ht]
	\includegraphics[width=6in]{../Figures/Temperature/Suppression_FSE8_Doorway_Array}
	\caption{Test 8 (FSE8): Suppression Doorway Array}
	\label{fig:Test_8_Suppression_Doorway_Array}
\end{figure}

\subsection{Heat Flux}
\label{subsec:Heat_Flux}

\begin{figure}[!ht]
	\includegraphics[width=6in]{../Figures/Heat_Flux/FSE_Test_8_Heat_Flux_Eastside}
	\caption{Test 8 (FSE8): Eastside Heat Flux}
	\label{fig:Test_8_Eastside_Heat_Flux}
\end{figure}

\begin{figure}[!ht]
	\includegraphics[width=6in]{../Figures/Heat_Flux/FSE_Test_8_Heat_Flux_Westside}
	\caption{Test 8 (FSE8): Westside Heat Flux}
	\label{fig:Test_8_Westside_Heat_Flux}
\end{figure}

\begin{figure}[!ht]
	\includegraphics[width=6in]{../Figures/Heat_Flux/FSE_Test_8_Heat_Flux_Hallway}
	\caption{Test 8 (FSE8): Hallway Heat Flux}
	\label{fig:Test_8_Hallway_Heat_Flux}
\end{figure}

\begin{figure}[!ht]
	\includegraphics[width=6in]{../Figures/Heat_Flux/FSE_Test_8_Heat_Flux_Near_Fire_Room}
	\caption{Test 8 (FSE8): Heat Flux Near Fire Room}
	\label{fig:Test_8_Heat_Flux_Near_Fire_Room}
\end{figure}

\subsection{Velocity}
\label{subsec:Velocity}

\begin{figure}[!ht]
	\includegraphics[width=6in]{../Figures/Velocity/FSE_Test_8_Hallway_Velocity}
	\caption{Test 8 (FSE8): Hallway Velocity}
	\label{fig:Test_8_Hallway_Velocity}
\end{figure}

\begin{figure}[!ht]
	\includegraphics[width=6in]{../Figures/Velocity/FSE_Test_8_Doorway_Velocity}
	\caption{Test 8 (FSE8): Doorway Velocity}
	\label{fig:Test_8_Doorway_Velocity}
\end{figure}

\clearpage

\chapter{References}
\label{chap:References}

\bibliography{../../../Bibliography/FDS_refs,../../../Bibliography/FDS_general,references}

	
\end{document}

