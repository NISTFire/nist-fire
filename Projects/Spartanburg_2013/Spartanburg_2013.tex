\documentclass{article}
\usepackage[utf8]{inputenc}

\title{Spartanburg 2013 Test Descriptions}
\author{Keith Stakes}


\begin{document}

\maketitle

\section{214 Folsom Street}

The first test house was a single story, single family residential structure of type 5 construction addressed as 214 Folsom Street.  The single story structure The non-burn rooms had a limited-combustible finish comprised of either paper-covered gypsum board or a finished plaster board.  The burn rooms had a combustible finish comprised of OSB sheets for wall linings.  Two tests were conducted in this structure: 1) Living room fire with exterior attack and 2) Bedroom fire with exterior attack.

\subsection{Test 1: Living Room Fire}

Ignition begins on a modern-style sofa in the living room on the A/D corner of the structure with all doors and windows closed. The fire grew uninhibited and reach the vent-limited stage. Fire fighting crew ventilated the living room window on the D side of the structure.The fire was allowed to stabilize with the new ventilation opening. The fire fighting crew then applied water, via a straight stream flow, from the exterior into the D side window for suppression.  After the exterior suppression, the front living room window on the A side of the structure was ventilated.  Water was then applied from the exterior on the A side into the living room window for further suppression. The fire fighting crew then opened the front door and made entry to complete suppression.

\subsection{Test 2: Bedroom Fire}

The fire was ignited on a modern-style soda in the bedroom located near the B/C corner of the structure.  The living room used during the previous burn was isolated from the remaining portion of the building for this test to not alter the results. The fire was allowed to grow uninhibited and reach the vent-limited stage. The fire self-vented the upper portion of the bedroom window on the B side of the structure.  The fire fighting crew ventilated a window on the D side of the building in the adjacent room located near the C/D corner.  The fire then self-vented the remaining portion of the window in the room of origin.  The small window into the kitchen on the B side of the structure was then ventilated.  Shortly thereafter, the fire fighting crew made an exterior attack via a narrow fog into the bedroom from the self-vented B side window.

\end{document}
